\documentclass[presentation]{beamer}
\usepackage[utf8]{inputenc}
\usepackage[T1]{fontenc}
\usepackage{fixltx2e}
\usepackage{graphicx}
\usepackage{longtable}
\usepackage{float}
\usepackage{wrapfig}
\usepackage{rotating}
\usepackage[normalem]{ulem}
\usepackage{amsmath}
\usepackage{textcomp}
\usepackage{marvosym}
\usepackage{wasysym}
\usepackage{amssymb}
\usepackage{hyperref}
\tolerance=1000
\usepackage{graphicx} \DeclareMathOperator{\argmin}{argmin}

\newcommand{\under}[2]{\underset{\scriptscriptstyle#1}{#2}}


\usetheme{simple}
\usecolortheme{}
\usefonttheme{serif}
\useinnertheme{}
\useoutertheme{}
\author{Talon Chandler}
\date{February 9, 2017}
\title{Update On 3D Orientation Reconstruction}

\begin{document}

\maketitle
\begin{frame}[label=sec-1]{Polarized diSPIM data---Summer 2017}
Since last time:
\begin{itemize}
\item Much faster reconstructions. Reconstructed 350K voxels in 1 min on my
  laptop. Bottlenecks are file I/O (not much I can do here) and visualization
  (chokes a bit on my computer, but live renders on diSPIM computer and RCC).
\item Previously I was just reconstructing the orientation in each voxel. Now
  I'm reconstructing the orientation and the ``number'' of fluorophores in each
  voxel. In each voxel I assign an oriented glyph scaled by the number of
  fluorophores in that voxel.
\end{itemize}
\end{frame}

\begin{frame}[label=sec-1]{Polarized diSPIM data---Summer 2017}
 \begin{center}
   \includegraphics[width=0.8\textwidth]{figs/last_frame.png}
 \end{center}
  Fly-around video here: \url{https://www.dropbox.com/s/szvssvu9fzxsajj/orientation-anim.avi?dl=0}
\end{frame}

\begin{frame}[label=sec-1]{Polarized diSPIM data---Summer 2017}
Notes:
\begin{itemize}
\item The video shows the entire volume 68$\times$108$\times$46 $\mu$m${}^3$.
\item I reconstructed ~10$\times$ as many voxels as the ones shown in the video then
  thresholded the smallest ones away. In Paraview I can change the threshold
  with a slider and choose slice planes quickly. 
\item I'd estimate that a majority of voxel orientations align parallel with the
  actin fibers within 10 degrees. The fiber with 25-30 degree error that we saw
  before is one of the worst cases. The orientations also clearly follow the
  edges of the cells, although the individual fibers are not resolvable in these
  regions. I'm still skeptical of our data ordering, but the results are
  encouraging.
  \end{itemize}
\end{frame}

\begin{frame}{PBT Film via Fred Lanni}
  poly(1,4-phenylene-2,6-benzo-bis-thiazole)
   \begin{center}
     \includegraphics[width=0.4\textwidth]{figs/snap.png}
   \end{center}
   \begin{itemize}
   \item \textbf{Goal}: evaluate this sample as a test specimen for 3D reconstructions
   \item Broad excitation and emission spectra
   \item Stable and insoluble
   \item No published data on fluorescence anisotropy---only Fred's verbal reports
   \end{itemize}
\end{frame}

\begin{frame}{PBT Film, Single-view FluoPolscope\\20$\times$/0.5 NA, 325$\times$325 $\mu$m${}^2$ \\February 6, 2018}
   \begin{center}
     \includegraphics[width=0.6\textwidth]{figs/capture.png}
   \end{center}
\end{frame}

\begin{frame}{PBT Film, Single-view FluoPolscope\\20$\times$/0.5 NA, 325$\times$325 $\mu$m${}^2$ \\February 6, 2018}
  Notes:
  \begin{itemize}
  \item I sandwiched the PBT film between a slide and cover slip so the fibers are 
    approximately flat.
  \item I calculated the orientation lines using the existing FluoPolscope
    algorithms and overlaid the lines on the average intensity image.
  \item Good alignment along the fibers as expected (except for one pesky fiber in
    the bottom center?).
  \end{itemize}
\end{frame}

\begin{frame}{Uncontrolled polarization dual-view data\\0.8/0.8 NA, February 3, 2018}
  Notes:
  \begin{itemize}
  \item We took two datasets with the PBT film on the symmetric diSPIM without
    polarizers. The illumination light was partially polarized on one arm, and
    almost completely polarized on the other arm---I don't expect any
    quantitative results given that we didn't use polarizers.
  \item During our first collection I accidentally saturated the detector for
    most parts of sample. 
  \item During the second collection I turned down the laser power to avoid
    saturation.
  \item The two views from the first (saturated) dataset registered easily.
  \item The two views from the second (unsaturated) dataset didn't register. I
    can't visually match up any of the features in the two datasets either. This
    could be because of orientation dependence, shadowing, something else? 
  \end{itemize}
\end{frame}

\begin{frame}{Saturated data---possible registration issues?}
  \begin{center}
    \includegraphics[width=0.15\textwidth]{figs/SPIMA-regROI}
    \hspace{1em}
    \includegraphics[width=0.15\textwidth]{figs/SPIMB-regROI}    
  \end{center}
  \begin{itemize}
  \item In the saturated registered dataset I noticed that the deformed dataset
    (right) is significantly degraded compared to the undeformed dataset (left).
    I'd expect some degradation on the order of a few pixels, but this seems
    much larger. Do these results match your experience, Min?
  \end{itemize}
\end{frame}


\begin{frame}{Conclusions}
  \begin{itemize}
  \item The PBT film is a convenient sample that displays fluorescence
    anisotropy.
  \item I think it will be a useful sample for testing the diSPIM with
    polarizers. We will be able to fray the sample into all orientations, and
    we won't have to worry about labeled cells being too flat or GUVs
    being too fragile. 
  \end{itemize}
\end{frame}

\begin{frame}{Next Up}
  Next week: 
  \begin{itemize}
  \item Thesis proposal document
  \end{itemize}    
  After that: 
  \begin{itemize}
  \item Formulate joint spatio-angular restoration problem
  \item Answer lingering questions about relating these reconstructions to
    existing work. How is the ``anisotropy'' related to these reconstructions?
    Can we understand Rudolf's existing reconstructions using spherical
    harmonics?
  \end{itemize}    
\end{frame}


\end{document}