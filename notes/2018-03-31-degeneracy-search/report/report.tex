\documentclass[11pt]{article}

%%%%%%%%%%%%
% Packages %
%%%%%%%%%%%%

\hyphenpenalty=10000
\usepackage{tikz}
\usetikzlibrary{shapes,arrows}

\usepackage{tocloft}
\renewcommand\cftsecleader{\cftdotfill{\cftdotsep}}
\def\undertilde#1{\mathord{\vtop{\ialign{##\crcr
$\hfil\displaystyle{#1}\hfil$\crcr\noalign{\kern1.5pt\nointerlineskip}
$\hfil\tilde{}\hfil$\crcr\noalign{\kern1.5pt}}}}}
\usepackage{cleveref}
\usepackage{xcolor}
\usepackage[colorlinks = true,
            linkcolor = black,
            urlcolor  = black,
            citecolor = black,
            anchorcolor = black]{hyperref}
\usepackage{epstopdf}
\usepackage{braket}
\usepackage{upgreek}
\usepackage{caption}
\usepackage{booktabs}
\usepackage{subcaption}
\usepackage{amssymb,latexsym,amsmath,gensymb}
\usepackage{latexsym}
\usepackage{graphicx}
\usepackage{float}
\usepackage{enumitem}
\usepackage{pdflscape}
\usepackage{url}
\usepackage{array}
\newcolumntype{C}{>{$\displaystyle} c <{$}}
\usepackage{tikz, calc}
\usetikzlibrary{shapes.geometric, arrows, calc}
\tikzstyle{norm} = [rectangle, rounded corners, minimum width=2cm, minimum height=1cm,text centered, draw=black]
\tikzstyle{arrow} = [thick, ->, >=stealth]

\newcommand{\argmin}{\arg\!\min}
\newcommand{\me}{\mathrm{e}}
\providecommand{\e}[1]{\ensuremath{\times 10^{#1}}} 
\providecommand{\mb}[1]{\mathbf{#1}}
\providecommand{\mf}[1]{\mathbf{#1}}
\providecommand{\ro}[1]{\mathbf{\mathbf{r}}_o}
\providecommand{\so}[1]{\mathbf{\hat{s}}_1}
\providecommand{\s}[1]{\mathbf{\hat{s}}}
\providecommand{\rb}[1]{\mathbf{r}_b}
\providecommand{\rbm}[1]{r_b^{\text{m}}}
\providecommand{\rd}[1]{\mathbf{r}_d}
\providecommand{\mh}[1]{\mathbf{\hat{#1}}}
\providecommand{\bs}[1]{\boldsymbol{#1}} 
\providecommand{\intinf}{\int_{-\infty}^{\infty}}
\providecommand{\fig}[4]{
  % filename, width, caption, label
\begin{figure}[h]
 \captionsetup{width=1.0\linewidth}
 \centering
 \includegraphics[width = #2\textwidth]{#1}
 \caption{#3}
 \label{fig:#4}
\end{figure}
}

\newcommand{\tensor}[1]{\overset{\text{\tiny$\leftrightarrow$}}{\mb{#1}}}
\newcommand{\tunderbrace}[2]{\underbrace{#1}_{\textstyle#2}}
\providecommand{\figs}[7]{
  % filename1, filename2, caption1, caption2, label1, label2, shift
\begin{figure}[H]
\centering
\begin{minipage}[b]{.45\textwidth}
  \centering
  \includegraphics[width=1.0\linewidth]{#1}
  \captionsetup{justification=justified, singlelinecheck=true}
  \caption{#3}
  \label{fig:#5}
\end{minipage}
\hspace{2em}
\begin{minipage}[b]{.45\textwidth}
  \centering
  \includegraphics[width=1.0\linewidth]{#2}
  \vspace{#7em}
  \captionsetup{justification=justified}
  \caption{#4}
  \label{fig:#6}
\end{minipage}
\end{figure}
}
\makeatletter

\providecommand{\code}[1]{
\begin{center}
\lstinputlisting{#1}
\end{center}
}

\newcommand{\crefrangeconjunction}{--}
%%%%%%%%%%%
% Spacing %
%%%%%%%%%%%
% Margins
\usepackage[
top    = 1.5cm,
bottom = 1.5cm,
left   = 1.5cm,
right  = 1.5cm]{geometry}

% Indents, paragraph space
%\usepackage{parskip}
\setlength{\parskip}{1.5ex}

% Section spacing
\usepackage{titlesec}
\titlespacing*{\title}
{0pt}{0ex}{0ex}
\titlespacing*{\section}
{0pt}{0ex}{0ex}
\titlespacing*{\subsection}
{0pt}{0ex}{0ex}
\titlespacing*{\subsubsection}
{0pt}{0ex}{0ex}

% Line spacing
\linespread{1.1}

%%%%%%%%%%%%
% Document %
%%%%%%%%%%%%
\begin{document}
\title{\vspace{-2.5em} Finding degeneracy in fluorescence orientation microscopes \vspace{-1em}}  \vspace{-2em} \author{}
\date{\vspace{-3em}\today\vspace{-2em}}
\maketitle

Consider a microscope with an angular point spread function $h_i(\s{})$---we're
assuming that the spatial and angular problems are decoupled. The forward model for
this microscope is
\begin{align}
  g_i = \int_{\mathbb{S}^2}d\s{}\, h_i(\s{})f(\s{}), \qquad i=1, 2, ..., N.
\end{align}
where $f(\s{})$ is the angular density of fluorophores and $i$ indexes $N$
measurements.

We would like to find the angular degeneracies of this microscope. In other
words, what rotations $\mb{R}$ can we apply to samples $f(\s{})$ and measure
the same data? Mathematically, we need to find the set of matrices $\mb{R}$ and
functions $f(\s{})$ that satisfy
\begin{align}
  \int_{\mathbb{S}^2}d\s{}\, h_i(\s{})f(\s{}) - \int_{\mathbb{S}^2}d\s{}\, h_i(\s{})f(\mb{R}^{-1}\s{}) = 0, \qquad i=1, 2, ..., N. \label{eq:tosolve}
\end{align}
We will start by restricting the problem to objects where all of the
fluorophores are oriented in the same direction so that
$f(\s{}) = \delta(\s{} - \so{})$. In this case we need to find the set of
directions $\so{}$ and matrices $\mb{R}$ that satisfy
\begin{align}
  h_i(\so{}) - h_i(\mb{R}^{-1}\so{}) = 0, \qquad i=1, 2, ..., N. \label{eq:tosolve2}
\end{align}

We have a set of $N$ equations that we need to solve simultaneously for
$\so{} \in \mathbb{S}^2$ and $\mb{R} \in \mathbb{SO}(3)$. As far as I can tell
there is no analytic way to solve this set of equations in the general case. I
tried expanding onto spherical and rotational harmonics, but that didn't help
because there's no easy way to relate a solution in the frequency domain to a
solution in the normal domain. Instead, we can find as many solutions as we can
``by hand'' then check our results numerically. I've used optimization packages
the can optimize on $\mathbb{S}^2\times \mathbb{SO}(3)$---my rough plan is to
seed an optimization procedure at many place on a grid in
$\mathbb{S}^2\times \mathbb{SO}(3)$ and check (not prove!) that we have all of
the solutions with those results. Other ideas welcome!

For now let's solve a simple case ``by hand''. If we have a single point
detector along the $z$ axis then our angular point spread function is given by
\begin{align}
  h^{(\mh{z})}(\so{}) = 1 - (\so{}\cdot\mh{z})^2. \label{eq:zpsf}
\end{align}
This is the first time I'm expressing an angular point spread function in vector
notation, but it's easy to check that this reduces to the familiar $h^{(\mh{z})}(\so{}) = \sin^2\vartheta$ if you choose spherical coordinates so that $\so{} = \cos\varphi\sin\vartheta\mh{x} + \sin\varphi\sin\vartheta\mh{y} + \cos\vartheta\mh{z}$. Plugging Eq. \ref{eq:zpsf} into Eq. \ref{eq:tosolve} gives
\begin{align}
  -(\so{}\cdot\mh{z})^2 + (\mb{R}^{-1}\so{}\cdot\mh{z})^2 = 0.
\end{align}
I can find the following solutions to this equation for all $\so{} \in \mathbb{S}^2$
\begin{align}
  \mb{R}_0 =
  \begin{bmatrix}
    \cos\theta& -\sin\theta & 0\\
    \sin\theta& \cos\theta & 0\\
    0&0&1\\
  \end{bmatrix} \text{for}\ 0 \leq \theta < 2\pi,
\end{align}
which is an arbitrary rotation about the $z$ axis;
\begin{align}
  \mb{R}_1 = \mb{I} + 2 \left[\mb{k}\right]_{\times}^2
\end{align}
where
\begin{align}
  \mb{k} = \frac{\so{}\times[(\so{}\cdot\mh{z})\mh{z}]}{|\so{}\times[(\so{}\cdot\mh{z})\mh{z}]|}, \qquad  \left[\mb{k}\right]_{\times} =
  \begin{bmatrix}
    0& -k_z& k_y\\
    k_z& 0 & -k_x\\
    -k_y& k_x& 0 \\
  \end{bmatrix},
\end{align}
which is a rotation that maps $\so{}$ to $-\so{}$; and
\begin{align}
  \mb{R}_2 = \mb{I} + 2(\so{}\cdot\mh{z})\sqrt{1 - (\so{}\cdot{z})^2}\left[\mb{k}\right]_{\times} + 2(\so{}\cdot\mh{z})^2\left[\mb{k}\right]_{\times}^2
\end{align}
which is a rotation that maps every $\so{}$ above the $x-y$ plane to an $\so{}$
below the $x-y$ plane and vice versa. Notice that
\begin{align}
  \mb{R}_3 =
  \begin{bmatrix}
    -1&0&0\\
    0&-1&0\\
    0&0&-1\\    
  \end{bmatrix}
\end{align}
is not a solution because $\mb{R}_3$ is not a member of $\mathbb{SO}(3)$---$\mb{R}_3$ is a member of the larger set $\mathbb{O}(3)$. This distinction is important
so that we don't double count solutions.

We can identify the solutions $\mb{R}_0$, $\mb{R}_1$, and $\mb{R}_2$ as matrix
representations of a group. This group is a \textit{Frieze group} with
\textit{dihedral symmetry}, an infinite number of possible rotations, and a
horizontal reflection axis. Using Sch\"onflies notation we say that
$D_{\infty h}$ is the symmetry group of the imaging system. 

\end{document}
