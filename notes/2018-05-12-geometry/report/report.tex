
\documentclass[11pt]{article}

%%%%%%%%%%%%
% Packages %
%%%%%%%%%%%%
\usepackage[dvipsnames]{xcolor}
\hyphenpenalty=10000
\usepackage{tikz}
\usetikzlibrary{shapes,arrows}
\usepackage{tocloft}
\renewcommand\cftsecleader{\cftdotfill{\cftdotsep}}
\def\undertilde#1{\mathord{\vtop{\ialign{##\crcr
$\hfil\displaystyle{#1}\hfil$\crcr\noalign{\kern1.5pt\nointerlineskip}
$\hfil\tilde{}\hfil$\crcr\noalign{\kern1.5pt}}}}}
\usepackage{cleveref}
\usepackage{xcolor}
\usepackage[colorlinks = true,
            linkcolor = black,
            urlcolor  = blue,
            citecolor = black,
            anchorcolor = black]{hyperref}
\usepackage{epstopdf}
\usepackage{braket}
\usepackage{upgreek}
\usepackage{caption}
\usepackage{booktabs}
\usepackage{subcaption}
\usepackage{amssymb,latexsym,amsmath,gensymb}
\usepackage{latexsym}
\usepackage{graphicx}
\usepackage{float}
\usepackage{enumitem}
\usepackage{pdflscape}
\usepackage{url}
\usepackage{array}
\newcolumntype{C}{>{$\displaystyle} c <{$}}
\usepackage{tikz, calc}
\usetikzlibrary{shapes.geometric, arrows, calc}
\tikzstyle{norm} = [rectangle, rounded corners, minimum width=2cm, minimum height=1cm,text centered, draw=black]
\tikzstyle{arrow} = [thick, ->, >=stealth]

\newcommand{\argmin}{\arg\!\min}
\newcommand{\me}{\mathrm{e}}
\providecommand{\e}[1]{\ensuremath{\times 10^{#1}}} 
\providecommand{\mb}[1]{\mathbf{#1}}
\providecommand{\mc}[1]{\mathcal{#1}}
\providecommand{\mf}[1]{\mathbf{#1}}
\providecommand{\ro}[1]{\mathbf{\mathbf{r}}_o}
\providecommand{\so}[1]{\mathbf{\hat{s}}_o}
\providecommand{\rb}[1]{\mathbf{r}_b}
\providecommand{\rbm}[1]{r_b^{\text{m}}}
\providecommand{\rd}[1]{\mathbf{r}_d}
\providecommand{\mh}[1]{\mathbf{\hat{#1}}}
\providecommand{\mbb}[1]{\mathbb{#1}}
\providecommand{\bs}[1]{\boldsymbol{#1}} 
\providecommand{\intinf}{\int_{-\infty}^{\infty}}
\providecommand{\fig}[4]{
  % filename, width, caption, label
\begin{figure}[h]
 \captionsetup{width=1.0\linewidth}
 \centering
 \includegraphics[width = #2\textwidth]{#1}
 \caption{#3}
 \label{fig:#4}
\end{figure}
}

\makeatletter
\renewcommand*\env@matrix[1][*\c@MaxMatrixCols c]{%
  \hskip -\arraycolsep
  \let\@ifnextchar\new@ifnextchar
  \array{#1}}
\makeatother

\newcommand{\tensor}[1]{\overset{\text{\tiny$\leftrightarrow$}}{\mb{#1}}}
\newcommand{\tunderbrace}[2]{\underbrace{#1}_{\textstyle#2}}
\providecommand{\figs}[7]{
  % filename1, filename2, caption1, caption2, label1, label2, shift
\begin{figure}[H]
\centering
\begin{minipage}[b]{.45\textwidth}
  \centering
  \includegraphics[width=1.0\linewidth]{#1}
  \captionsetup{justification=justified, singlelinecheck=true}
  \caption{#3}
  \label{fig:#5}
\end{minipage}
\hspace{2em}
\begin{minipage}[b]{.45\textwidth}
  \centering
  \includegraphics[width=1.0\linewidth]{#2}
  \vspace{#7em}
  \captionsetup{justification=justified}
  \caption{#4}
  \label{fig:#6}
\end{minipage}
\end{figure}
}
\makeatletter

\providecommand{\code}[1]{
\begin{center}
\lstinputlisting{#1}
\end{center}
}

\newcommand{\crefrangeconjunction}{--}
%%%%%%%%%%%
% Spacing %
%%%%%%%%%%%
% Margins
\usepackage[
top    = 1.5cm,
bottom = 1.5cm,
left   = 1.5cm,
right  = 1.5cm]{geometry}

% Indents, paragraph space
%\usepackage{parskip}
\setlength{\parskip}{1.5ex}

% Section spacing
\usepackage{titlesec}
\titlespacing*{\title}
{0pt}{0ex}{0ex}
\titlespacing*{\section}
{0pt}{0ex}{0ex}
\titlespacing*{\subsection}
{0pt}{0ex}{0ex}
\titlespacing*{\subsubsection}
{0pt}{0ex}{0ex}

% Line spacing
\linespread{1.1}

%%%%%%%%%%%%
% Document %
%%%%%%%%%%%%
\begin{document}
\title{\vspace{-2.5em} Geometry of anisotropic media and polarized light
  microscopy data \vspace{-1em}} \author{Talon Chandler, Min Guo, Hari Shroff,
  Rudolf Oldenbourg, Patrick La Rivi\`ere}
\date{\vspace{-1em}\today\vspace{-1em}}
\maketitle
\section{Introduction}
We describe the geometry of object space and data space for a large class of
polarized light microscopes. We briefly examine the mappings between object
and data spaces. Finally, we consider the transfer functions that can simplify
the mappings.

We denote the set of points on the $n$-dimensional real line $\mbb{R}^n$, the
set of points on the $n$-dimensional sphere $\mbb{S}^n$, and the set of
$n\times n$ Hermitian matrices $\mbb{H}^n$.

\section{Object spaces}
In fluorescence microscopy, the object is a three-dimensional field of oriented
fluorophores which can be described by a member of the set
$\mbb{L}_2(\mbb{R}^3\times\mbb{S}^2)$---the set of square integrable functions that
assign a scalar value to each direction at each point in 3D space.

In anisotropic microscopy, the object is a three-dimensional field of
anisotropic objects. Each point in the field can be described by its
generalized Jones matrix \cite{ortega13}---a $3\times3$ Hermitian matrix.
Although tempting, the object space is \textbf{not}
$\mbb{L}_2(\mbb{R}^3 \times \mbb{H}^3)$---the set of square integrable functions
that assigns a scalar value to each Hermitian matrix at each point in 3D
space---because the object only has a single Hermitian matrix at each point.
Instead, we will denote the object space as
$\mbb{L}_2(\mbb{R}^3; \mbb{H}^3)$---the set of square integrable functions that
assign a Hermitian matrix to each point in 3D space. Using this notation, the
objects in fluorescence microscopy would be members of the set
$\mbb{L}_2(\mbb{R}^3\times\mbb{S}^2; \mbb{H}^1)$ because $\mbb{H}^1$ are
scalars. Note that Barrett \cite{barrett2004} assumes scalar-valued functions so
the second argument of $\mbb{L}_2$ is assumed to be $\mbb{H}^1$. Meanwhile,
Sharafutdinov \cite{shara1994} makes no assumptions, so he only uses the second
argument and drops the first.

If we don't constrain the symmetric matrices, then
$\mbb{L}_2(\mbb{R}^3; \mbb{H}^3)$ can describe fields that display linear and
circular birefringence and dichroism. We often assume or know that our object
displays only birefringence or dichroism, and this prior can be formulated as a
constraint on the Hermitian matrix. Therefore, birefringence microscopy is a
subset of anisotropy microscopy. We will revisit these constraints later.

Also, we often assume or know that a birefringent object is uniaxial. To make
a Hermitian matrix describe a uniaxial object we constrain it to be rank 2.

\section{Data space}
The data space for a multiview microscope with variable polarizers on the
detection or illumination arm is a member of
$\mbb{V} = \mbb{L}_2(\mbb{S}^1\times\mbb{S}^2\times\mbb{R}^3)$ where the
$\mbb{S}^1$ dimension specifies the orientation of the detection or illumination
polarizer, the $\mbb{S}^2$ dimensions specify the optical axis of the view, and
the $\mbb{R}^3$ dimensions specifies the position on the 2D detector with 1D
defocusing. All of the polarized light microscopes we have considered so far
take discrete samples of $\mbb{V}$. Notice that $\mbb{V}$ is independent of the
object---we can use the same data space for fluorescence and anisotropy imaging.

Rudolf's existing PolScope designs take a single sample of the optical axis
dimensions (they are single view microscopes), four or five samples of the
polarizer dimension (one for each polarizer setting), and many samples of the
detector dimension (one for each pixel at each defocus position).

Our current diSPIM design takes two samples of the optical axis dimensions, four
samples of the polarizer dimension (eight would be double counting), and many
samples of the detector dimension. 

Light-field designs take a dense sampling of the optical axis dimension over a
small range of angles at a single detector plane. Locally, $\mbb{S}^2$ is
approximated by $\mbb{R}^2$, and one slice of $\mbb{R}^3$ is $\mbb{R}^2$, so a
polarized light-field microscope's data space is approximately
$\mbb{L}_2(\mbb{S}^1\times\mbb{R}^4)$. This identification points towards a link
between our multiview analysis and Ng's 4-dimensional light-field analysis
\cite{ng2005}.

If we would like to consider microscopes with polarizers on the illumination and
detection arms, we can add an extra $\mbb{S}^1$ dimension and the data space
becomes $\mbb{L}_2(\mbb{S}^1\times\mbb{S}^1\times\mbb{S}^2\times\mbb{R}^3)$.

\section{Mapping object space to data space}
In fluorescence microscopy we can describe the relationship between the object
and data spaces using an integral transform
\begin{align}
    [\mathcal{H}f](\hat{\mb{s}}_v, \hat{\mb{p}}, \rd{}) = \int_{\mbb{S}^2}d\so{}\int_{\mbb{R}^3}d\ro{}\, h(\rd{} -\ro{}, \so{}; \hat{\mb{s}}_v, \hat{\mb{p}})f(\ro{}, \so{}).\label{eq:beginning}
\end{align}
where $h(\rd{} -\ro{}, \so{}; \hat{\mb{s}}_v, \hat{\mb{p}})$ is the kernel of
the integral transform (when data space is larger than $\mbb{L}_2(\mbb{R}^2)$ I
think that ``kernel'' is more accurate than ``point spread function''). More
work is required to write the complete kernel for arbitrary viewing directions $\hat{\mb{s}}_v$.
The full kernel will be necessary for analyzing fluorescence light-field
microscopes.

I think we can make progress in anisotropic microscopy if we describe the
data as a spatially filtered tensor ray transform of the object. If we
illuminate the object incoherently (using a wide condenser back aperture under
K\"ohler illumination) and the sample is weakly anisotropic, then the electric
field at the output of the object will be the (input polarization dependent)
tensor ray transform of the object. Finally, we can propagate the output
electric field to the detector using the formalism we've developed for
fluorescence microscopy---this operation is essentially a low-pass spatial
filter.

Sharafutdinov 2.1.10 \cite{shara1994} gives an expressions for the tensor ray
transform that is a generalization of the Radon transform
\begin{align}
  [\mathcal{I}f](x, \xi) = \int \langle f(x + t\xi), \xi\rangle dt. 
\end{align}
I'm still digesting the details of how to evaluate this integral in practice,
but the analogy with the Radon transform and the physical meaning are clear.

The mapping between the object space and data space will look something like
\begin{align}
    [\mathcal{H}f](\hat{\mb{s}}_v, \hat{\mb{p}}, \rd{}) = \int_{\mbb{S}^2}d\so{}\int_{\mbb{R}^2}d\ro{}\, h(\rd{} -\ro{}, \so{}; \hat{\mb{s}}_v, \hat{\mb{p}})[\mathcal{I}f](\ro{}, \so{}).\label{eq:final}
\end{align}
One difference between Eq. \ref{eq:beginning} and Eq. \ref{eq:final} is the
interpretation of the variable $\so{}$. In the fluorescence case, $\so{}$ refers
to the orientation of the dipole axis of a fluorophore. In the anisotropy case,
$\so{}$ refers to the propagation direction of a light ray.

The incoherence of the light incident on the sample is an important simplifying
assumption. If the incident light is partially coherent, then the relationship
between the object and data is a bilinear integral transform (see Barrett 5.2
\cite{barrett2004}, Goodman 7.4 \cite{goodman1985}). This type of relationship
does not allow for a simple transfer function analysis, but a transmission
cross-coefficient analysis is still possible.

\section{Transfer functions}
All of the mappings we considered in the previous section are linear, so we can
rewrite them in terms of the transfer function. 

The first task is to find the geometry and dimension of the transfer function.
The transfer function must account for each dimension of the object and data
spaces while not double counting invariant dimensions between the spaces. For
example, a fluorescence microscope maps objects in
$\mbb{L}(\mbb{S}^2\times\mbb{R}^2)$ to data in
$\mbb{L}_2(\mbb{S}^1\times\mbb{S}^2\times\mbb{R}^2)$. The two spatial dimensions
are shift invariant, so we assign a single continuous spatial frequency index
$\bs{\nu}$ to those dimensions. I expect (calculation in progress) that the two
orientation dimensions are rotation invariant, so we can assign the indices $l$
and $m$ to those dimensions. Finally, we can assign the index $k$ to the data
space $\mbb{S}^1$ polarizer dimension. Therefore, the transfer function will be
of the form
\begin{align}
  H_{l,k}^{m}(\bs{\nu}) = \int_{\mbb{S}^1}d\hat{\mb{p}}\, z_k(\hat{\mb{p}}) \int_{\mbb{S}^2}d\so{}\, y_l^m(\so{})\int_{\mbb{R}^2}d\rd{}\, \me^{-i2\pi\bs{\nu}\cdot\rd{}} h(\rd{} -\ro{}, \so{}; \hat{\mb{s}}_v, \hat{\mb{p}}),
\end{align}
where $y_l^m(\so{})$ are the real spherical harmonics and $z_k(\hat{\mb{p}})$
are the real circular harmonics.

I expect to find similar transfer functions for the anisotropy imaging case,
but I can't make much more progress until I spend more time understanding the
tensor ray transform. Sharafutdinov 2.1.15 \cite{shara1994} gives the
relationship between the an object's tensor ray transform and its Fourier
transform (a generalization of the Fourier-Slice theorem), but I don't
understand all of the notation yet.

I suspect that the nine Gell-Mann matrices will play a role in the Fourier
analysis of anisotropy microscopes \cite{ortega13}. The Gell-Mann matrices are
an orthonormal basis for $3\times 3$ Hermitian matrices $\mbb{H}^3$, so they
take the role of the spherical harmonics in fluorescence imaging. The Gell-Mann
matrices have direct physical interpretations in terms of birefringence and
dichroism of the object, so we can formulate physical priors about the object as
constraining specific Gell-Mann matrices to zero.

\bibliography{report}{}
\bibliographystyle{unsrt}

\end{document}
