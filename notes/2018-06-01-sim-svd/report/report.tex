
\documentclass[11pt]{article}

%%%%%%%%%%%%
% Packages %
%%%%%%%%%%%%
\usepackage[dvipsnames]{xcolor}
\hyphenpenalty=10000
\usepackage{tikz}
\usetikzlibrary{shapes,arrows}

\usepackage{tocloft}
\renewcommand\cftsecleader{\cftdotfill{\cftdotsep}}
\def\undertilde#1{\mathord{\vtop{\ialign{##\crcr
$\hfil\displaystyle{#1}\hfil$\crcr\noalign{\kern1.5pt\nointerlineskip}
$\hfil\tilde{}\hfil$\crcr\noalign{\kern1.5pt}}}}}
\usepackage{cleveref}
\usepackage{xcolor}
\usepackage[colorlinks = true,
            linkcolor = black,
            urlcolor  = blue,
            citecolor = black,
            anchorcolor = black]{hyperref}
\usepackage{epstopdf}
\usepackage{braket}
\usepackage{upgreek}
\usepackage{caption}
\usepackage{booktabs}
\usepackage{subcaption}
\usepackage{amssymb,latexsym,amsmath,gensymb}
\usepackage{latexsym}
\usepackage{graphicx}
\usepackage{float}
\usepackage{enumitem}
\usepackage{pdflscape}
\usepackage{url}
\usepackage{array}
\newcolumntype{C}{>{$\displaystyle} c <{$}}
\usepackage{tikz, calc}
\usetikzlibrary{shapes.geometric, arrows, calc}
\tikzstyle{norm} = [rectangle, rounded corners, minimum width=2cm, minimum height=1cm,text centered, draw=black]
\tikzstyle{arrow} = [thick, ->, >=stealth]

\newcommand{\argmin}{\arg\!\min}
\newcommand{\me}{\mathrm{e}}
\providecommand{\e}[1]{\ensuremath{\times 10^{#1}}} 
\providecommand{\mb}[1]{\mathbf{#1}}
\providecommand{\mc}[1]{\mathcal{#1}}
\providecommand{\ro}[1]{\mathbf{r}_o}
\providecommand{\so}[1]{\mathbf{\hat{s}}_o}
\providecommand{\rb}[1]{\mathbf{r}_b}
\providecommand{\rbm}[1]{r_b^{\text{m}}}
\providecommand{\rd}[1]{\mathbf{r}_d}
\providecommand{\mh}[1]{\mathbf{\hat{#1}}}
\providecommand{\mf}[1]{\mathfrak{#1}}
\providecommand{\mbb}[1]{\mathbb{#1}}
\providecommand{\bs}[1]{\boldsymbol{#1}} 
\providecommand{\intinf}{\int_{-\infty}^{\infty}}
\providecommand{\fig}[4]{
  % filename, width, caption, label
\begin{figure}[h]
 \captionsetup{width=1.0\linewidth}
 \centering
 \includegraphics[width = #2\textwidth]{#1}
 \caption{#3}
 \label{fig:#4}
\end{figure}
}

\makeatletter
\renewcommand*\env@matrix[1][*\c@MaxMatrixCols c]{%
  \hskip -\arraycolsep
  \let\@ifnextchar\new@ifnextchar
  \array{#1}}
\makeatother

\newcommand{\tensor}[1]{\overset{\text{\tiny$\leftrightarrow$}}{\mb{#1}}}
\newcommand{\tunderbrace}[2]{\underbrace{#1}_{\textstyle#2}}
\providecommand{\figs}[7]{
  % filename1, filename2, caption1, caption2, label1, label2, shift
\begin{figure}[H]
\centering
\begin{minipage}[b]{.45\textwidth}
  \centering
  \includegraphics[width=1.0\linewidth]{#1}
  \captionsetup{justification=justified, singlelinecheck=true}
  \caption{#3}
  \label{fig:#5}
\end{minipage}
\hspace{2em}
\begin{minipage}[b]{.45\textwidth}
  \centering
  \includegraphics[width=1.0\linewidth]{#2}
  \vspace{#7em}
  \captionsetup{justification=justified}
  \caption{#4}
  \label{fig:#6}
\end{minipage}
\end{figure}
}
\makeatletter

\providecommand{\code}[1]{
\begin{center}
\lstinputlisting{#1}
\end{center}
}

\newcommand{\crefrangeconjunction}{--}
%%%%%%%%%%%
% Spacing %
%%%%%%%%%%%
% Margins
\usepackage[
top    = 1.5cm,
bottom = 1.5cm,
left   = 1.5cm,
right  = 1.5cm]{geometry}

% Indents, paragraph space
%\usepackage{parskip}
\setlength{\parskip}{1.5ex}

% Section spacing
\usepackage{titlesec}
\titlespacing*{\title}
{0pt}{0ex}{0ex}
\titlespacing*{\section}
{0pt}{0ex}{0ex}
\titlespacing*{\subsection}
{0pt}{0ex}{0ex}
\titlespacing*{\subsubsection}
{0pt}{0ex}{0ex}

% Line spacing
\linespread{1.1}

%%%%%%%%%%%%
% Document %
%%%%%%%%%%%%
\begin{document}
\title{\vspace{-2.5em} Singular value decomposition of single view\\
  structured illumination
  microscopes\vspace{-1em}} % \author{Talon Chandler, Min Guo, Hari
  % Shroff, Rudolf Oldenbourg, Patrick La Rivi\`ere}
\date{\vspace{-3em}\today\vspace{-1em}}
\maketitle
\section{Introduction}
In these notes we will find the kernel, transfer function, and singular value
decomposition of epi-illumination and epi-detection single view structured
illumination microscopes. We will model the relationship between the spatial
density of fluorophores in a two-dimensional sample---a member of
$\mbb{L}_2(\mbb{R}^2)$---and the complete data space---a member of
$\mbb{L}_2(\mbb{R}^2 \times \mbb{R}^2 \times \mbb{S}^1)$. We will show that
structured illumination microscopes sample this five-dimensional space---pixels
sample a pair of spatial dimensions $\mbb{R}^2$, illumination pattern
orientations and spatial frequencies sample another pair of dimensions
$\mbb{R}^2$, and the phase of the illumination pattern samples a circular
dimension $\mbb{S}^1$.

\section{Kernel}
Structured illumination microscopes interfere two coherent illumination beams to
create a sinusoid squared excitation pattern in the sample. The excitation
kernel takes the form
\begin{align}
  h_{\text{exc}}(\ro{}, \mb{k}, \phi) = \cos^2(\mb{k}\cdot\ro{} - \phi) = \frac{1}{4}\me{}^{i2(\mb{k} \cdot\ro{} - \phi)} + \frac{1}{2} + \frac{1}{4}\me{}^{-i2(\mb{k}\cdot\ro{} - \phi)},\label{eq:exc}
\end{align}
where $\ro{}$ is the two-dimensional position in the sample, $\mb{k}$ is the
wave vector of the illumination pattern, and $\phi$ is the phase of the
illumination pattern. The detection process of the microscope is shift
invariant, so we can model the detection process with a detection kernel
$h_{\text{det}}(\ro{})$. In these notes we will use the paraxial detection
model
\begin{align}
  h_{\text{det}}(\ro{}) = \left[\frac{J_1(2\pi\nu_o |\ro{}|)}{\pi\nu_o |\ro{}|}\right]^2,\label{eq:det}
\end{align}
where $J_1(\cdot)$ is a first-order Bessel function of the first kind, and
$\nu_o = \text{NA}/\lambda$ is the coherent cutoff frequency. We can easily plug
in a more sophisticated detection model if we need to.

We can model the relationship between the object and the data using an integral
transform
\begin{align}
  g(\rd{}, \mb{k}, \phi) = \left[\mc{H}f\right](\rd{}, \mb{k}, \phi) = \int_{\mbb{R}^2}d\ro{}\, h_{\text{exc}}(\ro{}, \mb{k}, \phi)h_{\text{det}}(\rd{} - \ro{})f(\ro{}), 
\end{align}
where $g(\rd{}, \mb{k}, \phi)$ is the five-dimensional data set and $f(\ro{})$ is
the two-dimensional density of fluorophores. 

The adjoint of the forward operator is given by
\begin{align}
  f(\ro{}) = [\mc{H}^{\dagger}g](\ro{}) = \int_{\mbb{S}^1}d\phi\int_{\mbb{R}^2}d\mb{k}\int_{\mbb{R}^2}d\rd{}\, h_{\text{exc}}(\ro{}, \mb{k}, \phi)h_{\text{det}}(\rd{} - \ro{})g_v(\rd{}, \mb{k}, \phi). 
\end{align}
\section{Transfer function}
We can rewrite the forward and adjoint operators in the frequency domain as
\begin{align}
  G_n(\bs{\nu}, \mb{k}) &= H_{n}(\bs{\nu}, \mb{k})F(\bs{\nu}), \label{eq:fwd} \\
  F(\bs{\nu}) &= \int_{\mbb{R}^2} d\mb{k}\sum_{n=0}^{\infty} H_{n}(\bs{\nu}, \mb{k})G_{n}(\bs{\nu}, \mb{k}),\label{eq:adj}
\end{align}
where
\begin{align}
  G_n(\bs{\nu}, \mb{k}) &= \int_{\mbb{S}^1}d\phi\, \me{}^{i2\pi\phi n}\int_{\mbb{R}^2}d\rd{}\, \me{}^{i2\pi\rd{}\cdot\bs{\nu}} g(\rd{}, \mb{k}, \phi),\\
  H_n(\bs{\nu}, \mb{k}) &= \int_{\mbb{S}^1}d\phi\, \me{}^{i2\pi\phi n}\int_{\mbb{R}^2}d\rd{}\, \me{}^{i2\pi\ro{}\cdot\bs{\nu}} h_{\text{exc}}(\ro{}, \mb{k}, \phi)h_{\text{det}}(\ro{}), \label{eq:trans}\\  
  F(\bs{\nu}) &= \int_{\mbb{R}^3}d\ro{}\, \me{}^{i2\pi\ro{}\cdot\bs{\nu}}f(\ro{}).
\end{align}
Notice that we have taken the Fourier transform of object space variable $\ro{}$
and taken the Fourier series of the data space variable $\phi$, but we have left
the data space variable $\mb{k}$ alone because it is ``natively'' in the
frequency domain. 

If we plug the kernels in Eqs. \ref{eq:exc} and \ref{eq:det} into Eq.
\ref{eq:trans} and evaluate the integrals we find the transfer function is
\begin{align}
  H_n(\bs{\nu}, \mb{k}) &= \frac{1}{4}H_{\text{det}}(\bs{\nu} - 2\mb{k})\delta_{n,2} + \frac{1}{2}H_{\text{det}}(\bs{\nu})\delta_{n,0} + \frac{1}{4}H_{\text{det}}(\bs{\nu} + 2\mb{k})\delta_{n,-2},
\end{align}
where $\delta_{n,m}$ is a Kronecker delta,
\begin{align}
  H_{\text{det}}(\bs{\nu}) = \frac{2}{\pi}\left[\arccos\left(\frac{|\bs{\nu}|}{2\nu_o}\right) - \frac{|\bs{\nu}|}{2\nu_o}\sqrt{1 - \left(\frac{|\bs{\nu}|}{2\nu_o}\right)^2}\right]\Pi\left(\frac{|\bs{\nu}|}{2\nu_o}\right),
\end{align}
is the detection transfer function for a fluorescence microscope without
structured illumination, and $\Pi\left(\cdot\right)$ is a rect function. 

\section{Singular value decomposition}
To find the singular value decomposition of the imaging operator we need to
solve the eigenvalue problem
\begin{align}
  [\mc{H}^{\dagger}\mc{H}]u_{\bs{\rho}}(\ro{}) = \mu_{\bs{\rho}}u_{\bs{\rho}}(\ro{}), \label{eq:eig}
\end{align}
where $u_{\bs{\rho}}(\ro{})$ are the object space singular functions of the
system. This eigenvalue problem is easily solved in the frequency domain, so we
write the object-space singular functions as
\begin{align}
  u_{\bs{\rho}}(\ro{}) = U(\bs{\rho})e^{i2\pi\ro{}\cdot\bs{\rho}}. \label{eq:ftu}
\end{align}
Plugging Eqs. \ref{eq:fwd}, \ref{eq:adj}, and \ref{eq:ftu} into Eq. \ref{eq:eig}
gives a new eigenvalue equation
\begin{align}
  \int_{\mbb{R}^2}d\mb{k}\sum_{n = 0}^{\infty}H_{n}(\bs{\rho}, \mb{k})H_{n}(\bs{\rho}, \mb{k})U(\bs{\rho}) = \mu_{\bs{\rho}}U(\bs{\rho}).
\end{align}
Therefore, the eigenvalues are given by
\begin{align}
  \mu_{\bs{\rho}} = \int_{\mbb{R}^2}d\mb{k}\sum_{n = 0}^{\infty}H^2_{n}(\bs{\rho}, \mb{k}),
\end{align}
and the eigenvectors are constants
\begin{align}
  U(\bs{\rho}) = 1. 
\end{align}
The singular values are given by the square root of the eigenvalues
\begin{align}
  \sigma_{\bs{\rho}} = \sqrt{\int_{\mbb{R}^2}d\mb{k}\sum_{n = 0}^{\infty}H^2_{n}(\bs{\rho}, \mb{k})},
\end{align}
and the object-space singular functions are the usual Fourier components of the
object
\begin{align}
  u_{\bs{\rho}}(\ro{}) = e^{i2\pi\ro{}\cdot\bs{\rho}}. 
\end{align}
Finally, the data space singular functions are given by
\begin{align}
  v_{\bs{\rho}}(\rd{}, \mb{k}, \phi) =\frac{1}{\sigma_{\bs{\rho}}}\mc{H}_{\text{v}}u_{\bs{\rho}}(\ro{}) = \frac{\sum_{n=-\infty}^{\infty}e^{i2\pi n\phi} H_{n}(\bs{\rho}, \mb{k})}{\sqrt{\int_{\mbb{R}^2}d\mb{k}\sum_{n = 0}^{\infty}H^2_{n}(\bs{\rho}, \mb{k})}}e^{i2\pi\rd{}\cdot\bs{\rho}}.
\end{align}
Looking closer at the singular value spectrum lets us see the limits of
structured illumination microscopy. If we can illuminate the sample with all
possible $\mb{k}$ vectors, then the singular spectrum would be a constant for
all $\bs{\rho}$. This is not practically possible because real microscopes are
limited by the illumination NA of the microscope. In this case, the integral is
over the illumination NA of the microscope, and we can clearly see that the
singular value spectrum will have twice the bandwidth of a widefield microscope.

Note that we can find a closed form for the singular value spectrum, but I
haven't done this. I think that the main insight from these notes is that the
singular value spectrum is the square root of the sum of the squared transfer
functions for each individual illumination pattern. This result applies to a
(more realistic) discrete sampling of $\mb{k}$ and $\phi$ space as well. 

Also note that the Fourier series expansion of the phase dimension plays a role
in the transfer function and singular value decomposition. I suspect that the
Fourier series expansion should play a role in an efficient reconstruction as
well (does it already?).

% Let's look closer at the singular value spectrum. If we could illuminate the
% sample with all possible $\mb{k}$ vectors, then the singular spectrum would be a
% constant for all $\bs{\rho}$. This is not practically possible because we are
% limited by the illumination NA of the microscope. In this case, we can 
% integrate over a much smaller region
% \begin{align}
%   \sigma_{\bs{\rho}} &= \sqrt{\int_0^{\nu_o/2}k dk\int_0^{2\pi}d\phi_k \sum_{n={-\infty}}^{\infty} H_n^2(\bs{\rho}, \mb{k})},\\
%   \sigma_{\bs{\rho}} &= \sqrt{\int_0^{\nu_o/2}k dk\int_0^{2\pi}d\phi_k \left[\frac{1}{16}H^2_{\text{det}}(\bs{\nu} - 2\mb{k}) + \frac{1}{4}H^2_{\text{det}}(\bs{\nu}) + \frac{1}{16}H^2_{\text{det}}(\bs{\nu} + 2\mb{k})\right]},\\
%   \sigma_{\bs{\rho}} &= \sqrt{\int_0^{\nu_o/2}k dk\int_0^{2\pi}d\phi_k \left[\frac{1}{8}H^2_{\text{det}}(\bs{\nu} - 2\mb{k}) + \frac{1}{4}H^2_{\text{det}}(\bs{\nu})\right]},\\ \end{align}

\end{document}

