\documentclass[11pt]{article}

%%%%%%%%%%%%
% Packages %
%%%%%%%%%%%%
\usepackage[dvipsnames]{xcolor}
\hyphenpenalty=10000
\usepackage{tikz}
\usetikzlibrary{shapes,arrows}

\usepackage{tocloft}
\usepackage[linesnumbered,ruled]{algorithm2e}
\renewcommand\cftsecleader{\cftdotfill{\cftdotsep}}
\def\undertilde#1{\mathord{\vtop{\ialign{##\crcr
$\hfil\displaystyle{#1}\hfil$\crcr\noalign{\kern1.5pt\nointerlineskip}
$\hfil\tilde{}\hfil$\crcr\noalign{\kern1.5pt}}}}}
\usepackage{bold-extra}
\usepackage{bm}
\usepackage{cleveref}
\usepackage{xcolor}
\usepackage[colorlinks = true,
            linkcolor = black,
            urlcolor  = blue,
            citecolor = black,
            anchorcolor = black]{hyperref}
\usepackage{epstopdf}
\usepackage{braket}
\usepackage{upgreek}
\usepackage{caption}
\usepackage{booktabs}
\usepackage{subcaption}
\usepackage{amssymb,latexsym,amsmath,gensymb}
\usepackage{latexsym}
\usepackage{graphicx}
\usepackage{float}
\usepackage{enumitem}
\usepackage{pdflscape}
\usepackage{url}
\usepackage{array}
\newcolumntype{C}{>{$\displaystyle} c <{$}}
\usepackage{tikz, calc}
\usetikzlibrary{shapes.geometric, arrows, calc}
\tikzstyle{norm} = [rectangle, rounded corners, minimum width=2cm, minimum height=1cm,text centered, draw=black]
\tikzstyle{arrow} = [thick, ->, >=stealth]

\newcommand{\argmin}{\arg\!\min}
\providecommand{\argmax}[1]{\underset{#1}{\mathrm{argmax}}\,}
\newcommand{\me}{\mathrm{e}}
\providecommand{\e}[1]{\ensuremath{\times 10^{#1}}}
\providecommand{\mb}[1]{\mathbf{#1}}
\providecommand{\mc}[1]{\mathcal{#1}}
\providecommand{\ro}{\mathbf{\mathfrak{r}}_o}
\providecommand{\so}{\mathbf{\hat{s}}_o}
\providecommand{\sd}{\mathbf{\hat{s}}_d}
\providecommand{\pp}{\mathbf{\hat{p}}}
\providecommand{\rb}{\mathbf{r}_b}
\providecommand{\rbm}{r_b^{\text{m}}}
\providecommand{\rd}{\mathbf{\mathfrak{r}}_d}
\providecommand{\mh}[1]{\mathbf{\hat{#1}}}
\providecommand{\mf}[1]{\mathfrak{#1}}
\providecommand{\ms}[1]{\mathsf{#1}}
\providecommand{\mbb}[1]{\mathbb{#1}}
\providecommand{\bs}[1]{\boldsymbol{#1}}
\providecommand{\tv}{\texttt{v}}
\providecommand{\tx}[1]{\text{#1}}
\providecommand{\tb}[1]{\textbf{#1}}
\providecommand{\ttt}[1]{\texttt{#1}}
\providecommand{\bv}{\bs{\nu}}
\providecommand{\bp}{\bs{\rho}}
\providecommand{\p}{\mh{p}}
\providecommand{\lmsum}{\sum_{l=0}^\infty\sum_{m=-l}^{l}}
\providecommand{\intr}[1]{\int_{\mbb{R}^{#1}}}
\providecommand{\ints}[1]{\int_{\mbb{S}^{#1}}}
\providecommand{\intinf}{\int_{-\infty}^{\infty}}
\let\originalleft\left
\let\originalright\right
\renewcommand{\left}{\mathopen{}\mathclose\bgroup\originalleft}
\renewcommand{\right}{\aftergroup\egroup\originalright}


\providecommand{\fig}[4]{
  % filename, width, caption, label
\begin{figure}[H]
 \captionsetup{width=1.0\linewidth}
 \centering
 \includegraphics[width = #2\textwidth]{#1}
 \caption{#3}
 \label{fig:#4}
\end{figure}
}

\makeatletter
\renewcommand*\env@matrix[1][*\c@MaxMatrixCols c]{%
  \hskip -\arraycolsep
  \let\@ifnextchar\new@ifnextchar
  \array{#1}}
\makeatother

\newcommand{\tensor}[1]{\overset{\text{\tiny$\leftrightarrow$}}{\mb{#1}}}
\newcommand{\tunderbrace}[2]{\underbrace{#1}_{\textstyle#2}}
\providecommand{\figs}[7]{
  % filename1, filename2, caption1, caption2, label1, label2, shift
\begin{figure}[H]
\centering
\begin{minipage}[b]{.45\textwidth}
  \centering
  \includegraphics[width=1.0\linewidth]{#1}
  \captionsetup{justification=justified, singlelinecheck=true}
  \caption{#3}
  \label{fig:#5}
\end{minipage}
\hspace{2em}
\begin{minipage}[b]{.45\textwidth}
  \centering
  \includegraphics[width=1.0\linewidth]{#2}
  \vspace{#7em}
  \captionsetup{justification=justified}
  \caption{#4}
  \label{fig:#6}
\end{minipage}
\end{figure}
}
\makeatletter

\providecommand{\code}[1]{
\begin{center}
\lstinputlisting{#1}
\end{center}
}

\newcommand{\crefrangeconjunction}{--}
%%%%%%%%%%%
% Spacing %
%%%%%%%%%%%
% Margins
\usepackage[
top    = 1.5cm,
bottom = 1.5cm,
left   = 1.5cm,
right  = 1.5cm]{geometry}

% Indents, paragraph space
%\usepackage{parskip}
\setlength{\parskip}{1.5ex}

% Section spacing
\usepackage{titlesec}
\titlespacing*{\title}
{0pt}{0ex}{0ex}
\titlespacing*{\section}
{0pt}{0ex}{0ex}
\titlespacing*{\subsection}
{0pt}{0ex}{0ex}
\titlespacing*{\subsubsection}
{0pt}{0ex}{0ex}

% Line spacing
\linespread{1.1}

%%%%%%%%%%%%
% Document %
%%%%%%%%%%%%
\begin{document}
\title{\vspace{-2.5em} Optimizing angular diSPIM measurements\vspace{-1em}} % \author{Talon Chandler, Min Guo, Hari
  % Shroff, Rudolf Oldenbourg, Patrick La Rivi\`ere}
\date{\vspace{-3em}\today\vspace{-1em}}
\maketitle

\section{Introduction}
Our goal in these notes is to find optimal angular sampling schemes for the
diSPIM with light-sheet tilting. We review the continuous angular forward model,
find the sampling schemes available to us, choose an objective function to
optimize, then search through the sampling schemes.

\section{Continuous forward model}
We briefly review the continuous angular forward model---see the
\href{https://github.com/talonchandler/polharmonic/blob/master/notes/2018-12-10-angular-svd/report/report.pdf}{previous
  notes} for a more detailed discussion. We can write the
continuous-to-continuous forward model in the form
\begin{align}
  \mb{g}_c = \mc{H}_{cc}\mb{f}_c,\label{eq:cc}
\end{align}
where $\mb{f}_c \in \mbb{L}_2(\mbb{S}^2)$ is the angular dipole density,
$\mb{g}_c \in \mbb{L}_2(\mbb{S}^2\times \mbb{S}^1)$ is the continuous irradiance
data taken by varying the detection optical axis (the $\mbb{S}^2$ dimension) and
the illumination polarizer (the $\mbb{S}^1$ dimension), and $\mc{H}_{cc}$ is a
linear continuous-to-continuous Hilbert-space operator between these spaces.

If we choose a delta-function basis for both object and data space then we can
rewrite Eq. \eqref{eq:cc} as 
\begin{align}
  g(\sd, \pp) = \ints{2}d\so\, h(\sd,\pp;\so)f(\so),\label{eq:fwd}
\end{align}
where $\so \in \mbb{S}^2$ is the object angular coordinate, $\sd \in \mbb{S}^2$
is the detection optical axis coordinate, $\pp \in \mbb{S}^1$ is the polarizer
coordinate, and $h(\sd,\pp;\so)$ is the kernel of the integral transform. We can
separate the kernel into an excitation and detection part since these processes
are incoherent
\begin{align}
  h(\sd, \pp;\so) = h^{\text{exc}}(\pp;\so)h^{\text{det}}(\sd;\so). \label{eq:kern}
\end{align}
The excitation kernel is straightforward---the squared dot product of the
polarizer orientation $\pp$ with the object angular coordinate $\so$
\begin{align}
  h^{\text{exc}}(\pp\cdot\so) \propto (\pp\cdot\so)^2. \label{eq:exckern}
\end{align}
The detection kernel requires more care since we are collecting light over the
NA of the objective. In the previous notes we started with the power along a
single ray then integrated over the aperture to find that the detection kernel is
\begin{align}
  h^{\text{det}}(\sd\cdot\so) \propto (1 - \sqrt{\cos\alpha})P_0(\sd\cdot\so) + \frac{1}{5}\left[\frac{1}{4}\sqrt{\cos\alpha}(7-3\cos(2\alpha)) - 1\right]P_2(\sd\cdot\so),
\end{align}
where $\text{NA} = n_o\sin\alpha$, and $P_{\ell}(x)$ are the Legendre
polynomials.

We can rewrite the mapping in Eq. \eqref{eq:cc} in an object-space spherical
harmonic basis as
\begin{align}
  g(\sd, \pp) = \sum_{\ell m}H_\ell^m(\sd, \pp)F_\ell^m,\label{eq:fwdsh}
\end{align}
where $F_\ell^m$ is the angular dipole spectrum given by 
\begin{align}
  F_\ell^m = \int_{\mbb{S}^2}d\so\, f(\so)Y_\ell^{m*}(\so),
\end{align}
and $H_\ell^m(\sd, \pp)$ is the kernel in this basis given by
\begin{align}
  H_\ell^m(\sd, \pp) &= \sum_{\ell' m'}\sum_{\ell'' m''}(-1)^m G_{\ell,\ell',\ell''}^{-m,m',m''}H_{\ell'}^{m',\text{exc}}(\pp)H_{\ell''}^{m'',\text{det}}(\sd),\label{eq:triple4}\\
    H_{\ell'}^{m',\text{exc}}(\pp) &\propto Y_{\ell'}^{m'*}(\pp)\left[\delta_{0\ell'} + \frac{2}{5}\delta_{2,\ell'}\right],\label{eq:dipexc}\\
    H_{\ell''}^{m'',\text{det}}(\sd) &\propto Y_{\ell''}^{m''*}(\sd)\left[(1 - \sqrt{\cos\alpha})\delta_{0,\ell''} + \left(\frac{1}{4}\sqrt{\cos\alpha}(7-3\cos(2\alpha)) - 1\right)\delta_{2,\ell''}\right].\label{eq:dipdet}
\end{align}
with the Gaunt coefficients given by 
\begin{align}
  G_{\ell,\ell',\ell''}^{m,m',m''} = \ints{2}d\so\, Y_\ell^{m}(\so)Y_{\ell'}^{m'}(\so)Y_{\ell''}^{m''}(\so).
\end{align}

It's useful to compare Eq. \eqref{eq:fwd} to Eq. \eqref{eq:fwdsh}---in a delta
function basis we need to compute an angular integral, while in the object-space
spherical harmonics basis the kernel has a finite number of terms so we can
compute it exactly. 

It's also useful to compare Eq. \eqref{eq:kern} to Eq. \eqref{eq:triple4}---in a
delta function basis the kernel is the product of the excitation and detection
kernels, while in the object-space spherical harmonic basis the kernel is a
generalized convolution of the excitation and detection kernels.

\section{Sampling schemes}
We can write the sampling operation as a continuous-to-discrete Hilbert space
operator $\mc{D}_w$
\begin{align}
  \mb{g}_d = \mc{D}_w\mb{g}_c = \mc{D}_w\mc{H}_{cc}\mb{f}_c. 
\end{align}
We can write the sampling operation in a delta function basis as
\begin{align}
  g_m = \int_{\mbb{S}^2}d\sd\int_{\mbb{S}^1}d\pp\, w_m(\sd,\pp)g(\sd,\pp).
\end{align}
where $g_m$ is the $m$th discrete measurement and $w_m(\sd,\pp)$ is the $m$th
sampling aperture. For now we will restrict ourselves to delta-function sampling
apertures
\begin{align}
  w_m(\sd,\pp) = \delta(\mh{s}_{d,m} - \sd)\delta(\mh{p}_{m} - \pp),
\end{align}
where $\mh{s}_{d,m}$ and $\mh{p}_{m}$ are the $m$th sampling points. This
restriction means that we will leave the polarizer setting and tilt angle fixed
during each volume acquisition. We could consider scanning the polarizer
setting and tilt angle in the future for faster acquisitions, but I think these
sampling schemes will always decrease ``angular SNR''.

So far we've formulated everything quite generally and haven't made any
reference to the diSPIM geometry. Now we find the constraints that the diSPIM
geometry places on the values of $\mh{s}_{d,m}$ and $\mh{p}_{m}$. If we define
the $\mh{x}$ and $\mh{z}$ axes as the optical axes of the two objective then we
have the following sampling constraint
\begin{align}
\mh{s}_{d,m} \in \{\mh{x}, \mh{z}\}.  \label{eq:first}
\end{align}
In words, this constraint says that we can only detect along one of two discrete
and orthogonal axes.

Next, we constrain the polarizer sampling points $\mh{p}_m$. We introduce a
variable $\mh{s}_{e,m}$ to denote the excitation optical axis for the $m$th
measurement. This variable does not affect the forward model (we've formulated
in terms of $\pp$ only on the excitation side), but it will allow us to
conveniently constrain $\pp$. First, we constrain the excitation optical axis to
be perpendicular to the detection optical axis
\begin{align}
  \mh{s}_{e,m} \cdot \mh{s}_{d,m} = 0.
\end{align}
This is a fairly loose constraint, but it ensures that the light sheet is in
focus across the field of view.

Next, we constrain the excitation optical axis to within the available tilt
angles of the instrument. If we use $\mc{R}$ to denote an ``axis swapping
operator'' ($\mc{R}(\mh{z}) = \mh{x}$ and $\mc{R}(\mh{x}) = \mh{z}$), then the tilting
constraint can be written as
\begin{align}
  \mh{s}_{e,m} \cdot \mc{R}(\mh{s}_{d,m}) \leq \Delta,
\end{align}
where $\Delta$ is the maximum tilt angle. For the current instrument $\Delta = 15^{\circ} = \pi/12$.

Finally, we constrain the excitation polarization $\pp_m$ to be perpendicular to
the excitation optical axis
\begin{align}
  \mh{s}_{e,m} \cdot \pp_m = 0.  \label{eq:final}
\end{align}
For convenience we will denote individual sampling points with $\mf{s}_m$ =
($\pp_m$, $\mh{s}_{d,m}$), and we will denote the set of all sampling points
$\mf{s}_m$ that satisfy Eqs. \eqref{eq:first}--\eqref{eq:final} by $\mc{S}$.
With this notation we can denote a complete sampling scheme with $M$ samples
by $\mf{s} \in \mc{S}^M$ ($M$ copies of the set $\mc{S}^M$).

\section{Objective functions}
Our goal is to choose an $M$-sample sampling scheme $\mf{s} \in \mc{S}^M$ that
is optimal in some sense. We propose that we optimize the Schatten 2-norm of the
continuous-discrete imaging operator
\begin{align}
  \mf{s}^* = \argmax{\mf{s} \in \mc{S}^M}||\mc{D}_{\mf{s}}\mc{H}_{cc}||_2. \label{eq:notime}
\end{align}
The Schatten $p$-norm is defined as the $\mbb{L}_p$ norm of the singular values
of the operator, so the Schatten $2$-norm is a reasonable measure of the amount
of information that the imaging system can pass. It favors sampling schemes that
have many large singular values, so it will prefer sampling schemes that measure
many orthogonal parameters of the object that are insensitive to noise.

It's useful to interpret the Schatten $2$-norm by analyzing a spatial imaging
system. The ``effective OTF'' that appears in many fluorescence microscopy
papers (including Hari and Yicong's 2018 SIM review) is a specific example of a
continuous singular value spectrum. We can calculate the Schatten $2$-norm by
taking the effective OTF, squaring its value at each spatial frequency,
integrating over the spatial frequencies, then taking the square root. The
Schatten $2$-norm will be largest for imaging systems that have many singular
values (a high spatial frequency bandwidth) that are large (we prefer effective
OTFs that are large and constant over the bandwidth). 

Eq. \eqref{eq:notime} will optimize the number and size of the singular values
without considering how interpretable the corresponding parameters are. One of
our motivations for adding light-sheet tilting was to measure rotationally
invariant parameters of the object which will be much more interpretable than
rotationally variant parameters. To achieve this objective we need to measure at
least the $\ell=0$ and $\ell = 2$ rotationally invariant subspaces---a
6-dimensional subspace of $\mbb{L}_2(\mbb{S}^2)$. This will require at least 6
measurements. To ensure that a sampling scheme has measured this rotationally
invariant subspace, we require that
\begin{align}
  \text{rank}(\mc{D}_{\mf{s}}\mc{H}_{cc}\mc{P}_{\ell_c = 2}) \geq 6 \label{eq:constraint}
\end{align}
where $\mc{P}_{\ell_c=2}$ is a projection operator onto the $\ell=0$ and
$\ell=2$ subspace. For convenience we will denote the set of sampling schemes
that satisfy the constraint in Eq. \eqref{eq:constraint} by
$\mc{Z} = \{\mf{s}\, |\, \text{rank}(\mc{D}_{\mf{s}}\mc{H}_{cc}\mc{P}_{\ell_c = 2})
\geq 6\}$.

The solutions of Eq. \eqref{eq:notime} will not necessarily satisfy Eq.
\eqref{eq:constraint}, and simply adding an extra constraint to Eq.
\eqref{eq:constraint} will not ensure that we are optimally measuring the
parameters in the subspace that we want to measure. Instead, we propose an
alternative optimization problem
\begin{align}
    \mf{s}_{\ell_c =2}^* = \argmax{\mf{s} \in \mc{S}^M\cap\mc{Z}}||\mc{D}_{\mf{s}}\mc{H}_{cc}\mc{P}_{\ell_c = 2}||_2,
\end{align}
which ensures that the objective function is independent of singular values that
measure parameters outside of the objective subspace. Note that the set
$\mc{S}^M\cap\mc{Z}$ will definitely be empty for $M \leq 6$, and we are not sure
what value of $M$ will yield solutions.

\section{Practical implementation details}
The two optimization problems above are sufficient to propose optimal sampling
schemes, but they lack many of the details required for an implementation. In
this section I'll go through some of the practical work of converting the
abstract forms into finite-dimensional optimization problems.

First, we need to choose a basis and find matrix representations for the
operators $\mc{D}_{\mf{s}}$ and $\mc{H}_{cc}$. At first glance this seems
difficult---$\mc{D}_{\mf{s}}$ is a CD operator and $\mc{H}_{cc}$ is a CC
operator so they do not have finite-dimensional matrix representations
individually. Luckily, the combination $\mc{D}_{\mf{s}} \mc{H}_{cc}$ is a
band-limited compact operator, so it has a finite-dimensional representation as
a $15\times M$ matrix that we will call $\ms{H}$. The entries of this matrix are
given by
\begin{align}
  \ms{H}_{jm} = H_j(\mh{s}_{d,m}, \mh{p}_m),
\end{align}
where $j$ is a single index over the spherical harmonics, and Eq.
\ref{eq:triple4} shows us how to calculate the entries in terms
of the excitation and detection kernels. 

Next, we need a matrix representation of $\mc{P}_{\ell_c=2}$. This is simple in
a spherical harmonics basis---we create a 15$\times$15 matrix called
$\ms{P}$, fill the first six diagonal elements with ones, and place zeroes
elsewhere
\begin{align}
  \ms{P}_{ij} = \delta_{ij}, \qquad i,j = 0,1,2,3,4,5.
\end{align}

Next, we need a way to calculate the Schatten 2-norm efficiently. We could
calculate the singular values of each $15\times 15$ matrix, but this would be
slow. We can take a shortcut and replace the Schatten 2-norm with the Frobenius
norm---the $\mbb{L}_2$ norm of the matrix entries. Note that this replacement is
only valid for finite-dimensional representations of the operators.

Finally, we need an efficient set of scalar parameters to search through the set
of sampling schemes $\mc{S}^M$. We will parameterize each sample with a discrete
view index $v \in \{0, 1\}$ to indicate the detection axis, a tilt angle defined
from the nominal excitation axis $\delta \in [-\Delta, \Delta]$, and a polarizer
angle defined from the detection axis $\phi \in [0, 2\pi)$.

This parameter space is challenging to search because it is continuous in some
parameters and discrete in others. It's not clear if it's possible to take
derivatives with respect to the continuous parameters, so as a first pass we
will discretize the continuous parameters and use a search heuristic.
Discretization has the advantage of providing solutions that will be easier to
implement experimentally since we can constrain our solutions to sampling
schemes that are easy to align, but of course we can't guarantee that our
sampling scheme solutions will be globally optimal.

We will discretize the tilt angle into 5 angles
$\delta \in \{-\Delta, -\Delta/2, 0,\Delta/2, \Delta\}$, and we will discretize
the polarizer angle into 8 angles $\phi \in \{0,\pi/8,\dots,7\pi/8\}$. This means
that a brute-force approach will need to search
$(2\,\text{views}\times 5\,\text{tilts}\times 8\,\text{polarizer settings})^M$
possible measurement schemes which becomes computationally infeasible for $M$
larger than about 5 or 6.

To avoid a brute-force search we will use a simple hill-climbing algorithm. In
words, we (1) start with an initial guess for each of the $3M$ parameters and
evaluate the objective function, (2) increment and decrement the first parameter
and move in the direction that improves the objective function, (3) repeat (2)
for each parameter one by one, (4) repeat (3) and keep cycling parameters until
no movement improves the objective function or a fixed number of objective
function evaluations is reached. Although the algorithm is extremely coarse and
provides no guarantees, it is simple to implement and it can be stopped at any
time and will always return its best guess.

\section{Results}

\end{document}

