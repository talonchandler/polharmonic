\documentclass[11pt]{article}

%%%%%%%%%%%%
% Packages %
%%%%%%%%%%%%
\usepackage[dvipsnames]{xcolor}
\hyphenpenalty=10000
\usepackage{tikz}
\usetikzlibrary{shapes,arrows}

\usepackage{tocloft}
\usepackage[linesnumbered,ruled]{algorithm2e}
\renewcommand\cftsecleader{\cftdotfill{\cftdotsep}}
\def\undertilde#1{\mathord{\vtop{\ialign{##\crcr
$\hfil\displaystyle{#1}\hfil$\crcr\noalign{\kern1.5pt\nointerlineskip}
$\hfil\tilde{}\hfil$\crcr\noalign{\kern1.5pt}}}}}
\usepackage{bold-extra}
\usepackage{bm}
\usepackage{cleveref}
\usepackage{xcolor}
\usepackage[colorlinks = true,
            linkcolor = black,
            urlcolor  = blue,
            citecolor = black,
            anchorcolor = black]{hyperref}
\usepackage{epstopdf}
\usepackage{braket}
\usepackage{upgreek}
\usepackage{caption}
\usepackage{booktabs}
\usepackage{subcaption}
\usepackage{amssymb,latexsym,amsmath,gensymb}
\usepackage{latexsym}
\usepackage{graphicx}
\usepackage{float}
\usepackage{enumitem}
\usepackage{pdflscape}
\usepackage{url}
\usepackage{array}
\newcolumntype{C}{>{$\displaystyle} c <{$}}
\usepackage{tikz, calc}
\usetikzlibrary{shapes.geometric, arrows, calc}
\tikzstyle{norm} = [rectangle, rounded corners, minimum width=2cm, minimum height=1cm,text centered, draw=black]
\tikzstyle{arrow} = [thick, ->, >=stealth]

\providecommand{\argmax}[1]{\underset{#1}{\mathrm{argmax}}\,}
\providecommand{\argmin}[1]{\underset{#1}{\mathrm{argmin}}\,}
\newcommand{\me}{\mathrm{e}}
\providecommand{\e}[1]{\ensuremath{\times 10^{#1}}}
\providecommand{\mb}[1]{\mathbf{#1}}
\providecommand{\mc}[1]{\mathcal{#1}}
\providecommand{\ro}{\mathbf{\mathfrak{r}}_o}
\providecommand{\so}{\mathbf{\hat{s}}_o}
\providecommand{\sd}{\mathbf{\hat{s}}_d}
\providecommand{\pp}{\mathbf{\hat{p}}}
\providecommand{\rb}{\mathbf{r}_b}
\providecommand{\rbm}{r_b^{\text{m}}}
\providecommand{\rd}{\mathbf{\mathfrak{r}}_d}
\providecommand{\mh}[1]{\mathbf{\hat{#1}}}
\providecommand{\mf}[1]{\mathfrak{#1}}
\providecommand{\ms}[1]{\mathsf{#1}}
\providecommand{\mbb}[1]{\mathbb{#1}}
\providecommand{\bs}[1]{\boldsymbol{#1}}
\providecommand{\tv}{\texttt{v}}
\providecommand{\tx}[1]{\text{#1}}
\providecommand{\tb}[1]{\textbf{#1}}
\providecommand{\ttt}[1]{\texttt{#1}}
\providecommand{\bv}{\bs{\nu}}
\providecommand{\bp}{\bs{\rho}}
\providecommand{\p}{\mh{p}}
\providecommand{\lmsum}{\sum_{l=0}^\infty\sum_{m=-l}^{l}}
\providecommand{\intr}[1]{\int_{\mbb{R}^{#1}}}
\providecommand{\ints}[1]{\int_{\mbb{S}^{#1}}}
\providecommand{\intinf}{\int_{-\infty}^{\infty}}
\let\originalleft\left
\let\originalright\right
\renewcommand{\left}{\mathopen{}\mathclose\bgroup\originalleft}
\renewcommand{\right}{\aftergroup\egroup\originalright}


\providecommand{\fig}[4]{
  % filename, width, caption, label
\begin{figure}[H]
 \captionsetup{width=1.0\linewidth}
 \centering
 \includegraphics[width = #2\textwidth]{#1}
 \caption{#3}
 \label{fig:#4}
\end{figure}
}

\makeatletter
\renewcommand*\env@matrix[1][*\c@MaxMatrixCols c]{%
  \hskip -\arraycolsep
  \let\@ifnextchar\new@ifnextchar
  \array{#1}}
\makeatother

\newcommand{\tensor}[1]{\overset{\text{\tiny$\leftrightarrow$}}{\mb{#1}}}
\newcommand{\tunderbrace}[2]{\underbrace{#1}_{\textstyle#2}}
\providecommand{\figs}[7]{
  % filename1, filename2, caption1, caption2, label1, label2, shift
\begin{figure}[H]
\centering
\begin{minipage}[b]{.45\textwidth}
  \centering
  \includegraphics[width=1.0\linewidth]{#1}
  \captionsetup{justification=justified, singlelinecheck=true}
  \caption{#3}
  \label{fig:#5}
\end{minipage}
\hspace{2em}
\begin{minipage}[b]{.45\textwidth}
  \centering
  \includegraphics[width=1.0\linewidth]{#2}
  \vspace{#7em}
  \captionsetup{justification=justified}
  \caption{#4}
  \label{fig:#6}
\end{minipage}
\end{figure}
}
\makeatletter

\providecommand{\code}[1]{
\begin{center}
\lstinputlisting{#1}
\end{center}
}

\newcommand{\crefrangeconjunction}{--}
%%%%%%%%%%%
% Spacing %
%%%%%%%%%%%
% Margins
\usepackage[
top    = 1.5cm,
bottom = 1.5cm,
left   = 1.5cm,
right  = 1.5cm]{geometry}

% Indents, paragraph space
%\usepackage{parskip}
\setlength{\parskip}{1.5ex}

% Section spacing
\usepackage{titlesec}
\titlespacing*{\title}
{0pt}{0ex}{0ex}
\titlespacing*{\section}
{0pt}{0ex}{0ex}
\titlespacing*{\subsection}
{0pt}{0ex}{0ex}
\titlespacing*{\subsubsection}
{0pt}{0ex}{0ex}

% Line spacing
\linespread{1.1}

%%%%%%%%%%%%
% Document %
%%%%%%%%%%%%
\begin{document}
\title{\vspace{-2.5em} Polarized tilting diSPIM design recommendation\vspace{-1em}} % \author{Talon Chandler, Min Guo, Hari
  % Shroff, Rudolf Oldenbourg, Patrick La Rivi\`ere}
\date{\vspace{-3em}\today\vspace{-1em}}
\maketitle

Here we recommend a polarized tilting diSPIM design and compare the design to
earlier non-tilting designs. These notes summarize two note sets
\href{https://github.com/talonchandler/polharmonic/blob/master/notes/2018-12-10-angular-svd/report/report.pdf}{here}
and
\href{https://github.com/talonchandler/polharmonic/blob/master/notes/2019-01-28-angular-optimize/report/report.pdf}{here}.

Our goal was to find a design that can optimally measure angular distributions
of fluorophores given the following requirements: (1) the design must meet
practical experimental constraints---we can only recommend diSPIM tilting angles
up to 15${}^{\circ}$ and the polarizer settings must be easy to align, (2) the
design must measure all fluorophore orientations in 3D without major blind spots, and
(3) the design should minimize the number of measurements. In the previous notes
we formulated this design problem as an optimization problem. We chose to
optimize the condition number of the system matrix---the ``invertibility'' of
the reconstruction problem---subject to the constraints above.

Figures 1--3 summarize three angular designs: our recommended design, the
current 3-polarization diSPIM without tilting, and a single-view design. In each
\textbf{top left} subfigure we show a schematic of the angular design. Each
arrow indicates a polarizer orientation and the center of each arrow indicates
the excitation optical axis (the tilt angle).

In each \textbf{top right} subfigure we show a set of single direction
reconstructions with the design. We start by creating a uniformly oriented
phantom for each of the 250 directions in the positive octant of the sphere
indicated by \textcolor{green}{green dots}. Next, we simulate the imaging
process, simulate Poisson noise with SNR = 20, reconstruct the phantom, then
find the maximum of the reconstructed ODF. We mark each reconstructed maximum with a
\textcolor{red}{red dot} and connect each phantom/reconstruction pair with a
black line. These figures show the ``blind spots'' of the design---long lines
indicate a large angular error.

In each \textbf{bottom} subfigure we show the singular value spectrum of the
imaging system with the singular functions plotted along the $x$ axis. The
singular function are mutually orthogonal functions, so they necessarily have
\textcolor{red}{positive (red)} and \textcolor{blue}{negative (blue)}
components. The singular value spectrum is a generalized version of the familiar
OTF. In an OTF the $x$ axis is always implicitly labeled with complex
exponentials, while here the $x$ axis is explicitly labeled with angular
functions.

\begin{figure}[p]
 \captionsetup{width=1.0\linewidth}
 \centering
 \includegraphics[width = 0.7\textwidth]{../figures/summary/optimal-6pol.pdf}
 \vspace{-1.5em}
 \caption{Our recommendation for a 6-measurement polarized tilting diSPIM
   design. First, collect the \textcolor{blue}{blue} samples by tilting the
   light sheet and cycling through the three polarizer settings (0${}^{\circ}$,
   60${}^{\circ}$, 120${}^{\circ}$). Next, collect the \textcolor{red}{red}
   samples from the other view by choosing the polarizer setting that points
   towards the \textcolor{blue}{blue} excitation axis (60${}^{\circ}$) then
   tilting the excitation sheet both directions. Finally, collect the
   \textcolor{green}{green} sample by changing the polarizer setting to
   120${}^{\circ}$ and exciting along the optic axis. This design requires 3 equally spaced polarizer settings and 3 equally spaced tilt settings on one arm.\\ \\
   The single direction reconstruction shows that the angular error introduced
   by this design is $\approx 15^{\circ}$. Also, the singular value spectrum
   shows that the last singular value is $\approx 0.15$, so it will not be lost
   in the
   noise at SNR = 20.\\ \\
   Note that the order of collection (\textcolor{blue}{blue},
   \textcolor{red}{red}, \textcolor{green}{green}) is not important---any order
   will work. Also, the design shows 15${}^{\circ}$ tilt angles, but any tilt
   angle will work. Larger tilt angles are better conditioned, so we recommend
   using all of the available tilt.\\ \\
   This design has the largest condition number for an instrument with 6
   measurements and 3 equally spaced polarization settings. We originally tried
   optimizing the root sum of squares of the singular values (the Schatten
   2-norm) and these optimizations resulted in intuitive symmetric
   designs---equally spaced polarizers with equally spaced tilts for both views.
   However, these designs performed poorly because the last singular value was
   very small and lost in the noise. Optimizing the condition number leads to
   less intuitive designs with better performance in noisy conditions.}
 \label{6pol}
\end{figure}

\begin{figure}[H]
 \captionsetup{width=1.0\linewidth}
 \centering
 \includegraphics[width = 0.55\textwidth]{../figures/summary/3dispim.pdf}
 \vspace{-1.55em}
 \caption{Current 3-polarization diSPIM design without tilting. The current
   design introduces errors as large as 30${}^{\circ}$ for single-direction
   reconstructions, and the last singular value is very small so we effectively
   measure a 5-dimensional space. Furthermore, this design has a blind spot in
   the $\ell=2$ band---see previous notes for more details. }
 \label{6pol}
\end{figure}

\begin{figure}[H]
 \captionsetup{width=1.0\linewidth}
 \centering
 \includegraphics[width = 0.55\textwidth]{../figures/summary/single.pdf}
 \vspace{-3em}
 \caption{Single-view PolScope design. This design ``projects'' single-direction
   reconstructions into the transverse plane of the microscope and measures a
   3-dimensional angular space.}
 \label{6pol}
\end{figure}

\end{document}

