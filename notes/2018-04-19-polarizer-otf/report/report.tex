\documentclass[11pt]{article}

%%%%%%%%%%%%
% Packages %
%%%%%%%%%%%%
\usepackage[dvipsnames]{xcolor}
\hyphenpenalty=10000
\usepackage{tikz}
\usetikzlibrary{shapes,arrows}

\usepackage{tocloft}
\renewcommand\cftsecleader{\cftdotfill{\cftdotsep}}
\def\undertilde#1{\mathord{\vtop{\ialign{##\crcr
$\hfil\displaystyle{#1}\hfil$\crcr\noalign{\kern1.5pt\nointerlineskip}
$\hfil\tilde{}\hfil$\crcr\noalign{\kern1.5pt}}}}}
\usepackage{cleveref}
\usepackage{xcolor}
\usepackage[colorlinks = true,
            linkcolor = black,
            urlcolor  = blue,
            citecolor = black,
            anchorcolor = black]{hyperref}
\usepackage{epstopdf}
\usepackage{braket}
\usepackage{upgreek}
\usepackage{caption}
\usepackage{booktabs}
\usepackage{subcaption}
\usepackage{amssymb,latexsym,amsmath,gensymb}
\usepackage{latexsym}
\usepackage{graphicx}
\usepackage{float}
\usepackage{enumitem}
\usepackage{pdflscape}
\usepackage{url}
\usepackage{array}
\newcolumntype{C}{>{$\displaystyle} c <{$}}
\usepackage{tikz, calc}
\usetikzlibrary{shapes.geometric, arrows, calc}
\tikzstyle{norm} = [rectangle, rounded corners, minimum width=2cm, minimum height=1cm,text centered, draw=black]
\tikzstyle{arrow} = [thick, ->, >=stealth]

\newcommand{\argmin}{\arg\!\min}
\newcommand{\me}{\mathrm{e}}
\providecommand{\e}[1]{\ensuremath{\times 10^{#1}}} 
\providecommand{\mb}[1]{\mathbf{#1}}
\providecommand{\mf}[1]{\mathbf{#1}}
\providecommand{\ro}[1]{\mathbf{\mathbf{r}}_o}
\providecommand{\so}[1]{\mathbf{\hat{s}}_o}
\providecommand{\rb}[1]{\mathbf{r}_b}
\providecommand{\rbm}[1]{r_b^{\text{m}}}
\providecommand{\rd}[1]{\mathbf{r}_d}
\providecommand{\mh}[1]{\mathbf{\hat{#1}}}
\providecommand{\bs}[1]{\boldsymbol{#1}} 
\providecommand{\intinf}{\int_{-\infty}^{\infty}}
\providecommand{\fig}[4]{
  % filename, width, caption, label
\begin{figure}[h]
 \captionsetup{width=1.0\linewidth}
 \centering
 \includegraphics[width = #2\textwidth]{#1}
 \caption{#3}
 \label{fig:#4}
\end{figure}
}

\makeatletter
\renewcommand*\env@matrix[1][*\c@MaxMatrixCols c]{%
  \hskip -\arraycolsep
  \let\@ifnextchar\new@ifnextchar
  \array{#1}}
\makeatother

\newcommand{\tensor}[1]{\overset{\text{\tiny$\leftrightarrow$}}{\mb{#1}}}
\newcommand{\tunderbrace}[2]{\underbrace{#1}_{\textstyle#2}}
\providecommand{\figs}[7]{
  % filename1, filename2, caption1, caption2, label1, label2, shift
\begin{figure}[H]
\centering
\begin{minipage}[b]{.45\textwidth}
  \centering
  \includegraphics[width=1.0\linewidth]{#1}
  \captionsetup{justification=justified, singlelinecheck=true}
  \caption{#3}
  \label{fig:#5}
\end{minipage}
\hspace{2em}
\begin{minipage}[b]{.45\textwidth}
  \centering
  \includegraphics[width=1.0\linewidth]{#2}
  \vspace{#7em}
  \captionsetup{justification=justified}
  \caption{#4}
  \label{fig:#6}
\end{minipage}
\end{figure}
}
\makeatletter

\providecommand{\code}[1]{
\begin{center}
\lstinputlisting{#1}
\end{center}
}

\newcommand{\crefrangeconjunction}{--}
%%%%%%%%%%%
% Spacing %
%%%%%%%%%%%
% Margins
\usepackage[
top    = 1.5cm,
bottom = 1.5cm,
left   = 1.5cm,
right  = 1.5cm]{geometry}

% Indents, paragraph space
%\usepackage{parskip}
\setlength{\parskip}{1.5ex}

% Section spacing
\usepackage{titlesec}
\titlespacing*{\title}
{0pt}{0ex}{0ex}
\titlespacing*{\section}
{0pt}{0ex}{0ex}
\titlespacing*{\subsection}
{0pt}{0ex}{0ex}
\titlespacing*{\subsubsection}
{0pt}{0ex}{0ex}

% Line spacing
\linespread{1.1}

%%%%%%%%%%%%
% Document %
%%%%%%%%%%%%
\begin{document}
\title{\vspace{-2.5em} Spatio-angular transfer functions for polarized fluorescence microscopes\vspace{-1em}}  \author{Talon Chandler, Min Guo, Hari Shroff, Rudolf Oldenbourg, Patrick La Rivi\`ere}
\date{\vspace{-1em}\today\vspace{-1em}}
\maketitle
\section{Introduction}
These notes are a continuation of the
\href{https://github.com/talonchandler/polharmonic/blob/master/notes/2018-02-23-spatio-angular-kernel/report/report.pdf}{2018-02-23
  notes} on the spatio-angular transfer functions for unpolarized fluorescence
microscopes. We will use the same notation and extend those results to microscopes
with polarized illumination and detection.

\section{Polarized illumination}
We can model polarized excitation using a spatio-angular excitation point
response function $h_{\text{exc}}(\ro{}, \so{})$. In these notes we will only
consider spatially uniform excitation patterns $h_{\text{exc}}(\so{})$, but we
would need the full function to model structured illumination patterns.

We calculated several spatio-angular excitation point response functions in our
Optics Express paper \cite{chandler17} (in the paper we called this function the
excitation efficiency for a single dipole, but this is equivalent to the
excitation point spread function). If we place a spatially incoherent and
spatially uniform light source (or its image) in the aperture plane with the
optical axis along the $\mh{z}$ axis, then the excitation point response function
is given by
\begin{align}
  h^{\mh{z}}_{\text{exc}}(\Theta, \Phi; \phi_{\text{exc}}) &= \bar{D}\{\bar{A} + \bar{B}\sin^{2}{\Theta} + \bar{C}\sin^{2}{\Theta} \cos{[2 (\Phi - \phi_{\text{exc}}})]\}\label{eq:scalarabs},
\end{align}
where
$\so{} = \cos\Phi\sin\Theta\mh{x} + \sin\Phi\sin\Theta\mh{y} + \cos\Theta\mh{z}$,
the illumination polarizer orientation is given by $\mh{p} = \cos\phi_{\text{exc}}\mh{x} + \sin\phi_{\text{exc}}\mh{y}$, and 
\begin{subequations}
\begin{align}
  \bar{A} &= \frac{1}{4} - \frac{3}{8} \cos{\alpha } + \frac{1}{8} \cos^{3}{\alpha },\\
  \bar{B} &= \frac{3}{16} \cos{\alpha } - \frac{3}{16} \cos^{3}{\alpha },\\
  \bar{C} &= \frac{7}{32} - \frac{3}{32} \cos{\alpha } - \frac{3}{32} \cos^{2}{\alpha } - \frac{1}{32} \cos^{3}{\alpha},\\
  \bar{D} &= \frac{4}{3(1 - \cos\alpha)},
\end{align}\label{eq:coefficients}%
\end{subequations}
where $\alpha = \arcsin(\text{NA}/n_o)$. I'm using bars on the constants to
avoid notation overlap with the previous note set. We can rewrite this
expression in terms of spherical harmonics as
\begin{align}
  h^{\mh{z}}_{\text{exc}}(\so{}; \mh{p}) \propto\, &[2A + (4/3)B]y_0^0(\so{}) - \frac{4\sqrt{5}}{15}By_2^0(\so{}) + \nonumber\\ &\frac{4C}{\sqrt{15}}\left\{[(\mh{p}\cdot\mh{x})^2 - (\mh{p}\cdot\mh{y})^2]y_2^2(\so{}) - (\mh{p}\cdot\mh{x})(\mh{p}\cdot\mh{y})y_2^{-2}(\so{})\right\}. \label{eq:genpsf}
\end{align}

% We can rewrite this expression in vector 
% notation as
% \begin{align}
% h^{\mh{z}}_{\text{exc}}(\so{}; \mh{p}) &= E + F[(\so{}\cdot\mh{x})^2 + (\so{}\cdot\mh{y})] + G\left[\mh{p} \cdot \frac{(\so{}\cdot\mh{x})\mh{x} + (\so{}\cdot\mh{y})\mh{y}}{\sqrt{(\so{}\cdot\mh{x})^2 + (\so{}\cdot\mh{y})^2}}\right]^2,
% \end{align}
% where
% \begin{subequations}
%   \begin{align}
%     E &\equiv DA = \frac{1}{3} - \frac{1}{6}\cos\alpha - \frac{1}{6}\cos^2\alpha,\\
%     F &\equiv D(B-C) = -\frac{7}{24} + \frac{1}{12}\cos\alpha + \frac{5}{24}\cos^2\alpha,\\
%     G &\equiv 2DC = \frac{7}{12} + \frac{1}{3}\cos\alpha + \frac{1}{12}\cos^2\alpha.
%   \end{align}
% \end{subequations}
% Finally, we can rewrite this expression in Cartesian coordinates as 
% \begin{align}
%     h^{\mh{z}}_{\text{exc}}(s_x, s_y; p_x, p_y) &= E + F[s_x^2 + s_y^2] + G\frac{(p_xs_x + p_ys_y)^2}{s_x^2 + s_y^2},
% \end{align}
% where $s_x^2 + s_y^2 + s_z^2 = 1$ and $p_x^2 + p_y^2 + p_z^2= 1$. 

The light-sheet excitation point response function is given by Eq.
\ref{eq:genpsf} in the limit of small NA which gives
\begin{align}
  h^{\mh{z}, \text{ls}}_{\text{exc}}(\so{}; \mh{p}) \equiv \lim_{\alpha \rightarrow 0} h^{\mh{z}}_{\text{exc}}(\so{}) \propto y_0^0(\so{}) + \frac{7}{4\sqrt{15}}\left\{[(\mh{p}\cdot\mh{x})^2 - (\mh{p}\cdot\mh{y})^2]y_2^2(\so{}) - (\mh{p}\cdot\mh{x})(\mh{p}\cdot\mh{y})y_2^{-2}(\so{})\right\}.
\end{align}
To find the excitation point response function for illumination along the
$\mh{x}$ axis we need to change every $\mh{x}$ and $\mh{y}$ to $\mh{z}$ and
$-\mh{y}$, respectively. Rotating the spherical harmonics is not trivial,
but we show the relevant transformation matrix in Appendix \ref{sec:sh}. The
excitation point response function for illumination along the $\mh{x}$ axis is
given by
\begin{align}
  h^{\mh{x}, \text{ls}}_{\text{exc}}(\so{}; \mh{p}) \propto y_0^0(\so{}) + \frac{7}{4\sqrt{15}}\left\{[(\mh{p}\cdot\mh{z})^2 - (\mh{p}\cdot\mh{y})^2]\left[\frac{\sqrt{3}}{2}y_2^0(\so{}) + \frac{1}{2}y_2^2(\so{})\right] + (\mh{p}\cdot\mh{x})(\mh{p}\cdot\mh{y})y_2^{-1}(\so{})\right\}.
\end{align}

To find the excitation transfer functions we need to calculate
\begin{align}
  H_{l, \text{exc}}^m = \int_{\mathbb{S}^2} d\so{}\, h_{\text{exc}}(\so{}) y_l^m(\so{}).
\end{align}
This calculation is straightforward now that we've expressed the excitation
point response functions in terms of spherical harmonics. The excitation
transfer functions are
\begin{align}
  H^{m,\mh{z}}_{l,\text{exc}}(\mh{p}) \propto\, &[2A + (4/3)B]\delta(l, m) - \frac{4\sqrt{5}}{15}B\delta(l-2, m)(\so{}) + \nonumber\\ &\frac{4C}{\sqrt{15}}\left\{[(\mh{p}\cdot\mh{x})^2 - (\mh{p}\cdot\mh{y})^2]\delta(l-2, m-2) - (\mh{p}\cdot\mh{x})(\mh{p}\cdot\mh{y})\delta(l-2, m+2)\right\},\\
  H^{m,\mh{z}, \text{ls}}_{l,\text{exc}}(\mh{p}) \propto\, &\delta(l,m) + \frac{7}{4\sqrt{15}}\left\{[(\mh{p}\cdot\mh{x})^2 - (\mh{p}\cdot\mh{y})^2]\delta(l-2,m-2) - (\mh{p}\cdot\mh{x})(\mh{p}\cdot\mh{y})\delta(l-2, m+2)\right\},\\
  H^{m,\mh{x}, \text{ls}}_{l, \text{exc}}(\mh{p}) \propto\, &\delta(l,m) + \frac{7}{4\sqrt{15}}\left\{[(\mh{p}\cdot\mh{z})^2 - (\mh{p}\cdot\mh{y})^2]\left[\frac{\sqrt{3}}{2}\delta(l-2,m) + \frac{1}{2}\delta(l-2, m-2)\right] + (\mh{p}\cdot\mh{x})(\mh{p}\cdot\mh{y})\delta(l-2, m+1)\right\}.                                                              
\end{align}

\section{Polarized detection}
In the previous note set we found the spatio-angular detection point spread
function for unpolarized fluorescence microscopes. In this section we will find
the spatio-angular detection point spread function when there is a linear
polarizer in the detection path.

The electric field in the back-focal plane of a fluorescence microscope under
the paraxial approximation is given by
\begin{align}
   \mb{\tilde{e}}^{(p)}_b(\rb{};\ro{}, \so{}) \propto
  \begin{bmatrix}
    y_1^1(\so{}) -\frac{2r_b}{f_o}\cos\phi_by_1^0(\so{})\\
    y_1^{-1}(\so{}) -\frac{2r_b}{f_o}\sin\phi_by_1^0(\so{})\\
    0
  \end{bmatrix}.
\end{align}
If we place a polarizer oriented along $\mh{p}_d$ in the back focal plane then
the electric field becomes
\begin{align}
   \mb{\tilde{e}}^{(p)}_b(\rb{};\ro{}, \so{}) \propto \mh{p}_d \cdot
  \begin{bmatrix}
    y_1^1(\so{}) -\frac{2r_b}{f_o}\cos\phi_by_1^0(\so{})\\
    y_1^{-1}(\so{}) -\frac{2r_b}{f_o}\sin\phi_by_1^0(\so{})\\
    0
  \end{bmatrix}.
\end{align}
To find the electric field in the detector plane we need to take the
two-dimensional Fourier transform of each component. The dot product with the
polarizer is linear, so we can pull the Fourier transform inside and use the same
result from the previous notes
\begin{align}
  \tilde{\mb{e}}^{(p)}_d(\rd{}', \so{}; \mh{p}_d) &\propto \mh{p}_d \cdot
  \begin{bmatrix}
    a^{(p)}(r_d')y_1^{1}(\so{}) + 2ib^{(p)}(r_d')\cos\phi_d'y_1^{0}(\so{})\\
    a^{(p)}(r_d')y_1^{-1}(\so{}) + 2ib^{(p)}(r_d')\sin\phi_d'y_1^{0}(\so{})\\
    0\\
  \end{bmatrix}\label{eq:paracsf}. 
\end{align}
Finally, we can find the spatio-angular point spread function by taking the
squared modulus of the electric field on the detector.

In progress. Calculate the spatio-angular PSF then the spatio-angular OTF. 
% \begin{equation}
%   \begin{split}
%   h(\rd{}', \so{}; \mh{p}_d) \propto &\left(a^2(r_d') + 2b^2(r_d') + c^2(r_d')\right)y_0^0(\so{}) -\frac{2\sqrt{15}}{5}a(r_d')c(r_d')\sin(2\phi_d')y_2^{-2}(\so{})\\ &+ \frac{1}{\sqrt{5}}\left(-a^2(r_d') + 4b^2(r_d') - c^2(r_d')\right)y_2^{0}(\so{}) +\frac{2\sqrt{15}}{5}a(r_d')c(r_d')\cos(2\phi_d')y_2^{2}(\so{}). \label{eq:kerna}
% \end{split}
% \end{equation}

\appendix
\section{Rotation of spherical harmonics} \label{sec:sh}
Given a spherical function and its spherical harmonic coefficients $c_j$ (we're
using a single index over the spherical harmonics), the spherical harmonic
coefficients of the rotated function $c'_i$ can be computed with the linear
transformation
\begin{align}
  c_i' = \sum_j M_{ij}c_j, 
\end{align}
where the elements of the linear transformation can be computed with
\begin{align}
  M_{ij} = \int_{\mathbb{S}^2} d\so{}\, y_j(\mathbf{R}\so{})y_i(\so{}), 
\end{align}
where $\mathbf{R}$ is the rotation matrix that maps the original function to the
rotated function \cite{kautz2002}. If we assemble the elements of the linear transformation into
a matrix $\mathbf{M}$, then the matrix is block sparse
\begin{align}
  \mathbf{M} =
  \begin{bmatrix}[c|ccc|cccccc]    
    1&0&0&0&0&0&0&0&0&\cdots\\ \hline
    0&\mathbf{X}&\mathbf{X}&\mathbf{X}&0&0&0&0&0&\cdots\\
    0&\mathbf{X}&\mathbf{X}&\mathbf{X}&0&0&0&0&0&\cdots\\
    0&\mathbf{X}&\mathbf{X}&\mathbf{X}&0&0&0&0&0&\cdots\\ \hline
    0&0&0&0&\mathbf{X}&\mathbf{X}&\mathbf{X}&\mathbf{X}&\mathbf{X}&\cdots\\
    0&0&0&0&\mathbf{X}&\mathbf{X}&\mathbf{X}&\mathbf{X}&\mathbf{X}&\cdots\\
    0&0&0&0&\mathbf{X}&\mathbf{X}&\mathbf{X}&\mathbf{X}&\mathbf{X}&\cdots\\
    0&0&0&0&\mathbf{X}&\mathbf{X}&\mathbf{X}&\mathbf{X}&\mathbf{X}&\cdots\\
    0&0&0&0&\mathbf{X}&\mathbf{X}&\mathbf{X}&\mathbf{X}&\mathbf{X}&\cdots\\    
    \vdots&\vdots&\vdots&\vdots&\vdots&\vdots&\vdots&\vdots&\vdots&\ddots\\        
  \end{bmatrix}.
\end{align}
This means the spherical harmonic coefficients from each band transform
independently from the other bands.

To efficiently find the point spread functions and transfer functions for the
diSPIM we will compute the spherical harmonic coefficient transformation matrix
for a rotation that maps the $\mh{z}$ axis to the $\mh{x}$. The rotation matrix is
given by
\begin{align}
  \mb{R}_{\mh{z}\rightarrow\mh{x}} =
  \begin{bmatrix}
    0&0&1\\
    0&1&0\\
    -1&0&0    
  \end{bmatrix}, 
\end{align}
and the spherical harmonic coefficient transformation matrix is given by
\begin{align}
  \mathbf{M}_{\mh{z}\rightarrow\mh{x}} =
  \begin{bmatrix}[c|ccc|cccccc]
    1&0&0&0&0&0&0&0&0&\cdots\\ \hline 
    0&0&0&1&0&0&0&0&0&\cdots\\
    0&0&1&0&0&0&0&0&0&\cdots\\
    0&-1&0&0&0&0&0&0&0&\cdots\\ \hline 
    0&0&0&0&-1/2&0&0&\sqrt{3}/2&0&\cdots\\
    0&0&0&0&0&-1&0&0&0&\cdots\\
    0&0&0&0&0&0&0&0&-1&\cdots\\
    0&0&0&0&\sqrt{3}/{2}&0&0&1/2&0&\cdots\\
    0&0&0&0&0&0&1&0&0&\cdots\\    
    \vdots&\vdots&\vdots&\vdots&\vdots&\vdots&\vdots&\vdots&\vdots&\ddots\\        
  \end{bmatrix},
\end{align}
when the matrix elements are ordered as
$[y_0^0 | y_1^1, y_1^{-1}, y_1^0 | y_2^0, y_2^1, y_2^{-1}, y_2^2, y_2^{-2}]$.
For more complicated transformations (higher order coefficients or arbitrary
rotations) we can use recurrence relations to find the spherical harmonic
coefficient transformation matrix \cite{ivanic1996}.

\bibliography{report}{}
\bibliographystyle{unsrt}

\end{document}
