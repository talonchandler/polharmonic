\documentclass[10pt]{article}
\usepackage{float}
\usepackage{booktabs}
\usepackage{menukeys}
\usepackage{amsmath}
\usepackage{amsfonts}
\usepackage[normalem]{ulem}
\providecommand{\e}[1]{\ensuremath{\times 10^{#1}}} 
\providecommand{\mb}[1]{\mathbf{#1}}
\providecommand{\mh}[1]{\mathbf{\hat{#1}}}
\providecommand{\bs}[1]{\boldsymbol{#1}} 
\providecommand{\intinf}{\int_{-\infty}^{\infty}}
\providecommand{\mbb}[1]{\mathbb{#1}}
\linespread{1.3}

\usepackage[
top    = 1.5cm,
bottom = 1.5cm,
left   = 1.5cm,
right  = 1.5cm]{geometry}

\begin{document}
\title{\vspace{-2.5em} Harmonic Analysis of Fluorophore Orientation Measurement Systems \vspace{-1em}}
\maketitle
% \author{Talon Chandler,\authormark{1,*} Shalin Mehta,\authormark{1,2,3} Hari
%   Shroff,\authormark{4,5} Rudolf Oldenbourg,\authormark{2,6} and Patrick J. La
%   Rivi\`ere\authormark{1,5}}
% \address{\authormark{1}University of Chicago, Department of Radiology, Chicago, Illinois 60637, USA\\
%   \authormark{2}Marine Biological Laboratory, Bell Center, Woods Hole, Massachusetts 02543, USA\\
%   \authormark{3}(present address) Chan Zuckerberg Biohub, San Francisco, California 94158, USA\\
%   \authormark{4}Section on High Resolution Optical Imaging, National Institute
%   of Biomedical Imaging and Bioengineering, National Institutes of Health,
%   Bethesda, Maryland 20892, USA\\
%   \authormark{5}Marine Biological Laboratory, Whitman Center, Woods Hole,
%   Massachusetts 02543, USA\\
%   \authormark{6}Brown University, Department of Physics, Providence, Rhode
%   Island 02912, USA}
% \email{*talonchandler@talonchandler.com} %% email address is required

%%%%%%%%%%%%%%%%%%% abstract and OCIS codes %%%%%%%%%%%%%%%%

\begin{abstract} 
TODO 
\end{abstract}

% \ocis{(110.0110) Imaging systems; (180.2520) Fluorescence microscopy; (180.0180)
%   Microscopy; (180.6900) Three-dimensional microscopy; (130.5440)
%   Polarization-selective devices; (260.5430) Polarization.}

%%%%%%%%%%%%%%%%%%%%%%% References %%%%%%%%%%%%%%%%%%%%%%%%%
% Comment before submission
 % \bibliography{paper}
 % \bibliographystyle{osajnl}


% Copy .bbl and uncomment these before submission

% \begin{thebibliography}{10}
% \newcommand{\enquote}[1]{``#1''}
% \end{thebibliography}

\section{Introduction}\label{sec:intro}
TODO

\section{Theory}
We use $\mh{r}$ and $(\theta, \phi)$ interchangeably to represent points on the
unit sphere $\mbb{S}^2$ using the parameterization
\begin{align}
\mh{r} = (\sin\theta\cos\phi, \sin\theta\sin\phi, \cos\theta).
\end{align}
Whenever possible we use the notation and vocabulary of Barrett and Myers
\cite{barrett}.

\subsection{Laplace series}
Consider a square-integrable function $f: \mbb{S}^2 \rightarrow \mbb{R}$ that
maps points on the unit sphere to the real numbers. We can express $f$ as a
weighted sum of spherical harmonics called the Laplace series \cite{arfken,
  sloan}
\begin{align}
  f(\theta, \phi) &= \sum_{l=0}^{\infty}\sum_{m=-l}^{l}F_l^m y_l^m(\theta, \phi) \label{eq:double}
\end{align}
where 
\begin{align}
  y_l^m(\theta, \phi) =
  \begin{cases}
    \sqrt{2}K_l^m\cos(m\phi)P_l^m(\cos\theta), & m > 0\\
    K_l^0P_l^0(\cos\theta), & m = 0\\
    \sqrt{2}K_l^m\sin(-m\phi)P_l^{-m}(\cos\theta), & m < 0\\
  \end{cases}
\end{align}
\begin{align}
  K_l^m = \sqrt{\frac{(2l+1)}{4\pi}\frac{(l-|m|)!}{(l+|m|)!}}
\end{align}
and $P_l^m(x)$ are the associated Legendre polynomials. We have chosen to use
the real spherical harmonics because we are representing real functions
only. For convenience we will rewrite equation \ref{eq:double} using a single
index $i = l(l+1) + m$ which gives
\begin{align}
  f(\theta, \phi) &= \sum_{i=0}^{\infty}F_iy_i(\theta, \phi). \label{eq:concrete}
\end{align}
We will also rewrite the functions $f$ and $y_i$ as vectors using a single index
to represent points on the sphere which gives
\begin{align}
  \mb{f} = \sum_{i=0}^{\infty}F_i\mb{y}_i. \label{eq:concrete2}
\end{align}
where $\mb{f} \in \mbb{R}^{\infty\times 1}$ and $\mb{y}_i \in \mbb{R}^{\infty\times 1}$

Finally, we express equation \ref{eq:concrete2} in its final form using the
notation
\begin{align}
 \mb{f} = \mb{Y}^T\mb{F} \label{eq:abstract}
\end{align}
where $\cdot^T$ denotes a transpose, $\mb{F}\in\mbb{R}^{\infty\times 1}$ is an
ordered vector of the Laplace series coefficients, and
$\mb{Y}\in\mbb{R}^{\infty \times \infty}$ is a matrix with each column
consisting of a spherical harmonic $\mb{y}_i$. Note that equation
\ref{eq:abstract} is just a reindexed version of equation \ref{eq:double}.

A key property of the spherical harmonics is that they form an orthonormal basis
on the sphere---that is
\begin{align}
  \mb{Y}^T\mb{Y} = \mb{Y}\mb{Y}^T = \mb{I} \label{eq:ortho}
\end{align}
where $\mb{I}$ is the identity matrix. We can exploit this orthogonality to
compute the Laplace coefficients $\mb{F}$ for a given function $\mb{f}$. If we
left-multiply both sides of equation \ref{eq:abstract} by $\mb{Y}$ and use
equation \ref{eq:ortho} to simplify we get
\begin{align}
  \mb{F} = \mb{Y}\mb{f}. 
\end{align}
We call $\mb{Y}$ the \textit{Laplace expansion operator} because it acts on a
function $\mb{f}$ and generates its Laplace coefficients $\mb{F}$. Note that the
Laplace series of a function on $\mbb{S}^2$ is analogous to the Fourier series
of a function on $\mbb{S}^1$. When we expand a function into a Laplace or
Fourier series, we are expressing the function in a convenient orthonormal
basis.

\subsection{Harmonic analysis of orientation measurements}
Consider a single fluorophore with a fixed dipole axis oriented along a
direction $\mh{r}$. To measure the orientation of the dipole axis we set up an
illumination system to excite the fluorophore---a light source and a combination
of lenses and polarizers---and a detection system to detect the light that the
fluorophore emits as it relaxes---a detector and a combination of lenses and
polarizers. We are deliberately considering very general orientation
measurement systems at this stage.

We can model the complete measurement system using a function
$h(\mh{r}): \mbb{S}^2 \rightarrow \mbb{R}_+$ that maps the dipole orientation to
measured intensity values. We call $h(\mh{r})$ the \textit{point response
  function} of the measurement system. To find the point response function we
can model the measurement system mathematically, or we can build the measurement
system and take measurements as we manipulate the orientation of a single
fluorophore.

Single fluorophores with fixed dipole axes only appear in a very small set of
samples---samples with few fluorophores fixed to immobile structures or samples
with photoactivatable fluorophores. Most samples of interest contain many
rotating fluorophores, and we would like to measure the quantity and orientation
distribution of these fluorophores. Let
$f(\mh{r}): \mbb{S}^2 \rightarrow \mbb{R}_+$ be a function that maps a direction
(a point on the sphere) to half the number of fluorophores pointing along that
direction. We call $f$ the \textit{orientation distribution
  function}. For a single fluorophore oriented along $\mh{r}_0$ the orientation
distribution function is
\begin{align}
  f_{\text{single}}(\mh{r}) = \frac{1}{2}\left[\delta(\mh{r} - \mh{r}_0) + \delta(\mh{r} + \mh{r}_0)\right]
\end{align}
because dipoles are antipodally symmetric. When we add fluorophores to the measurement
volume we add the individual orientation distribution functions
\begin{align}
  f_{N}(\mh{r}) = \sum_{i=0}^N \frac{1}{2}\left[\delta(\mh{r} - \mh{r}_i) + \delta(\mh{r} + \mh{r}_i)\right]
\end{align}
As the number of fluorophores increases the orientation distribution function
approaches a continuous function.

The total intensity measured by the system is given by multiplying the point
response function with the orientation distribution function and integrating
over the sphere
\begin{align}
  g = \int_{\mathbb{S}^2}d\mh{r}\ h(\mh{r})f(\mh{r}). \label{eq:forward}
\end{align}

Equation \ref{eq:forward} assumes that the measurement system is
\textit{linear}---if $\int_{\mathbb{S}^2}d\mh{r}\ h(\mh{r})f_1(\mh{r}) = g_1$
and $\int_{\mathbb{S}^2}d\mh{r}\ h(\mh{r})f_2(\mh{r}) = g_2$ then
$\int_{\mathbb{S}^2}d\mh{r}\ h(\mh{r})[af_1(\mh{r}) + bf_2(\mh{r})] =
ag_1+bg_2$. This requirement is satisfied in most experimental cases as long as
the fluorophores and detectors are not saturated. Note that this requirement
does not mean that $h$ and $f$ have to be linear functions.

We simplify our notation by reindexing $h$ and $f$ and rewriting equation
\ref{eq:forward} as
\begin{align}
  g = \mb{h}^T\mb{f}. 
\end{align}
If we expand $\mb{h}$ and $\mb{f}$ into their respective Laplace series using equation
\ref{eq:abstract} then
\begin{align}
  g = (\mb{Y}^T\mb{H})^T(\mb{Y}^T\mb{F})
\end{align}
where $\mb{H} \in \mbb{R}^{\infty\times 1}$ and
$\mb{F} \in \mbb{R}^{\infty \times 1}$ are vectors made of the Laplace
coefficients of $\mb{h}$ and $\mb{f}$, respectively. Exploiting the
orthogonality of the spherical harmonics (equation \ref{eq:ortho}) gives
\begin{align}
  g = \mb{H}^T\mb{F} \label{eq:dot}
\end{align}
Equation \ref{eq:dot} gives us a fundamental insight into fluorophore
orientation measurement systems---the measured intensity is a sum of the Laplace
coefficients of the orientation distribution function weighted by the Laplace
coefficients of the point response function.

Most orientation measurement systems make multiple intensity measurements with
different point response functions by using multiple detectors or by changing
the geometry of the measurement system. We can represent an orientation
measurement system that takes $N$ measurements using the compact notation
\begin{align}
  \mb{g} = \mb{\Psi}\mb{F} \label{eq:fw_mat}
\end{align}
where $\mb{g} \in \mathbb{R}^N_+$ is a vector of the intensity measurements,
and $\mb{\Psi} \in \mbb{R}^{N\times \infty}$ is the \textit{system matrix}
where the $n$th row of the system matrix consists of the Laplace coefficients of
the $n$th point response function. We can also rewrite equation \ref{eq:fw_mat}
as
\begin{align}
  \mb{g} = \mathcal{H}\mb{f}
\end{align}
where $\mathcal{H} \equiv \mb{\Psi}\mb{Y}^T \in \mbb{R}^{N\times \infty}$ is the 
continuous-to-discrete forward operator of the orientation measurement system. 

In general, equation \ref{eq:fw_mat} is a product of an $N\times \infty$ system
matrix with a $\infty \times 1$ vector. Luckily, all orientation measurements
systems are \textit{band-limited}---the system matrix only consists of $M$
non-zero columns which simplifies equation \ref{eq:fw_mat} to a product of an
$N\times M$ system matrix with an $M \times 1$ vector.

In practice the intensity measurements made in equation \ref{eq:fw_mat} are
corrupted by Poisson noise and background counts, so the complete forward model is
given by
\begin{align}
  \mb{g} \sim \text{Pois}(\mathcal{H}\mb{f} + \mb{b})
\end{align}
where $\mb{b} \in \mbb{R}_+^N$ is a vector of background intensity
measurements.

\subsection{Analysis and design}
We can use the formalism developed in the previous section to characterize and
optimize an orientation imaging system. The first task for the designer is to
find the point response function for each intensity measurement
$\{h_1, h_2, \ldots h_M\}$ via modeling or direct measurement. The second task
is to find the Laplace series coefficients for the point response functions and
use them to assemble the system matrix $\mb{\Psi}$. The third task is to
calculate the \textit{crosstalk matrix} defined as
\begin{align}
  \mb{B} \equiv \mb{\Psi}^T\mb{\Psi}.
\end{align}
The crosstalk matrix is an infinite positive semidefinite matrix that
characterizes the aliasing properties of the measurement system. The
off-diagonal entries $\beta_{i,j}$ measure the degree of aliasing between the
$i$th and $j$th Laplace components---a zero indicates that the components are
orthogonal, and a one indicates that the components are linearly dependent. The
diagonal entries $\beta_{i,i}$ indicate how efficiently the $i$th Laplace
component is transferred by the measurement system---a zero indicates that the
component cannot be recovered from the data, and a one indicates that the data
is efficiently transmitted. The ideal crosstalk matrix is the identity
matrix---such a system could transmit all Laplace components without aliasing
and could be used to reconstruct all possible fluorophore distributions.

The final task of the designer is to optimize the system based on the
information in the crosstalk matrix. The designer may want to add extra
measurements so that extra Laplace components are transmitted by the system. The
designer may also have access to other parameters---polarizer orientations,
numerical aperture, illumination and detection geometry---that may be optimized
to extend and diagonalize the crosstalk matrix.

\subsection{Reconstruction methods}
Our goal is to estimate the orientation distribution function from a set of
intensity measurements $\mb{g}$. If nothing is known about the orientation
distribution function and the noise on the measurements is small, then we can
use the minimum-norm least-squares (MNLS) estimate
\begin{align}
  \hat{\mb{F}}_\text{MNLS} = \mb{\Psi}^+(\mb{g}-\mb{b}) \label{eq:mnls_est}
\end{align}
where $\mathbf{\Psi}^+$ is the Moore-Penrose pseudoinverse of
$\mb{\Psi}$. Once we have an estimate of the Laplace coefficients, we can
find the estimated orientation distribution function with
\begin{align}
  \hat{\mb{f}}_\text{MNLS} = \mb{Y}^T\hat{\mb{F}}_\text{MNLS}. \label{eq:mnls}
\end{align}

The MNLS algorithm is very efficient, but it amplifies noise when $\mb{\Psi}$
has small singular values. In low-noise applications this is not an issue, but
in many applications amplifying noise is not acceptable.

A more generally applicable algorithm is the Richardson-Lucy (RL) algorithm with
updates given by
\begin{align}
  \hat{\mb{F}}^{(i+1)} = \text{diag}(\mathbf{H}^T\textbf{1})^{-1}\text{diag}\left[\mb{\Psi}^T\text{diag}(\mb{\Psi}\hat{\mb{F}}^{(i)} + \mb{b})^{-1}\mb{g}\right]\hat{\mb{F}}^{(i)}. \label{eq:rl_est}
\end{align}
Once the RL algorithm has converged to an estimate of the Laplace coefficients
$\hat{\mb{F}}_\text{RL}$, we can find the estimated orientation distribution function
with
\begin{align}
  \hat{\mb{f}}_\text{RL} = \mb{Y}^T\hat{\mb{F}}_\text{RL}. \label{eq:rl}
\end{align}

\subsection{Reconstruction with priors (in progress)}

We would also like the ability to incorporate priors into our reconstruction
algorithms. For example, we may know that we are measuring a single dipole, or
we may know that the orientation distribution function is rotationally
symmetric. All such priors can be formulated as a restriction on the set of
possible orientation distributions. Our goal is to find a way to map our
estimates of the measurable Laplace coefficients to a member of the set of
possible orientation measurements.

Let $\{\mb{z}_1, \mb{z}_2,\ldots \mb{z}_p\}$ denote the \textit{prior set}---the
(possibly infinite) set of orientation distribution functions that could be
present in the sample. We can assemble the set into the rows of a matrix
$\mb{z}\in\mbb{R}^{p\times\infty}$ and apply the Laplace expansion
operator to obtain a matrix $\mb{Z}\in\mbb{R}^{p\times\infty}$ with the
Laplace coefficients of each prior in each row
\begin{align}
  \mb{Z} = \mb{Y}\mb{z}.
\end{align}
When $\mb{Z}$ acts on vectors of Laplace coefficients it returns the expansion
of those Laplace coefficients in terms of the Laplace expansion of members of
the prior set. Therefore, if we have an estimate of the directly measurable
Laplace coefficients $\hat{\mb{F}}$ (from equation \ref{eq:mnls_est} or
\ref{eq:rl_est}), we can estimate the corresponding Laplace coefficients in the
prior set by multiplication with the matrix $\mb{Z}^T\mb{Z}$
\begin{align}
  \hat{\mb{F}}_{\text{prior}} = \mb{Z}^T\mb{Z}\hat{\mb{F}}. \label{eq:prior}
\end{align}
Multiplication by $\mb{Z}^T\mb{Z}$ amounts to projecting the Laplace
coefficients onto the nearest member of the prior set. As usual, we can find
the estimated orientation distribution function with
\begin{align}
  \hat{\mb{f}}_{\text{prior}} = \mb{Y}^T\hat{\mb{F}}_{\text{prior}}.
\end{align}

We need to choose our prior set carefully for a given measurement system. If we
choose a prior set that is too large for a measurement system, then multiple
members of the prior set will give rise to the same intensity
measurements---they will be indistinguishable. We call members of the prior set
that give rise to the same measurements \textit{degenerate}.

\subsection{Single-molecule prior (in progress)}
\textbf{I strongly suspect that} if we can measure all of the spherical
harmonics in a single band and we have a single-molecule prior, then we can
reconstruct the orientation of a single molecule in any orientation. I'm still
developing a proper argument for this.

From \cite{barrett} section 6.7.7: ``$\mbb{SO}(3)$...has a set of irreducible
representations usually denoted by an index $l$ and the $l$th irreducible
representation has $2l+1$ dimensions.'' These irreducible representations
correspond to the rows in Figure XXX.

\subsection{Rotational symmetry prior (in progress)}

\subsection{Cram\'er-Rao lower bounds (in progress)}
In the previous sections we have broken down the reconstruction into three
steps
\begin{enumerate}
\item Estimate the measurable Laplace coefficients using equation
  \ref{eq:mnls_est} or equation \ref{eq:rl_est}.
\item If applicable, use equation \ref{eq:prior} to incorporate priors.
\item Use the Laplace coefficients to calculate the orientation distribution
  function.
\end{enumerate}
If our goal is to estimate the orientation distribution function from intensity
measurements, then we would like to know how noise on our measurements affects
the estimate of the orientation distribution function. 

\section{Methods (in progress)}
To demonstrate the value of using harmonic analysis for designing orientation
measurement systems we will work through several examples of orientation
measurement systems.

For each system we will calculate the point response functions, the system matrix,
and the crosstalk matrix. Next, we will optimize the orientation measurement system
and make measurement recommendations. Finally, we will analyze the ability of
the systems to measure the orientation of single fluorophores.

\section{Results (in progress)}
\subsection{Point detectors}
Consider an excited dipole being measured by a single point detector. The point
response function for this measurement system is given by
\begin{align}
  h_1(\theta, \phi) = \sin^2\theta
\end{align}
where $\theta$ is the angle between the dipole axis and the detector. We can
expand the point response function in a Laplace series given by
\begin{align}
  h_1(\theta, \phi) = \frac{2\sqrt{\pi}}{3}y_0^0(\theta, \phi) + \frac{4\sqrt{5\pi}}{15}y_2^0(\theta, \phi).
\end{align}

In other words, a small single detector measures a linear combination of two
spherical harmonics of the orientation distribution function. As expected, we
cannot learn much about the orientation distribution function from this
measurement unless we have a prior that either (1) restricts the orientation
distribution functions to one of the spherical harmonics $y_0^0$ or $y_2^0$ or
(2) restricts the orientation distribution function to set that directly maps to
either $y_0^0$ or $y_2^0$.

If we add two more point detector along orthogonal directions their point response
functions and Laplace expansions are
\begin{align}
  h_2(\theta, \phi) &= \cos^2\theta + \sin^2\theta\sin^2\phi = \frac{4\sqrt{\pi}}{3}y_0^0(\theta, \phi) - \frac{\sqrt{30\pi}}{15}y_2^{-2}(\theta, \phi) + \frac{2\sqrt{5\pi}}{15}y_2^{0}(\theta, \phi) - \frac{\sqrt{30\pi}}{15}y_2^2(\theta, \phi)\\
  h_3(\theta, \phi) &= \cos^2\theta + \sin^2\theta\cos^2\phi = \frac{4\sqrt{\pi}}{3}y_0^0(\theta, \phi) + \frac{\sqrt{30\pi}}{15}y_2^{-2}(\theta, \phi) + \frac{2\sqrt{5\pi}}{15}y_2^{0}(\theta, \phi) + \frac{\sqrt{30\pi}}{15}y_2^2(\theta, \phi)
\end{align}
These detectors add two more measurements, but they also introduce two more
unknowns.

\textbf{Still investigating} I suspect that we need $>=6$ detectors along
independent axes to create an invertible system that can reconstruct the
orientation of single molecules in any orientation. Single detectors measure
linear combinations of spherical harmonics in the $l=0$ (one harmonic) and $l=2$
(five harmonics) bands for a total of six harmonics.

\subsection{Single view microscopes}
\subsubsection{Epi-illumination and epi-detection}
Point response functions for a wide range of microscopes have been calculated in
\cite{chandler}. First, consider a epi-illumination microscope with polarized
illumination and epi-detection. The point response function is
\begin{align}
  h_{\text{epi-epi}}(\theta, \phi) = 2D\{A+B\sin^2\theta + C\sin^2\theta\cos[2(\phi - \phi_{\text{exc}})]\}(A + B\sin^2\theta)
\end{align}
where $A, B, C, D$ are constants depending on the numerical aperture and $\phi_{\text{exc}}$ is the excitation polarization angle. The Laplace expansion is given by
\begin{align*}
  h_{\text{epipol-epi}}(\theta, \phi) = &\frac{2\sqrt{\pi}D}{15}(15A^2 + 20AB + 5AC + 8B^2 + 4BC)y_0^1(\theta, \phi)\\
                                     &-\frac{\sqrt{30\pi}CD}{105}(7iA\sin(2\phi_{\text{exc}}) - 7A\cos(2\phi_{\text{exc}}) + 6iB\sin(2\phi_{\text{exc}}) - 6B\cos(2\phi_{\text{exc}}))y_2^{-2}(\theta, \phi)\\
                                     &-\frac{2\sqrt{5\pi}D}{105}(28AB + 7AC + 16B^2 + 8BC)y_2^0(\theta, \phi)\\
                                     &+\frac{\sqrt{30\pi}CD}{105}(7iA\sin(2\phi_{\text{exc}}) - 7A\cos(2\phi_{\text{exc}}) + 6iB\sin(2\phi_{\text{exc}}) - 6B\cos(2\phi_{\text{exc}}))y_2^{-2}(\theta, \phi)\\
                                     &-\frac{2\sqrt{10\pi}BCD}{105}e^{-2i\phi_{\text{exc}}}y_4^{-2}(\theta, \phi)\\
                                     &+\frac{8\sqrt{\pi}BD}{105}(2B + C)y_4^{0}(\theta, \phi)\\
                                     &-\frac{2\sqrt{10\pi}BCD}{105}e^{-2i\phi_{\text{exc}}}y_4^{-2}(\theta, \phi)\\
\end{align*}
We can see that adding illumination polarizers extends our measurements into the
$l=4$ band.

I am surprised to see complex constants show up here. This could indicate an issue
with my integration program or integrating around poles?

\subsubsection{Orthogonal illumination and detection}
The point response function for low-NA polarized illumination (light sheet
illumination) with an orthogonal detector is
\begin{align}
  h_{\text{orthopol-epi}} = 2(\sin^2\theta\cos^2(\phi - \phi_{\text{exc}}))(A + B(1 - \cos^2\phi\sin^2\theta)).
\end{align}
It's Laplace expansion is
\begin{align*}
  h_{\text{orthopol-epi}} = &\cdots y_0^0(\theta, \phi)\\
                            &+ \cdots y_2^{-2}(\theta, \phi)
                            + \cdots y_2^0(\theta, \phi)
                            + \cdots y_2^2(\theta, \phi)\\
                            &+ \cdots y_4^{-4}(\theta, \phi)
                            + \cdots y_4^{-2}(\theta, \phi)
                            + \cdots y_4^0(\theta, \phi)
                            + \cdots y_4^2(\theta, \phi)
                            + \cdots y_4^4(\theta, \phi)
  \end{align*}

\subsection{Dual view microscopes}
\subsubsection{Epi-illumination and epi-detection}
\subsubsection{Orthogonal illumination and detection}
\section{Discussion and conclusions (in progress)}
\begin{itemize}
\item Single detectors measure members of the $l=0$ and $l=2$ bands. Changing the axis of the detector changes what members of the $l=2$ band are measured. 
\item Polarized illumination with a single detector measures members of the $l=0$, $l=2$, and $l=4$ bands.
\item \textbf{In progress} Polarized illumination and detection measures members of the $l=0$, $l=2$, $l=4$, and $l=6$ bands. 
\end{itemize}

\section*{Funding}
% National Institute of Health (NIH) (R01GM114274, R01EB017293).

\section*{Acknowledgments}
% TC was supported by a University of Chicago Biological Sciences Division
% Graduate Fellowship. HS and PL were supported by Marine Biological Laboratory
% Whitman Center Fellowships. HS was supported by the Intramural Research
% Programs of the National Institute of Biomedical Imaging and Bioengineering.

\section*{Disclosures}
% The authors declare that there are no conflicts of interest related to this article.

% Comment before submission
\bibliography{paper}{}
\bibliographystyle{unsrt}
%\bibliographystyle{default}

\end{document}
