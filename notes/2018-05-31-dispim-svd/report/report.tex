
\documentclass[11pt]{article}

%%%%%%%%%%%%
% Packages %
%%%%%%%%%%%%
\usepackage[dvipsnames]{xcolor}
\hyphenpenalty=10000
\usepackage{tikz}
\usetikzlibrary{shapes,arrows}

\usepackage{tocloft}
\renewcommand\cftsecleader{\cftdotfill{\cftdotsep}}
\def\undertilde#1{\mathord{\vtop{\ialign{##\crcr
$\hfil\displaystyle{#1}\hfil$\crcr\noalign{\kern1.5pt\nointerlineskip}
$\hfil\tilde{}\hfil$\crcr\noalign{\kern1.5pt}}}}}
\usepackage{cleveref}
\usepackage{xcolor}
\usepackage[colorlinks = true,
            linkcolor = black,
            urlcolor  = blue,
            citecolor = black,
            anchorcolor = black]{hyperref}
\usepackage{epstopdf}
\usepackage{braket}
\usepackage{upgreek}
\usepackage{caption}
\usepackage{booktabs}
\usepackage{subcaption}
\usepackage{amssymb,latexsym,amsmath,gensymb}
\usepackage{latexsym}
\usepackage{graphicx}
\usepackage{float}
\usepackage{enumitem}
\usepackage{pdflscape}
\usepackage{url}
\usepackage{array}
\newcolumntype{C}{>{$\displaystyle} c <{$}}
\usepackage{tikz, calc}
\usetikzlibrary{shapes.geometric, arrows, calc}
\tikzstyle{norm} = [rectangle, rounded corners, minimum width=2cm, minimum height=1cm,text centered, draw=black]
\tikzstyle{arrow} = [thick, ->, >=stealth]

\newcommand{\argmin}{\arg\!\min}
\newcommand{\me}{\mathrm{e}}
\providecommand{\e}[1]{\ensuremath{\times 10^{#1}}} 
\providecommand{\mb}[1]{\mathbf{#1}}
\providecommand{\mc}[1]{\mathcal{#1}}
\providecommand{\ro}[1]{\mathbf{\mathfrak{r}}_o}
\providecommand{\so}[1]{\mathbf{\hat{s}}_o}
\providecommand{\rb}[1]{\mathbf{r}_b}
\providecommand{\rbm}[1]{r_b^{\text{m}}}
\providecommand{\rd}[1]{\mathbf{\mathfrak{r}}_d}
\providecommand{\mh}[1]{\mathbf{\hat{#1}}}
\providecommand{\mf}[1]{\mathfrak{#1}}
\providecommand{\mbb}[1]{\mathbb{#1}}
\providecommand{\bs}[1]{\boldsymbol{#1}} 
\providecommand{\intinf}{\int_{-\infty}^{\infty}}
\providecommand{\fig}[4]{
  % filename, width, caption, label
\begin{figure}[h]
 \captionsetup{width=1.0\linewidth}
 \centering
 \includegraphics[width = #2\textwidth]{#1}
 \caption{#3}
 \label{fig:#4}
\end{figure}
}

\makeatletter
\renewcommand*\env@matrix[1][*\c@MaxMatrixCols c]{%
  \hskip -\arraycolsep
  \let\@ifnextchar\new@ifnextchar
  \array{#1}}
\makeatother

\newcommand{\tensor}[1]{\overset{\text{\tiny$\leftrightarrow$}}{\mb{#1}}}
\newcommand{\tunderbrace}[2]{\underbrace{#1}_{\textstyle#2}}
\providecommand{\figs}[7]{
  % filename1, filename2, caption1, caption2, label1, label2, shift
\begin{figure}[H]
\centering
\begin{minipage}[b]{.45\textwidth}
  \centering
  \includegraphics[width=1.0\linewidth]{#1}
  \captionsetup{justification=justified, singlelinecheck=true}
  \caption{#3}
  \label{fig:#5}
\end{minipage}
\hspace{2em}
\begin{minipage}[b]{.45\textwidth}
  \centering
  \includegraphics[width=1.0\linewidth]{#2}
  \vspace{#7em}
  \captionsetup{justification=justified}
  \caption{#4}
  \label{fig:#6}
\end{minipage}
\end{figure}
}
\makeatletter

\providecommand{\code}[1]{
\begin{center}
\lstinputlisting{#1}
\end{center}
}

\newcommand{\crefrangeconjunction}{--}
%%%%%%%%%%%
% Spacing %
%%%%%%%%%%%
% Margins
\usepackage[
top    = 1.5cm,
bottom = 1.5cm,
left   = 1.5cm,
right  = 1.5cm]{geometry}

% Indents, paragraph space
%\usepackage{parskip}
\setlength{\parskip}{1.5ex}

% Section spacing
\usepackage{titlesec}
\titlespacing*{\title}
{0pt}{0ex}{0ex}
\titlespacing*{\section}
{0pt}{0ex}{0ex}
\titlespacing*{\subsection}
{0pt}{0ex}{0ex}
\titlespacing*{\subsubsection}
{0pt}{0ex}{0ex}

% Line spacing
\linespread{1.1}

%%%%%%%%%%%%
% Document %
%%%%%%%%%%%%
\begin{document}
\title{\vspace{-2.5em} Singular value decomposition of a dual\\ orthogonal-view
  fluorescence microscope\vspace{-1em}} % \author{Talon Chandler, Min Guo, Hari
  % Shroff, Rudolf Oldenbourg, Patrick La Rivi\`ere}
\date{\vspace{-3em}\today\vspace{-1em}}
\maketitle
\section{Introduction}
In these notes we will find the kernel, transfer function, and singular value
decomposition of a dual orthogonal view fluorescence microscope (diSPIM). We
will model the relationship between the spatial density of fluorophores in a
three-dimensional sample---a member of $\mbb{L}_2(\mbb{R}^3)$---and the two
three-dimensional volumes---two members of $\mbb{L}_2(\mbb{R}^3)$. We could
consider the diSPIM as a microscope that makes two samples of a larger space
$\mbb{L}_2(\mbb{R}^3 \times \mbb{S}^2)$, but I don't think this will give us any
extra insight at this point.

\section{Kernel}
The diSPIM illuminates the sample with a uniform-width Gaussian beam (we ignore
light-sheet broadening). In path A the light sheet is in the $xy$ plane for
viewing by an objective with an optical axis along the $z$ axis. In path B the
roles are reversed and the light sheet is in the $yz$ plane for viewing by an
objective with an optical axis along the $x$ axis.

We can model the relationship between the object and the data using an integral
transform
\begin{align}
  g_{\text{v}}(\rd{}) = \left[\mc{H}f\right]_{\text{v}}(\rd{}) = \int_{\mbb{R}^3}d\ro{}\, h_{\text{v}}(\rd{} - \ro{})f(\ro{}), \qquad \text{v} = \{\text{A}, \text{B}\}, 
\end{align}
where $g_{\text{v}}(\rd{})$ is the three-dimensional data collected from the
$\text{v}$th view, $f(\ro{})$ is the three-dimensional density of fluorophores,
and $h_{\text{v}}(\rd{} - \ro{})$ is the kernel (or point-spread function) for
the ${\text{v}}$th view. A typical model for the kernels of the diSPIM is
\begin{align}
  h_{\text{A}}(\ro{}) &= \mf{g}(r_x, \sigma_{\text{tr}})\mf{g}(r_y, \sigma_{\text{tr}})\mf{g}(r_z, \sigma_{\text{ax}}), \label{eq:ha}\\
  h_{\text{B}}(\ro{}) &= \mf{g}(r_x, \sigma_{\text{ax}})\mf{g}(r_y, \sigma_{\text{tr}})\mf{g}(r_z, \sigma_{\text{tr}}), \label{eq:hb}
\end{align}
where $\mf{g}(x, \sigma) \equiv \text{exp}(x^2/2\sigma^2)/\sqrt{2\pi\sigma^2}$,
$\ro{} = r_x\mh{x} + r_y\mh{y} + r_z\mh{z}$ is the three-dimensional position
vector in the object, $\sigma_{\text{tr}}$ is the spatial standard deviation of
the transverse point spread function (set by the detection NA), and
$\sigma_{\text{ax}}$ is the spatial standard deviation of the axial point spread
function (width/thickness of the light sheet).

The adjoint of forward operator is given by
\begin{align}
  f(\ro{}) = [\mc{H}^{\dagger}g](\ro{}) = \sum_{{\text{v}} = \{\text{A}, \text{B}\}}\int_{\mbb{R}^3}d\rd{}\, h_{\text{v}}(\rd{} - \ro{})g_v(\rd{}). 
\end{align}
\section{Transfer function}
We can rewrite the forward and adjoint operators in the frequency domain as
\begin{align}
  G_{\text{v}}(\bs{\nu}) &= H_{\text{v}}(\bs{\nu})F(\bs{\nu}), \label{eq:fwd} \\
  F(\bs{\nu}) &= \sum_{{\text{v}} = \{\text{A}, \text{B}\}}H_{\text{v}}(\bs{\nu})G_{\text{v}}(\bs{\nu}),\label{eq:adj}
\end{align}
where
\begin{align}
  G_{\text{v}}(\bs{\nu}) &= \int_{\mbb{R}^3}d\rd{}\, g_{\text{v}}(\rd{})e^{i2\pi\rd{}\cdot\bs{\nu}},\\
  H_{\text{v}}(\bs{\nu}) &= \int_{\mbb{R}^3}d\ro{}\, h_{\text{v}}(\ro{})e^{i2\pi\ro{}\cdot\bs{\nu}},\\
  F(\bs{\nu}) &= \int_{\mbb{R}^3}d\ro{}\, f(\ro{})e^{i2\pi\ro{}\cdot\bs{\nu}}.
\end{align}
If we use the kernels in Eqs. \ref{eq:ha} and \ref{eq:hb} then the transfer
functions are
\begin{align}
  H_{\text{A}}(\bs{\nu}) &= \mf{g}(\nu_x, 1/\sigma_{\text{tr}})\mf{g}(\nu_y, 1/\sigma_{\text{tr}})\mf{g}(\nu_z, 1/\sigma_{\text{ax}}),\\
  H_{\text{B}}(\bs{\nu}) &= \mf{g}(\nu_x, 1/\sigma_{\text{ax}})\mf{g}(\nu_y, 1/\sigma_{\text{tr}})\mf{g}(\nu_z, 1/\sigma_{\text{tr}}), 
\end{align}
where $\bs{\nu} = \nu_x\mh{x} + \nu_y\mh{y} + \nu_z\mh{z}$ is a
three-dimensional spatial frequency vector.

\section{Singular value decomposition}
To find the singular value decomposition of the imaging operator we need to
solve the eigenvalue problem
\begin{align}
  [\mc{H}^{\dagger}\mc{H}]u_{\bs{\rho}}(\ro{}) = \mu_{\bs{\rho}}u_{\bs{\rho}}(\ro{}), \label{eq:eig}
\end{align}
where $u_{\bs{\rho}}(\ro{})$ are the object space singular functions of the
system. This eigenvalue problem is easily solved in the frequency domain, so we
write the object-space singular functions as
\begin{align}
  u_{\bs{\rho}}(\ro{}) = U(\bs{\rho})e^{i2\pi\ro{}\cdot\bs{\rho}}. \label{eq:ftu}
\end{align}
Plugging Eqs. \ref{eq:fwd}, \ref{eq:adj}, and \ref{eq:ftu} into Eq. \ref{eq:eig}
gives a new eigenvalue equation
\begin{align}
  \sum_{{\text{v}} = \{\text{A}, \text{B}\}}H_{\text{v}}(\bs{\rho})H_{\text{v}}(\bs{\rho})U(\bs{\rho}) = \mu_{\bs{\rho}}U(\bs{\rho}).
\end{align}
Therefore, the eigenvalues are given by
\begin{align}
  \mu_{\bs{\rho}} = \sum_{{\text{v}} = \{\text{A}, \text{B}\}}H^2_{\text{v}}(\bs{\rho}),
\end{align}
and the eigenvectors are constants
\begin{align}
  U(\bs{\rho}) = 1. 
\end{align}
The singular values are given by the square root of the eigenvalues
\begin{align}
  \sigma_{\bs{\rho}} = \sqrt{\sum_{\text{v} = \{\text{A}, \text{B}\}}H^2_\text{v}(\bs{\rho})} = \sqrt{H^2_{\text{A}}(\bs{\rho}) + H^2_{\text{B}}(\bs{\rho})},
\end{align}
and the object-space singular functions are the usual Fourier components of the
object
\begin{align}
  u_{\bs{\rho}}(\ro{}) = e^{i2\pi\ro{}\cdot\bs{\rho}}. 
\end{align}
Finally, the data space singular functions are given by
\begin{align}
  [v_{\bs{\rho}}]_{\text{v}}(\rd{}) = \frac{1}{\sigma_{\bs{\rho}}}\mc{H}_{\text{v}}u_{\bs{\rho}}(\ro{}) = \frac{H_{\text{v}}(\bs{\rho})}{\sqrt{H^2_{\text{A}}(\bs{\rho}) + H^2_{\text{B}}(\bs{\rho})}}e^{i2\pi\rd{}\cdot\bs{\rho}}.
\end{align}

\section{Operator decompositions}
We can use the singular vectors and singular values to decompose the forward and
adjoint operators of the microscope. Table 1.2 of Barrett is particularly
helpful.

The forward operator decomposes into
\begin{align}
  \mc{H}_{\text{v}}f(\ro{}) = \sigma_{\bs{\rho}}\mb{v}\mb{u}^{\dagger}f(\ro{}) = \mc{F}^{-1}\left\{H_{\text{v}}(\bs{\rho})\mc{F}\left\{f(\ro{})\right\}\right\}. 
\end{align}
This result is essentially the Fourier-convolution theorem---it means that we
can apply the forward operator by taking the Fourier transform of the object,
multiplying the result by the Fourier transform of the kernel, then taking the
inverse Fourier transform.

The adjoint operator decomposes into 
\begin{align}
  \mc{H}^{\dagger}g_\text{v}(\rd{}) = \sigma_{\bs{\rho}}\mb{u}\mb{v}^{\dagger}g_\text{v}(\rd{}) = \mc{F}^{-1}\left\{\sum_{\text{v} = \{\text{A}, \text{B}\}}H_{\text{v}}(\bs{\rho})\mc{F}\left\{g_{\text{v}}(\ro{})\right\}\right\}. 
\end{align}
Once again this result is not too surprising, and I suspect that this is how the
existing MLEM/RL reconstructions have implemented the adjoint operator.

A potentially useful result is to decompose the Moore-Penrose pseudoinverse
operator
\begin{align}
  \mc{H}^+g_{\text{v}}(\rd{}) = \frac{1}{\sigma_{\bs{\rho}}}\mb{u}\mb{v}^{\dagger}g_{\text{v}}(\rd{}) = \mc{F}^{-1}\left\{\sum_{\text{v} = \{\text{A}, \text{B}\}}\frac{H_{\text{v}}(\bs{\rho})}{H_{\text{A}}^2(\bs{\rho}) + H_{\text{B}}^2(\bs{\rho})}\mc{F}\left\{g_{\text{v}}(\ro{})\right\}\right\}. \label{eq:mp}
\end{align}
This gives us a (new?) one-step reconstruction algorithm. In words:
\begin{enumerate}
\item Calculate or measure the kernel and transfer function of the microscope
  for both views and precalculate two filters:
  $H_{\text{A}}(\bs{\rho})/[H^2_{\text{A}}(\bs{\rho}) + H^2_{\text{B}}(\bs{\rho})]$ and $H_{\text{B}}(\bs{\rho})/[H^2_{\text{A}}(\bs{\rho}) + H^2_{\text{B}}(\bs{\rho})]$.
\item Take the 3D Fourier transform of the data from each view, multiply the
  result by the corresponding precalculated filter from step 1, then add the
  results.
\item Take the inverse Fourier transform. 
\end{enumerate}
This reconstruction is fast---it only requires three 3D Fourier transforms and a
single product---so it might be useful for live reconstructions during imaging.

The reconstruction has two limitations compared to MLEM/RL. First, the
Moore-Penrose pseudoinverse solution is a minimum-norm least-squares solution so
it assumes Gaussian noise and does not account for Poisson noise at low counts.
Second, the reconstruction is not constrained to positive values so noisy data
may cause a (usually slightly) negative value in the reconstructed object. These
limitations might not be a problem if an online reconstruction can help guide
the experiment while its happening and a complete MLEM/RL reconstruction can be
performed offline.

\section{Discussion}
The singular values of the dual view microscope confirm a result that matches
our intuition---the singular spectrum for a dual view microscope is the root sum
of squares of the transfer functions for each view. If we can describe the
imaging system as linear-shift invariant, I expect that this result will hold
for other multiview designs---the three-view diSPIM, possibly the mirror
diSPIM(?), and possibly unpolarized light-field designs. An analogous
calculation for the light-field microscope might yield a reconstruction
algorithm that is much faster than Broxton's spatial-domain reconstruction. TBD.
 
We've considered the diSPIM as a CC-CD imaging system and shown that the CD part
of the microscope is described by a rank-1 operator. Compare this with polarized
light microscopes where the angular part of the microscope is described by a
rank-3 (single-view) or rank-6 (dual-view) operator.

Finally, notice that the Moore-Penrose pseudoinverse operator in Eq.
\ref{eq:mp} reduces to a simple inverse filter if we only use the data from one
view.

\end{document}

