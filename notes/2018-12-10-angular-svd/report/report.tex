\documentclass[11pt]{article}

%%%%%%%%%%%%
% Packages %
%%%%%%%%%%%%
\usepackage[dvipsnames]{xcolor}
\hyphenpenalty=10000
\usepackage{tikz}
\usetikzlibrary{shapes,arrows}

\usepackage{tocloft}
\usepackage[linesnumbered,ruled]{algorithm2e}
\renewcommand\cftsecleader{\cftdotfill{\cftdotsep}}
\def\undertilde#1{\mathord{\vtop{\ialign{##\crcr
$\hfil\displaystyle{#1}\hfil$\crcr\noalign{\kern1.5pt\nointerlineskip}
$\hfil\tilde{}\hfil$\crcr\noalign{\kern1.5pt}}}}}
\usepackage{bold-extra}
\usepackage{bm}
\usepackage{cleveref}
\usepackage{xcolor}
\usepackage[colorlinks = true,
            linkcolor = black,
            urlcolor  = blue,
            citecolor = black,
            anchorcolor = black]{hyperref}
\usepackage{epstopdf}
\usepackage{braket}
\usepackage{upgreek}
\usepackage{caption}
\usepackage{booktabs}
\usepackage{subcaption}
\usepackage{amssymb,latexsym,amsmath,gensymb}
\usepackage{latexsym}
\usepackage{graphicx}
\usepackage{float}
\usepackage{enumitem}
\usepackage{pdflscape}
\usepackage{url}
\usepackage{array}
\newcolumntype{C}{>{$\displaystyle} c <{$}}
\usepackage{tikz, calc}
\usetikzlibrary{shapes.geometric, arrows, calc}
\tikzstyle{norm} = [rectangle, rounded corners, minimum width=2cm, minimum height=1cm,text centered, draw=black]
\tikzstyle{arrow} = [thick, ->, >=stealth]

\newcommand{\argmin}{\arg\!\min}
\newcommand{\argmax}{\arg\!\max}
\newcommand{\me}{\mathrm{e}}
\providecommand{\e}[1]{\ensuremath{\times 10^{#1}}}
\providecommand{\mb}[1]{\mathbf{#1}}
\providecommand{\mc}[1]{\mathcal{#1}}
\providecommand{\ro}{\mathbf{\mathfrak{r}}_o}
\providecommand{\so}{\mathbf{\hat{s}}_o}
\providecommand{\sd}{\mathbf{\hat{s}}_d}
\providecommand{\pp}{\mathbf{\hat{p}}}
\providecommand{\rb}{\mathbf{r}_b}
\providecommand{\rbm}{r_b^{\text{m}}}
\providecommand{\rd}{\mathbf{\mathfrak{r}}_d}
\providecommand{\mh}[1]{\mathbf{\hat{#1}}}
\providecommand{\mf}[1]{\mathfrak{#1}}
\providecommand{\ms}[1]{\mathsf{#1}}
\providecommand{\mbb}[1]{\mathbb{#1}}
\providecommand{\bs}[1]{\boldsymbol{#1}}
\providecommand{\tv}{\texttt{v}}
\providecommand{\tx}[1]{\text{#1}}
\providecommand{\tb}[1]{\textbf{#1}}
\providecommand{\ttt}[1]{\texttt{#1}}
\providecommand{\bv}{\bs{\nu}}
\providecommand{\bp}{\bs{\rho}}
\providecommand{\p}{\mh{p}}
\providecommand{\lmsum}{\sum_{l=0}^\infty\sum_{m=-l}^{l}}
\providecommand{\intr}[1]{\int_{\mbb{R}^{#1}}}
\providecommand{\ints}[1]{\int_{\mbb{S}^{#1}}}
\providecommand{\intinf}{\int_{-\infty}^{\infty}}
\let\originalleft\left
\let\originalright\right
\renewcommand{\left}{\mathopen{}\mathclose\bgroup\originalleft}
\renewcommand{\right}{\aftergroup\egroup\originalright}


\providecommand{\fig}[4]{
  % filename, width, caption, label
\begin{figure}[H]
 \captionsetup{width=1.0\linewidth}
 \centering
 \includegraphics[width = #2\textwidth]{#1}
 \caption{#3}
 \label{fig:#4}
\end{figure}
}

\makeatletter
\renewcommand*\env@matrix[1][*\c@MaxMatrixCols c]{%
  \hskip -\arraycolsep
  \let\@ifnextchar\new@ifnextchar
  \array{#1}}
\makeatother

\newcommand{\tensor}[1]{\overset{\text{\tiny$\leftrightarrow$}}{\mb{#1}}}
\newcommand{\tunderbrace}[2]{\underbrace{#1}_{\textstyle#2}}
\providecommand{\figs}[7]{
  % filename1, filename2, caption1, caption2, label1, label2, shift
\begin{figure}[H]
\centering
\begin{minipage}[b]{.45\textwidth}
  \centering
  \includegraphics[width=1.0\linewidth]{#1}
  \captionsetup{justification=justified, singlelinecheck=true}
  \caption{#3}
  \label{fig:#5}
\end{minipage}
\hspace{2em}
\begin{minipage}[b]{.45\textwidth}
  \centering
  \includegraphics[width=1.0\linewidth]{#2}
  \vspace{#7em}
  \captionsetup{justification=justified}
  \caption{#4}
  \label{fig:#6}
\end{minipage}
\end{figure}
}
\makeatletter

\providecommand{\code}[1]{
\begin{center}
\lstinputlisting{#1}
\end{center}
}

\newcommand{\crefrangeconjunction}{--}
%%%%%%%%%%%
% Spacing %
%%%%%%%%%%%
% Margins
\usepackage[
top    = 1.5cm,
bottom = 1.5cm,
left   = 1.5cm,
right  = 1.5cm]{geometry}

% Indents, paragraph space
%\usepackage{parskip}
\setlength{\parskip}{1.5ex}

% Section spacing
\usepackage{titlesec}
\titlespacing*{\title}
{0pt}{0ex}{0ex}
\titlespacing*{\section}
{0pt}{0ex}{0ex}
\titlespacing*{\subsection}
{0pt}{0ex}{0ex}
\titlespacing*{\subsubsection}
{0pt}{0ex}{0ex}

% Line spacing
\linespread{1.1}

%%%%%%%%%%%%
% Document %
%%%%%%%%%%%%
\begin{document}
\title{\vspace{-2.5em} Measuring the angular dipole spectrum with arbitrary\\
  polarized illumination microscopes\vspace{-1em}} % \author{Talon Chandler, Min Guo, Hari
  % Shroff, Rudolf Oldenbourg, Patrick La Rivi\`ere}
\date{\vspace{-3em}\today\vspace{-1em}}
\maketitle
\section{Forward model in a delta function basis}
We develop and analyze a simplified model of polarized excitation fluorescence
microscopes with arbitrary angular designs. We simplify the complete problem by
considering large objects and ignoring the irradiance pattern on the detector
(we effectively integrate over a few pixels). These simplifications are
equivalent to the implicit assumptions made by the majority of polarized
fluorescence microscopists. Our main goal is to improve our intuition for the
angular band limit of arbitrary microscopes and to incorporate light-sheet
tilting into our model.

With these simplifications we can model a continuous-to-continuous polarized
excitation fluorescence microscope system as
\begin{align}
  g(\sd, \pp) = \ints{2}d\so\, h(\sd,\pp;\so)f(\so),\label{eq:fwd}
\end{align}
where $\so$ is an orientation in the object, $f(\so)$ is the angular dipole
density (with units of sr${}^{-1}$), $\pp$ is the orientation of an excitation
polarizer, $\sd$ is the orientation of the detection optical axis, $g(\sd, \pp)$
is the measured irradiance with detection optical axis $\sd$ and excitation
polarization $\pp$. Equation \eqref{eq:fwd} assumes that object space is
$\mbb{U} = \mbb{L}_2(\mbb{S}^2)$---functions on the sphere---and data space is
$\mbb{V} = \mbb{L}_2(\mbb{S}^2 \times \mbb{S}^1)$---functions on the product
space of a sphere and a circle (a doughnut made of Timbits!).

Our next task is to find the form of the kernel $h(\sd,\pp;\so)$. We know that
the kernel is separable into excitation and detection parts so
\begin{align}
  h(\sd, \pp;\so) = h^{\text{exc}}(\pp;\so)h^{\text{det}}(\sd;\so). \label{eq:kern}
\end{align}
We restrict ourselves to low-NA excitation (like light-sheet excitation), which
gives the following excitation kernel
\begin{align}
  h^{\text{exc}}(\pp\cdot\so) \propto (\pp\cdot\so)^2. \label{eq:exckern}
\end{align}
In other words, the probability of exciting a fluorophore oriented along $\so$
with light polarized along $\pp$ is proportional to $(\pp\cdot\so)^2$---a
familiar $\cos^2\theta$ dependence where $\theta$ is the angle between $\pp$ and
$\so$. Notice that the excitation kernel is rotationally symmetric, so we can
write it as a function of $\pp\cdot\so$.

The detection kernel is given by
\begin{align}
  h^{\text{det}}(\sd,\so) = \ints{2}d\mh{s}\,\Pi\left(\frac{\mh{s}\cdot\sd}{\cos\alpha}\right)(\mh{s}\cdot\sd)^{-1/2}[1 - (\mh{s}\cdot\so)^2], \label{eq:detkern}
\end{align}
where the $[1 - (\mh{s}\cdot\so)^2]$ term models the familiar $\sin^2\vartheta$
dependence of dipole radiation where $\vartheta$ is the angle between $\sd$ and
$\so$, the $(\mh{s}\cdot\sd)^{-1/2}$ term models the apodization of an aplanatic
objective, and the integral over
$\Pi\left(\frac{\mh{s}\cdot\sd}{\cos\alpha}\right)$ models the finite aperture
of the objective with $\text{NA} = n_o\sin\alpha$. In Appendix
\ref{sec:spherical} we evaluate this integral in closed form to find that
\begin{align}
  h^{\text{det}}(\sd\cdot\so) \propto (1 - \sqrt{\cos\alpha})P_0(\sd\cdot\so) + \frac{1}{5}\left[\frac{1}{4}\sqrt{\cos\alpha}(7-3\cos(2\alpha)) - 1\right]P_2(\sd\cdot\so).
\end{align}
The detection kernel is rotationally symmetric, so it can be written as a
function of $\sd\cdot\so$. For very low NA the detection kernel reduces to
$[1 - (\mh{s}\cdot\so)^2]$ as we would expect.

\section{Forward model in a spherical harmonic basis} \label{sec:shsec}We can rewrite the right-hand side of Eq. \eqref{eq:fwd} in a spherical harmonic
basis by applying the generalized Plancherel theorem for spherical functions
\begin{align}
  g(\sd, \pp) = \sum_{\ell m}H_\ell^m(\sd, \pp)F_\ell^m,\label{eq:fwdsh}
\end{align}
where we have defined the \textit{dipole angular transfer function} as
\begin{align}
  H_\ell^m(\sd, \pp) = \ints{2}d\so\, h(\sd,\pp;\so)Y_\ell^{m*}(\so),
\end{align}
and the \textit{dipole angular spectrum} as
\begin{align}
  F_\ell^m = \ints{2}d\so\, f(\so)Y_\ell^{m*}(\so).
\end{align}

Our remaining task is to calculate the dipole angular transfer function. We can
start by expanding the kernel into excitation and detection parts
\begin{align}
  H_\ell^m(\sd, \pp) = \ints{2}d\so\, h^{\text{exc}}(\pp\cdot\so)h^{\text{det}}(\sd\cdot\so)Y_\ell^{m*}(\so). 
\end{align}
We can expand both the excitation and detection kernels into a sum of spherical harmonics (a Laplace series)
\begin{align}
    H_\ell^m(\sd, \pp) = \ints{2}d\so\, \left[\sum_{\ell' m'}H_{\ell'}^{m',\text{exc}}(\pp\cdot\so)Y_{\ell'}^{m'}(\so)\right]\left[\sum_{\ell'' m''}H_{\ell''}^{m'',\text{det}}(\sd\cdot\so)Y_{\ell''}^{m''}(\so)\right]Y_\ell^{m*}(\so),\label{eq:triple}
\end{align}
where $H_{\ell'}^{m',\text{exc}}(\pp\cdot\so)$ and $H_{\ell''}^{m'',\text{det}}(\sd\cdot\so)$ are the spherical harmonic transforms of the excitation and detection kernels
\begin{align}
  H_{\ell'}^{m',\text{exc}}(\pp) &= \ints{2}d\so\, h^{\text{exc}}(\pp\cdot\so)Y_{\ell'}^{m'*}(\so),\label{eq:sh1}\\
  H_{\ell''}^{m'',\text{det}}(\sd) &= \ints{2}d\so\, h^{\text{det}}(\sd\cdot\so)Y_{\ell''}^{m''*}(\so)\label{eq:sh2}.
\end{align}
In Appendix B we evaluate these spherical Fourier transforms as
\begin{align}
    H_{\ell'}^{m',\text{exc}}(\pp) &\propto Y_{\ell'}^{m'*}(\pp)\left[\delta_{0\ell'} + \frac{2}{5}\delta_{2,\ell'}\right],\label{eq:dipexc}\\
    H_{\ell''}^{m'',\text{det}}(\sd) &\propto Y_{\ell''}^{m''*}(\sd)\left[(1 - \sqrt{\cos\alpha})\delta_{0,\ell''} + \left(\frac{1}{4}\sqrt{\cos\alpha}(7-3\cos(2\alpha)) - 1\right)\delta_{2,\ell''}\right].\label{eq:dipdet}
\end{align}

We continue simplifying Eq. \eqref{eq:triple} by rewriting it as
\begin{align}
    H_\ell^m(\sd, \pp) = \sum_{\ell' m'}\sum_{\ell'' m''}H_{\ell'}^{m',\text{exc}}(\pp)H_{\ell''}^{m'',\text{det}}(\sd)\ints{2}d\so\, Y_\ell^{m*}(\so)Y_{\ell'}^{m'}(\so)Y_{\ell''}^{m''}(\so).\label{eq:triple2}
\end{align}
We can simplify this equation further as
\begin{align}
    H_\ell^m(\sd, \pp) = \sum_{\ell' m'}\sum_{\ell'' m''}(-1)^m G_{\ell,\ell',\ell''}^{-m,m',m''}H_{\ell'}^{m',\text{exc}}(\pp)H_{\ell''}^{m'',\text{det}}(\sd),\label{eq:triple3}
\end{align}
where $G_{\ell,\ell',\ell''}^{m,m',m''}$ are the Gaunt coefficients (sometimes
called the tripling coefficients or triple integrals of the spherical harmonics)
defined by
\begin{align}
  G_{\ell,\ell',\ell''}^{m,m',m''} = \ints{2}d\so\, Y_\ell^{m}(\so)Y_{\ell'}^{m'}(\so)Y_{\ell''}^{m''}(\so).
\end{align}
Equation \eqref{eq:triple3} is a key result because it allows us to calculate
the dipole angular transfer function from the excitation and detection angular
transfer functions.

We can draw some general conclusions about the band limits of this type of
imaging system by examining the properties of the Gaunt coefficients. First, the
Gaunt coefficient can only be non-zero when the coefficients $\ell$, $\ell'$,
and $\ell''$ satisfy the \textit{triangle condition}---each of the three
coefficients is between the sum and difference of the other two. The most
relevant condition for us is 
\begin{align}
  |\ell' - \ell''| \leq \ell \leq \ell' + \ell'', \label{eq:cond1}
\end{align}
which means that we can only have non-zero coefficients for $0\leq \ell \leq 4$
since $0\leq \ell' \leq 2$ and $0\leq \ell'' \leq 2$. Also, the Gaunt
coefficients can only be non-zero when the $\ell, \ell'$, and $\ell''$ form an
even sum
\begin{align}
  \ell + \ell' + \ell'' = 2n, \quad \text{for}\, n \in \mbb{N}, \label{eq:cond2}
\end{align}
which ensures inversion symmetry. Finally, the Gaunt coefficients can only be
non-zero when each $\ell$ is less than the absolute value of the corresponding
$m$
\begin{align}
  \ell \leq |m|,\quad\ell' \leq |m'|,\quad\ell'' \leq |m''|.\label{eq:cond3}
\end{align}
These conditions are obvious if you consider that the spherical harmonics are
zero (or undefined) when these conditions are not satisfied.

Given that $0\leq \ell' \leq 2$ and $0 \leq \ell'' \leq 2$ for our measurements, the
conditions in Eqs. \eqref{eq:cond1}--\eqref{eq:cond3} imply that the dipole
angular transfer function has at most 15 terms---1 for the $\ell=0$ band, 5 for
the $\ell=2$ band, and 9 for the $\ell=4$ band.

Note that a single choice for a measurement $\sd$ and $\pp$ can have 15 non-zero
terms in the dipole angular transfer function if both $\sd$ and $\pp$ do not lie
on a node (zero) of any of the $\ell = 2$ spherical harmonics. I mention this to
emphasize that the number of terms in the dipole angular transfer function has
little to no bearing on how much we can measure about the object---the choice of
coordinate system changes the nodes of the spherical harmonics and our results
should never depend on a choice of coordinates. The object space singular
vectors are the functions that we care about when we ask: what can this instrument 
measure about the object?

Notice that all of our results so far are completely coordinate-free---we have
used unit vectors everywhere instead of Cartesian or spherical coordinates. 

\section{Continuous-to-discrete model}
So far we have modeled the continuous data space
$\mbb{V} = \mbb{L}_2(\mbb{S}^2 \times \mbb{S}^1)$. Now we consider discrete
samples of this data space with the following forward model
\begin{align}
  g_i = \int_{\mbb{S}^2}d\sd\int_{\mbb{S}^1}d\pp\, w_i(\sd, \pp)\ints{2}d\so\,h(\sd,\pp;\so)f(\so),\label{eq:discrete}
\end{align}
where $g_i$ is a discrete set of $M$ measurements and $w_i(\sd, \pp)$ is the
``sampling aperture'' that indicates which part of data space $(\sd, \pp)$ we
sample during the $i$th measurement. For example, if we have a detector along
the $\mh{z}$ axis and we illuminate with $\mh{x}$ polarized light for the first
measurement and $\mh{y}$ polarized light for the second measurement, then the
sampling aperture is
\begin{align}
  w_i^{(0)}(\sd,\pp) = \delta(\sd - \mh{z})[\delta(\pp - \mh{x})\delta_{i,0} + \delta(\pp - \mh{y})\delta_{i,1}].
\end{align}
It is more convenient to rewrite Eq. \eqref{eq:discrete} as
\begin{align}
  g_i = \ints{2}d\so\,h_i(\so)f(\so),
\end{align}
where
\begin{align}
h_i(\so) = \int_{\mbb{S}^2}d\sd\int_{\mbb{S}^1}d\pp\, w_i(\sd, \pp) h(\sd,\pp;\so).
\end{align}
In our example, the continuous-to-discrete kernel is given by 
\begin{align}
  h_i^{(0)}(\so) = h(\mh{z},\mh{x};\so)\delta_{i,0} + h(\mh{z},\mh{y};\so)\delta_{i,1}.
\end{align}
\section{Singular value decomposition}
Our continuous-to-discrete forward model can be written in any of the following
equivalent forms
\begin{align}
  \mb{g} &= \mc{H}\mb{f},\label{eq:f1}\\
  g_i &= \ints{2}d\so\,h_i(\so)f(\so),\label{eq:f2}\\
  g_i &= \sum_{\ell,m}H_{i,\ell}^m F_{\ell}^m,\label{eq:f3}\\
  g_i &= \sum_{j}H_{i,j}F_j.\label{eq:f4}
\end{align}
where Eq. \eqref{eq:f1} is the Hilbert-space form, Eq. \eqref{eq:f2} is the
delta-function basis form, Eq. \eqref{eq:f3} is the spherical harmonic basis
form, and Eq. \eqref{eq:f3} is the spherical harmonic basis form with a
collapsed index $j(\ell,m) = [\ell(\ell+1)]/2 + m$ (don't confuse the spherical
Fourier transform with collapsed coefficients, $F_j$, with the Fourier-Legendre
transform, $F_\ell$---I'll consider alternative notations but it should be clear
from the context). Notice that the last two forms blur the line between a CD
operator and a DD operator---we are still considering a continuous object space
and discrete data space, but we have found a basis that allows us to express
this mapping in a completely discrete way.

Similarly, we can write the discrete-to-continuous adjoint operator as
\begin{align}
  \mb{f} &= \mc{H}^{\dagger}\mb{g},\label{eq:a1}\\
  f(\so) &= \sum_i h_i(\so)g_i,\label{eq:a2}\\
  F_{\ell}^m &= \sum_{i} H_{i,\ell}^m g_i,\label{eq:a3}\\
  F_j &= \sum_i H_{i,j}g_i,\label{eq:a4}
\end{align}
where the forms of Eqs. \eqref{eq:a1}--\eqref{eq:a4} match with the forms of Eqs. \eqref{eq:f1}--\eqref{eq:f4}.

To find the object space singular functions we need to solve the eigenvalue
problem in one of its forms
\begin{align}
  \mc{H}^\dagger\mc{H}\mb{u}_n &= \mu_n\mb{u}_n,\label{eq:s1}\\
  \sum_i h_i(\so)\ints{2} d\so'\, h_i(\so')u_n(\so') &= \mu_n u_n(\so),\label{eq:s2}\\
  \sum_i H_{i,\ell}^m \sum_{\ell' m'}, H_{i,\ell'}^{m'} U_{n,\ell'}^{m'} &= \mu_n U_{n,\ell}^m,\label{eq:s3}\\
  \sum_i H_{i,j} \sum_{j'}, H_{i,j'} U_{n,j'} &= \mu_n U_{n,j},\label{eq:s4}
\intertext{where $n$ indexes the singular values and singular functions, and the letter
``u'' is used with varying type and arguments to denote the object space
singular functions. We can tighten up the notation by introducing kernels of the
operators and denoting them with the letter ``k'' with varying type and
arguments:}
  \mc{K}\mb{u}_n &= \mu_n\mb{u}_n,\label{eq:k1}\\
  \ints{2} d\so'\, k(\so, \so')u_n(\so') &= \mu_n u_n(\so),\label{eq:k2}\\
  \sum_{\ell' m'} K_{\ell,\ell'}^{m,m'} U_{n,\ell'}^{m'} &= \mu_n U_{n,\ell}^m,\label{eq:k3}\\
  \sum_{j'} K_{j,j'} U_{n,j'} &= \mu_n U_{n,j},\label{eq:k4}
\intertext{where}
  \mc{K} &= \mc{H}\mc{H}^\dagger,\\
  k(\so,\so') &= \sum_i h_i(\so)h_i(\so'),\\
  K_{\ell,\ell'}^{m,m'} &= \sum_i H_{i,\ell}^m H_{i,\ell'}^{m'},\\
  K_{j,j'} &= \sum_i H_{i,j}H_{i,j'}. \label{eq:kk4}
\end{align}

Now we are in position to solve for the object space singular functions and the
singular values. The first step is to calculate the matrix $K_{j,j'}$ using Eq.
\eqref{eq:kk4} and the closed form expressions for the dipole angular transfer
function, $H_{i,j}$, that we calculated in Sec. \ref{sec:shsec}. As we discussed
previously, $H_{i,j}$ is a $M\times15$ matrix so $K_{j,j'}$ is $15\times15$ and
it is computationally trivial (though impossible analytically) to calculate its
eigenvalues $\mu_n$ and eigenvectors $U_{n,j}$. With $U_{n,j}$ in hand we can
calculate the object-space singular vectors in a delta function basis using an
inverse spherical harmonics transform
\begin{align}
  u_n(\so) = \sum_{j}U_{n,j}Y_j(\so).
\end{align}

Before examining specific designs, we briefly examine and interpret the object
space singular vectors and singular values. The object space singular vectors
span the space of measurable objects---they are orthogonal (and can be chosen to
be orthonormal) because the operator $\mc{H}^\dagger\mc{H}$ is Hermitian. The
singular values tell us ``how efficiently'' each object space singular vector is
transferred through the system---the singular values (the eigenvalues of
$\mc{H}^\dagger\mc{H}$) are real (because $\mc{H}^\dagger\mc{H}$ is Hermitian)
and positive (because of the form of $\mc{H}^\dagger\mc{H}$ see Barrett 1.4.4).
In other words, $\mc{H}^\dagger\mc{H}$ is positive-semidefinite because the
eigenvalues are greater than or equal to zero.

One desirable condition for the operator $\mc{H}^\dagger\mc{H}$ is that it has
the maximum possible rank---$R=15$ in our case which is limited by the
excitation and detection physics. Achieving $R=15$ will always require at least
15 distinct measurements which is far more than we would like, so we search for
designs with fewer measurements.

Another desirable condition is that the object space singular vectors span
subspaces of $\mbb{L}_2(\mbb{S}^2)$ that are invariant under rotation. The
spherical harmonic bands are exactly the subspaces we're looking for: the
$\ell$th band spans a $(2\ell + 1)$-dimension space that is invariant under
rotation---equivalently, the $\ell$th band forms a ($2\ell +1$)-dimensional
representation for the rotation group. Furthermore, we know that the $\ell$th
band is an \textit{irreducible} representation of the rotation group, which
means that we cannot decompose the $\ell$th band of spherical harmonics to find
smaller representations of the rotation group. Therefore, if we would like our
object space singular vectors to span a rotationally-invariant subspace then we
need to take at least 1, 6, or 15 measurements.

% We would like to find the minimum number of measurements required for the object
% space singular vectors to span the $\ell=0$ and $\ell=2$ rotationally invariant
% subspaces. We always measure a linear combination of spherical harmonic
% coefficients from the $\ell=0,2,4$ bands, so we need to make more than 6
% measurements because we are always measuring at least one dimension from the
% $\ell=4$ band. A minimal measurement scheme will only have singular vectors that
% span a single dimension in the 9-dimensional $\ell=4$ subspace, so the minimum
% number of measurements for $\ell=0$ and $\ell=2$ rotationally invariant singular
% vectors is 7.

% How do we choose measurements with singular vectors that span a single dimension
% of the $\ell=4$ band? Surprisingly, the only requirement is that we fix $\pp$ or
% $\sd$. To demonstrate this we'll fix $\sd$ and choose a coordinate system so
% that $\mh{z} = \sd$. In this case the dipole excitation and dipole detection transfer
% functions (see Eqs. \eqref{eq:dipexc} and \eqref{eq:dipdet}) are
%   \begin{align}
%     H_{\ell'}^{m',\text{exc}}(\pp) &\propto Y_{\ell'}^{m'*}(\pp)\left[\delta_{0\ell'} + \frac{2}{5}\delta_{2,\ell'}\right],\label{eq:dipexc2}\\
%     H_{\ell''}^{m'',\text{det}}(\mh{z}) &\propto \delta_{m'',0}\left[(1 - \sqrt{\cos\alpha})\delta_{0,\ell''} + \left(\frac{1}{4}\sqrt{\cos\alpha}(7-3\cos(2\alpha)) - 1\right)\delta_{2,\ell''}\right].\label{eq:dipdet2}
% \end{align}
% Notice that the detection transfer function is only non-zero for $m'' = 0$
% because of our choice of coordinate system. To find the complete dipole angular transfer function we can use the equation
% \begin{align}
%     H_\ell^m(\mh{z}, \pp) = \sum_{\ell' m'}\sum_{\ell'' m''}(-1)^m G_{\ell,\ell',\ell''}^{-m,m',m''}H_{\ell'}^{m',\text{exc}}(\pp)H_{\ell''}^{m'',\text{det}}(\mh{z}).
% \end{align}
% We care about the Gaunt coefficients that put terms in the $\ell=4$ band
% $G_{4,2,2}^{m,m',m''}$. We've chosen a coordinate system so that $m''=0$ and
% from a quick computation of the Gaunt coefficients we know that 
% \begin{align}
%   G_{4,2,2}^{m,m',0} = \frac{3}{7\sqrt{pi}}\delta_{m,0}\delta{m',0}
% \end{align}


\section{Specific designs (in progress)}
one-view epi,
one-view lightsheet,
two-view lightsheet,
two-view lightsheet two-tilt one-side,
two-view lightsheet two-tilt two-side,
two-view lightsheet three-tilt one-side two-tilt one-side.

\appendix
\section{Spherical integral of the product of two rotationally symmetric
  functions} \label{sec:spherical}

Consider two rotationally symmetric functions on the sphere
$f(\mh{s}\cdot\mh{s}')$ and $g(\mh{s}\cdot\mh{s}'')$. In this appendix we will
reduce integrals of the form
\begin{align}
  g(\mh{s}', \mh{s}'') = \ints{2}d\mh{s}\,h(\mh{s}\cdot\mh{s}')f(\mh{s}\cdot\mh{s}'')\label{eq:integral}
\end{align}
to two one-dimensional integrals. The key step is to expand both $f$ and $g$
into a Fourier-Legendre series
\begin{align}
  f(x) = \sum_{\ell=0}^{\infty}F_{\ell}P_{\ell}(x), \label{eq:flseries}
\end{align}
where $P_{\ell}(x)$ are the Legendre polynomials. We can find
the coefficients $F_{\ell}$ using Fourier's trick---multiply both sides of Eq.
\eqref{eq:flseries} by $P_{\ell}(x)$ (the Legendre polynomials
are real so we can ignore complex conjugates) then integrate over the interval
$[-1,1]$ and apply the orthogonality relation
\begin{align}
  \int_{-1}^1dx\,P_\ell(x)P_{\ell'}(x) = \frac{2}{2\ell+1}\delta_{\ell\ell'},
\end{align}
which implies that the Legendre polynomials are orthogonal but not orthonormal.
After applying these steps to Eq. \eqref{eq:flseries} we find that
\begin{align}
  F_{\ell} = \frac{2\ell+1}{2}\int_{-1}^1dx\,f(x)P_{\ell}(x).
\end{align}
The coefficients $F_{\ell}$ are often referred to as the
\textit{Fourier-Legendre transform} of $f(x)$. Now we can expand $f$ and $g$ in
Eq. \eqref{eq:integral} into two Fourier-Legendre series
\begin{align}
  g(\mh{s}', \mh{s}'') &= \ints{2}d\mh{s}\,\left[\sum_{\ell} H_\ell P_\ell (\mh{s}\cdot\mh{s}')\right]\left[\sum_{\ell'}F_{\ell'}P_{\ell'}(\mh{s}\cdot\mh{s}'')\right],\\
  &=\sum_{\ell}\sum_{\ell'} \ints{2}d\mh{s}\,H_\ell F_{\ell'} P_\ell (\mh{s}\cdot\mh{s}')P_{\ell'}(\mh{s}\cdot\mh{s}'').\label{eq:double}
\end{align}
The next step is to expand the Legendre polynomials in terms of spherical harmonics using the \textit{spherical harmonic addition theorem}
\begin{align}
  P_\ell(\mh{s}\cdot\mh{s}') = \frac{4\pi}{2\ell + 1}\sum_{m=-\ell}^{\ell}Y_\ell^m(\mh{s})Y_\ell^{m*}(\mh{s}'). \label{eq:shaddition}
\end{align}
Applying Eq. \eqref{eq:shaddition} to Eq. \eqref{eq:double} yields
\begin{align}
  g(\mh{s}',\mh{s}'') &= \sum_{\ell}\sum_{\ell'} \ints{2}d\mh{s}\,H_\ell F_{\ell'} \left[\frac{4\pi}{2\ell + 1}\sum_{m=-\ell}^{\ell}Y_\ell^m(\mh{s})Y_\ell^{m*}(\mh{s}')\right]\left[\frac{4\pi}{2\ell' + 1}\sum_{m=-\ell'}^{\ell'}Y_{\ell'}^{m'}(\mh{s}'')Y_{\ell'}^{m'*}(\mh{s})\right],\label{eq:one}\\ 
                      &= \sum_{\ell m}\sum_{\ell' m'} H_\ell F_{\ell'} \left[\frac{4\pi}{2\ell + 1}Y_\ell^{m*}(\mh{s}')\right]\left[\frac{4\pi}{2\ell' + 1}Y_{\ell'}^{m'}(\mh{s}'')\right]\ints{2}d\mh{s}Y_\ell^m(\mh{s})Y_{\ell'}^{m'*}(\mh{s}),\label{eq:two}\\
                      &= \sum_{\ell m}\sum_{\ell' m'} H_\ell F_{\ell'} \left[\frac{4\pi}{2\ell + 1}Y_\ell^{m*}(\mh{s}')\right]\left[\frac{4\pi}{2\ell' + 1}Y_{\ell'}^m(\mh{s}'')\right]\delta_{\ell\ell'}\delta_{m m'},\label{eq:three}\\
                      &= \sum_{\ell m}H_\ell F_{\ell} \left(\frac{4\pi}{2\ell + 1}\right)^2 Y_\ell^{m*}(\mh{s}')Y_{\ell'}^m(\mh{s}''),\label{eq:four}\\
  g(\mh{s}'\cdot\mh{s}'') &= \sum_{\ell}\frac{4\pi}{2\ell + 1}H_\ell F_{\ell} P_\ell(\mh{s}'\cdot\mh{s}'').\label{eq:five}
\end{align}
where we have (\ref{eq:one}-\ref{eq:two}) rearranged the equation,
(\ref{eq:two}-\ref{eq:three}) used the orthogonality of spherical harmonics,
(\ref{eq:three}-\ref{eq:four}) used the discrete sifting theorem then,
(\ref{eq:four}-\ref{eq:five}) applied the spherical harmonic addition theorem.

Eq. \eqref{eq:five} is a key result. In words, we can compute the spherical
integral of the product of two rotationally symmetric functions by taking their
Fourier-Legendre transforms then summing over $\ell$ with appropriate weights.
Perhaps surprisingly, the result only depends on $\mh{s}'\cdot\mh{s}''$ so it is
a rotationally symmetric function, too.  

We can use this result to evaluate the detection kernel
\begin{align}
  h^{\text{det}}(\sd,\so) = \ints{2}d\mh{s}\,\Pi\left(\frac{\mh{s}\cdot\sd}{\cos\alpha}\right)(\mh{s}\cdot\sd)^{-1/2}[1 - (\mh{s}\cdot\so)^2]. \label{eq:kern}
\end{align}
Comparing Eq. \eqref{eq:kern} to Eq. \eqref{eq:integral} means that
\begin{align}
  f(\mh{s}\cdot\sd) &= 1 - (\mh{s}\cdot\so)^2,  \\
  h(\mh{s}\cdot\sd) &= \Pi\left(\frac{\mh{s}\cdot\sd}{\cos\alpha}\right)(\mh{s}\cdot\sd)^{-1/2}.
\end{align}
Evaluating the Fourier-Legendre transform of these functions gives
\begin{align}
  F_\ell &= \frac{2\ell + 1}{2}\int_{-1}^1dx\,(1 - x^{2})P_\ell(x) = \frac{2}{3}(\delta_{\ell,0} + \delta_{\ell,2}),\label{eq:fl}\\
  H_\ell &= \frac{2\ell + 1}{2}\int_{\cos\alpha}^1dx\,x^{-1/2}P_\ell(x) = (1 - \sqrt{\cos\alpha})\delta_{\ell,0} + \left[\frac{1}{4}\sqrt{\cos\alpha}(7-3\cos(2\alpha)) - 1\right]\delta_{\ell,2}+ \cdots,\label{eq:hl}
\end{align}
Notice that we only need to evaluate the first two terms of $H_\ell$ since
only two terms of $F_\ell$ are non-zero. As we've noted before, the physical
dipole emission process is the band limiting step here---not the detection
process.

Plugging Eqs. \eqref{eq:fl} and \eqref{eq:hl} into Eq. \eqref{eq:five} gives the
closed form result for the detection kernel
\begin{align}
  h^{\text{det}}(\sd\cdot\so) \propto (1 - \sqrt{\cos\alpha})P_0(\sd\cdot\so) + \frac{1}{5}\left[\frac{1}{4}\sqrt{\cos\alpha}(7-3\cos(2\alpha)) - 1\right]P_2(\sd\cdot\so).
\end{align}

\section{Spherical Fourier transform of a rotationally symmetric function}
Consider a rotationally symmetric function on the sphere
$f(\mh{s}\cdot\mh{s}')$. In this appendix we will reduce the spherical Fourier
transform
\begin{align}
  F_\ell^m = \ints{2}d\mh{s}f(\mh{s}\cdot\mh{s}')Y_\ell^{m*}(\mh{s})
\end{align}
to a single one-dimensional integral. Similar to Appendix A, we expand $f$ in a Fourier-Legendre series and proceed step by step
\begin{align}
  F_\ell^m &= \ints{2}d\mh{s}\left[\sum_{\ell'}F_{\ell'} P_{\ell'}(\mh{s}\cdot\mh{s}')\right]Y_\ell^{m*}(\mh{s}),\label{eq:one1}\\
  F_\ell^m &= \sum_{\ell'}F_{\ell'}\ints{2}d\mh{s}\,P_{\ell'}(\mh{s}\cdot\mh{s}')Y_\ell^{m*}(\mh{s}),\label{eq:two1}\\
  F_\ell^m &= \sum_{\ell'm'}\frac{4\pi}{2\ell' + 1}F_{\ell'}Y_{\ell'}^{m'*}(\mh{s}')\ints{2}d\mh{s}\,Y_{\ell'}^{m'}(\mh{s})Y_\ell^{m*}(\mh{s}),\label{eq:three1}\\
  F_\ell^m &= \sum_{\ell'm'}\frac{4\pi}{2\ell' + 1}F_{\ell'}Y_{\ell'}^{m'*}(\mh{s}')\delta_{\ell\ell'}\delta_{mm'},\label{eq:four1}\\
  F_\ell^m &= \frac{4\pi}{2\ell + 1}Y_{\ell}^{m*}(\mh{s}')F_{\ell}\label{eq:five1},
\end{align}
where we have (\ref{eq:one1}-\ref{eq:two1}) rearranged the equation,
(\ref{eq:two1}-\ref{eq:three1}) applied the spherical harmonic addition theorem,
(\ref{eq:three1}-\ref{eq:four1}) used the orthogonality of spherical harmonics
then, (\ref{eq:four1}-\ref{eq:five1}) used the discrete sifting theorem. Eq.
\eqref{eq:five1} is the result we were seeking---we can find the spherical
Fourier transform of a rotationally symmetric function by finding the
Fourier-Bessel transform $F_\ell$ then multiplying by
$\frac{4\pi}{2\ell + 1}Y_{\ell}^{m*}(\mh{s}')$.

We can use this result to evaluate the excitation and detection angular transfer
functions
\begin{align}
  H_{\ell'}^{m',\text{exc}}(\pp\cdot\so) &= \ints{2}d\so\, h^{\text{exc}}(\pp\cdot\so)Y_{\ell'}^{m'*}(\so),\\
  H_{\ell''}^{m'',\text{det}}(\sd\cdot\so) &= \ints{2}d\so\, h^{\text{det}}(\sd\cdot\so)Y_{\ell''}^{m''*}(\so).
\end{align}
where
\begin{align}
  h^{\text{exc}}(\pp\cdot\so) &\propto (\pp\cdot\so)^2, \\
  h^{\text{det}}(\sd\cdot\so) &\propto (1 - \sqrt{\cos\alpha})P_0(\sd\cdot\so) + \frac{1}{5}\left[\frac{1}{4}\sqrt{\cos\alpha}(7-3\cos(2\alpha)) - 1\right]P_2(\sd\cdot\so).
\end{align}
Starting with the excitation kernel
\begin{align}
  H_{\ell'}^{m',\text{exc}}(\pp) &= \frac{4\pi}{2\ell + 1}Y_{\ell'}^{m'*}(\pp)\left[\frac{2\ell' + 1}{2}\int_{-1}^1dx\, x^2P_{\ell'}(x)\right],\\
  H_{\ell'}^{m',\text{exc}}(\pp) &=\frac{4\pi}{2\ell + 1}Y_{\ell'}^{m'*}(\pp)\left[\frac{1}{3}\delta_{0\ell'} + \frac{2}{3}\delta_{2,\ell'}\right],\\
  H_{\ell'}^{m',\text{exc}}(\pp) &\propto Y_{\ell'}^{m'*}(\pp)\left[\delta_{0\ell'} + \frac{2}{5}\delta_{2,\ell'}\right].
\end{align}
Notice that the polarizer orientation $\pp$ only affects the $m$ components of
the angular transfer function while the relative contribution to the two $\ell$ bands is fixed.

Finally, the detection kernel is
\begin{align}
  H_{\ell''}^{m'',\text{det}}(\sd) &= \frac{4\pi}{2\ell'' + 1}Y_{\ell''}^{m''*}(\sd)\left[\frac{2\ell'' + 1}{2}\int_{-1}^1dx\, h^{\text{det}}(x)P_{\ell''}(x)\right],\\
  H_{\ell''}^{m'',\text{det}}(\sd) &= \frac{4\pi}{2\ell'' + 1}Y_{\ell''}^{m''*}(\sd)\left[(1 - \sqrt{\cos\alpha})\delta_{0,\ell''} + \frac{1}{5}\left(\frac{1}{4}\sqrt{\cos\alpha}(7-3\cos(2\alpha)) - 1\right)\delta_{2,\ell''}\right],\\
  H_{\ell''}^{m'',\text{det}}(\sd) &\propto Y_{\ell''}^{m''*}(\sd)\left[(1 - \sqrt{\cos\alpha})\delta_{0,\ell''} + \left(\frac{1}{4}\sqrt{\cos\alpha}(7-3\cos(2\alpha)) - 1\right)\delta_{2,\ell''}\right].
\end{align}
Notice that the detection axis $\sd$ only affects the $m$ components of the
detection axis and the detection NA affects the relative contribution to the two $\ell$ bands. 

\end{document}

