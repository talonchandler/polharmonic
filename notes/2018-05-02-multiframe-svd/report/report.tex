
\documentclass[11pt]{article}

%%%%%%%%%%%%
% Packages %
%%%%%%%%%%%%
\usepackage[dvipsnames]{xcolor}
\hyphenpenalty=10000
\usepackage{tikz}
\usetikzlibrary{shapes,arrows}

\usepackage{tocloft}
\renewcommand\cftsecleader{\cftdotfill{\cftdotsep}}
\def\undertilde#1{\mathord{\vtop{\ialign{##\crcr
$\hfil\displaystyle{#1}\hfil$\crcr\noalign{\kern1.5pt\nointerlineskip}
$\hfil\tilde{}\hfil$\crcr\noalign{\kern1.5pt}}}}}
\usepackage{cleveref}
\usepackage{xcolor}
\usepackage[colorlinks = true,
            linkcolor = black,
            urlcolor  = blue,
            citecolor = black,
            anchorcolor = black]{hyperref}
\usepackage{epstopdf}
\usepackage{braket}
\usepackage{upgreek}
\usepackage{caption}
\usepackage{booktabs}
\usepackage{subcaption}
\usepackage{amssymb,latexsym,amsmath,gensymb}
\usepackage{latexsym}
\usepackage{graphicx}
\usepackage{float}
\usepackage{enumitem}
\usepackage{pdflscape}
\usepackage{url}
\usepackage{array}
\newcolumntype{C}{>{$\displaystyle} c <{$}}
\usepackage{tikz, calc}
\usetikzlibrary{shapes.geometric, arrows, calc}
\tikzstyle{norm} = [rectangle, rounded corners, minimum width=2cm, minimum height=1cm,text centered, draw=black]
\tikzstyle{arrow} = [thick, ->, >=stealth]

\newcommand{\argmin}{\arg\!\min}
\newcommand{\me}{\mathrm{e}}
\providecommand{\e}[1]{\ensuremath{\times 10^{#1}}} 
\providecommand{\mb}[1]{\mathbf{#1}}
\providecommand{\mf}[1]{\mathbf{#1}}
\providecommand{\ro}[1]{\mathbf{\mathbf{r}}_o}
\providecommand{\so}[1]{\mathbf{\hat{s}}_o}
\providecommand{\rb}[1]{\mathbf{r}_b}
\providecommand{\rbm}[1]{r_b^{\text{m}}}
\providecommand{\rd}[1]{\mathbf{r}_d}
\providecommand{\mh}[1]{\mathbf{\hat{#1}}}
\providecommand{\mbb}[1]{\mathbb{#1}}
\providecommand{\bs}[1]{\boldsymbol{#1}} 
\providecommand{\intinf}{\int_{-\infty}^{\infty}}
\providecommand{\fig}[4]{
  % filename, width, caption, label
\begin{figure}[h]
 \captionsetup{width=1.0\linewidth}
 \centering
 \includegraphics[width = #2\textwidth]{#1}
 \caption{#3}
 \label{fig:#4}
\end{figure}
}

\makeatletter
\renewcommand*\env@matrix[1][*\c@MaxMatrixCols c]{%
  \hskip -\arraycolsep
  \let\@ifnextchar\new@ifnextchar
  \array{#1}}
\makeatother

\newcommand{\tensor}[1]{\overset{\text{\tiny$\leftrightarrow$}}{\mb{#1}}}
\newcommand{\tunderbrace}[2]{\underbrace{#1}_{\textstyle#2}}
\providecommand{\figs}[7]{
  % filename1, filename2, caption1, caption2, label1, label2, shift
\begin{figure}[H]
\centering
\begin{minipage}[b]{.45\textwidth}
  \centering
  \includegraphics[width=1.0\linewidth]{#1}
  \captionsetup{justification=justified, singlelinecheck=true}
  \caption{#3}
  \label{fig:#5}
\end{minipage}
\hspace{2em}
\begin{minipage}[b]{.45\textwidth}
  \centering
  \includegraphics[width=1.0\linewidth]{#2}
  \vspace{#7em}
  \captionsetup{justification=justified}
  \caption{#4}
  \label{fig:#6}
\end{minipage}
\end{figure}
}
\makeatletter

\providecommand{\code}[1]{
\begin{center}
\lstinputlisting{#1}
\end{center}
}

\newcommand{\crefrangeconjunction}{--}
%%%%%%%%%%%
% Spacing %
%%%%%%%%%%%
% Margins
\usepackage[
top    = 1.5cm,
bottom = 1.5cm,
left   = 1.5cm,
right  = 1.5cm]{geometry}

% Indents, paragraph space
%\usepackage{parskip}
\setlength{\parskip}{1.5ex}

% Section spacing
\usepackage{titlesec}
\titlespacing*{\title}
{0pt}{0ex}{0ex}
\titlespacing*{\section}
{0pt}{0ex}{0ex}
\titlespacing*{\subsection}
{0pt}{0ex}{0ex}
\titlespacing*{\subsubsection}
{0pt}{0ex}{0ex}

% Line spacing
\linespread{1.1}

%%%%%%%%%%%%
% Document %
%%%%%%%%%%%%
\begin{document}
\title{\vspace{-2.5em} Singular value decomposition of multiframe\\ polarized
  fluorescence microscopes\vspace{-1em}} \author{Talon Chandler, Min Guo, Hari
  Shroff, Rudolf Oldenbourg, Patrick La Rivi\`ere}
\date{\vspace{-1em}\today\vspace{-1em}}
\maketitle
\section{Introduction}
In these notes we will develop the continuous models for several multiframe
polarized fluorescence microscopes. For each design we will calculate the
spatio-angular point spread function, optical transfer function, and singular
system consisting of the singular values, object-space singular functions, and
data-space singular functions.

All of the microscopes we will consider are imaging fields of oriented
fluorophores. To a good approximation any field of oriented fluorophores can be
represented by a member of the set
$\mbb{U} = \mbb{L}_2(\mbb{R}^3 \times \mbb{S}^2)$---square-integrable functions
that assign a scalar value to each position and orientation.

We will be considering multiframe microscopes that capture multiple images of
the same object (we assume that the object is static over the imaging time). The
data for the $i$th frame can be represented by a member of
$\mbb{V}_i = \mbb{L}_2(\mbb{R}^{N})$---square-integrable functions that assign a
scalar value to each point in a $2$-dimensional Euclidean space. If the
microscope collects $N$ frames, then all of the data can be represented by a
member of the larger set $\mbb{V} = \mbb{L}_2(\mbb{R}^{2N})$---square-integrable
functions that assign a scalar value to each point in a $2N$-dimensional
Euclidean space. We can say that the complete data space $\mbb{V}$ is built by
taking the \textit{orthogonal direct sum} of the data space for each frame
$\mbb{V}_i$
\begin{align}
  \mbb{V} = \bigoplus_{i=1}^N \mbb{V}_i.
\end{align}
Notice that we are assuming that data space is continuous---we
are ignoring the effects of finite pixels and a finite field-of-view.

We can model any linear relationship between object space and data space using
an integral transform
\begin{align}
  g(\mathfrak{r}_d) = \int_{\mbb{S}^2}d\so{}\int_{\mbb{R}^3}d\ro{}\, h(\mathfrak{r}_d; \ro{}, \so{})f(\ro{}, \so{}), \label{eq:full}
\end{align}
where $g(\mathfrak{r}_d) \in \mbb{U}$ is the data, $f(\ro{}, \so{}) \in \mbb{V}$
is the object, and $h(\mathfrak{r}_d; \ro{}, \so{}) \in \mbb{U} \times \mbb{V}$ is
the point response function of the imaging system. Notice that $\mathfrak{r}_d$
is a 2$N$-dimensional coordinate where the first two dimensions are the
coordinates of the first image, the next two dimensions are the coordinates of
the second image, and so on. Usually it will be more convenient for us to
rewrite Eq.~\ref{eq:full} as the sum of $N$ terms
\begin{align}
    g(\mathfrak{r}_d) = \sum_{i=1}^N \int_{\mbb{S}^2}d\so{}\int_{\mbb{R}^3}d\ro{}\, h_i(\rd{}; \ro{}, \so{})f(\ro{}, \so{}), \label{eq:sys2}
\end{align}
where $\rd{}$ is a 2-dimensional coordinate for each frame. We can also rewrite
the forward model as $N$ separate equations
\begin{align}
    g_i(\rd{}) = \int_{\mbb{S}^2}d\so{}\int_{\mbb{R}^3}d\ro{}\, h_i(\rd{}; \ro{}, \so{})f(\ro{}, \so{}),\qquad i=1, 2,\ldots,N, \label{eq:system}
\end{align}
but this is potentially less convenient than Eqs. \ref{eq:full} and
\ref{eq:sys2} because it hides the fact there is a single integral transform
that relates object space to data space.

In these notes we'll only be considering shift-invariant microscopes, so we can
simplify the model to
\begin{align}
    g(\mathfrak{r}_d) = \sum_{i=1}^N \int_{\mbb{S}^2}d\so{}\int_{\mbb{R}^3}d\ro{}\, h_i(\rd{} - \ro{}, \so{})f(\ro{}, \so{}). \label{eq:sys3}
\end{align}
We can rewrite the complete forward model in Eq. \ref{eq:sys3}) more compactly
using operator notation
\begin{align}
  g(\mathfrak{r}_d) = [\mathcal{H}\mathbf{f}](\mathfrak{r}_d).
\end{align}
In these notes we will ignore magnification---in the previous notes we showed
that we can make a change of variables that puts a system with magnification in
the form of a system without magnification. In other words, we can drop the
primes that indicated magnified quantities in previous note sets. We will also
restrict ourselves to the paraxial approximation and drop the $(p)$
superscripts.

\section{Polarized epi-illumination with unpolarized epi-detection}
We will start by restricting our analysis of epi-illumination microscopes to
in-focus objects. This means that our object space is
$\mbb{L}_2(\mbb{R}^2 \times \mbb{S}^2)$.

\subsection{Point response function}
In the previous notes we showed that the excitation point response function for
polarized epi-illumination is given by
\begin{align}
  h^{\mh{z}}_{\text{exc}}(\so{}; \mh{p}) &= y_0^0(\so{}) - \frac{1}{\sqrt{5}}\tilde{A}y_2^0(\so{}) + \sqrt{\frac{3}{5}}\tilde{B}\left\{[(\mh{p}\cdot\mh{x})^2 - (\mh{p}\cdot\mh{y})^2]y_2^2(\so{}) - 2(\mh{p}\cdot\mh{x})(\mh{p}\cdot\mh{y})y_2^{-2}(\so{})\right\}, \label{eq:genpsf}
\end{align}
where
\begin{subequations}
\begin{align}
  \tilde{A} &\equiv \cos^2(\alpha/2)\cos(\alpha),\\
  \tilde{B} &\equiv \frac{1}{12}(\cos^2\alpha + 4\cos\alpha + 7),
\end{align}\label{eq:coefficients}%
\end{subequations}
and $\alpha \equiv \arcsin(\text{NA}/n_o)$.

We also showed that the point response function for unpolarized epi-detection is
given by
\begin{align}
  h_{\text{det}}(\ro{}, \so{}) &= [{a}^2(r_o) + 2b^2(r_o)]y_0^0(\so{}) + \frac{1}{\sqrt{5}}\left[- a^2(r_o) + 4b^2(r_o)\right]y_2^0(\so{}),
\end{align}
where
\begin{align}
  a(r_o) = \frac{J_1(2\pi \nu_or_o)}{\pi \nu_or_o}, 
  &\hspace{2em}
    b(r_o) = \frac{\text{NA}}{n_o}\left[\frac{J_2(2\pi \nu_or_o)}{\pi \nu_or_o}\right],  \label{eq:abparadef}
  \intertext{and}
  \nu_o \equiv \frac{\text{NA}}{\lambda},&\hspace{2em}
  \text{NA} = n_o\sin\alpha.
\end{align}
The excitation and detection processes are incoherent, so to find the complete
point response function we can multiply the excitation and detection point response
functions which gives
\begin{align}
  h(\ro{}, \so{}; \mh{p}) = h^{\mh{z}}_{\text{exc}}(\so{}; \mh{p})h_{\text{det}}(\ro{}, \so{}) = \frac{1}{\tilde{N}}\sum_{l=0}^{\infty}\sum_{m=-l}^l h_l^m(\ro{}; \mh{p})y_l^m(\so{}), 
\end{align}
where $\tilde{N} = \frac{\tilde{A}}{10} + \frac{1}{2}$ is a normalization constant
and the terms in the series are given by
\begin{align}
  h_0^0(r_o) &= \left[\frac{\tilde{A}}{10} + \frac{1}{2}\right]a^2(r_o) + \left[-\frac{2\tilde{A}}{5} + 1\right]b^2(r_o),\\
  h_2^{-2}(r_o; \mh{p}) &= \left[\frac{9\sqrt{15}}{70}a^2(r_o) + \frac{3\sqrt{15}}{35}b^2(r_o)\right]\tilde{B}[(\mh{p}\cdot\mh{x})^2 - (\mh{p}\cdot\mh{y})^2],\\
  h_2^0(r_o) &= \left[-\frac{\sqrt{5}\tilde{A}}{14} + \frac{\sqrt{5}}{10}\right]a^2(r_o) + \left[-\frac{11\sqrt{5}\tilde{A}}{35} + \frac{2}{\sqrt{5}}\right]b^2(r_o),\\
  h_2^2(r_o; \mh{p}) &= -2\left[\frac{9\sqrt{15}}{70}a^2(r_o) + \frac{3\sqrt{15}}{35}b^2(r_o)\right]\tilde{B}(\mh{p}\cdot\mh{x})(\mh{p}\cdot\mh{y}),\\
  h_4^{-2}(r_o; \mh{p}) &= \left[-\frac{3\sqrt{5}}{70}a^2(r_o) + \frac{6\sqrt{5}}{35}b^2(r_o)\right]\tilde{B}[(\mh{p}\cdot\mh{x})^2 - (\mh{p}\cdot\mh{y})^2],\\
  h_4^0(r_o) &= \frac{3\tilde{A}}{35}[a^2(r_o) - 4b^2(r_o)],\\
  h_4^2(r_o; \mh{p}) &= -2\left[-\frac{3\sqrt{5}}{70}a^2(r_o) + \frac{6\sqrt{5}}{35}b^2(r_o)\right]\tilde{B}(\mh{p}\cdot\mh{x})(\mh{p}\cdot\mh{y}),
\end{align}
and all other $h_l^m$ terms in the series are zero. Finally, the complete
forward model for this class of microscope is given by
\begin{align}
    g(\mathfrak{r}_d) = \sum_{i=1}^N\int_{\mbb{S}^2}d\so{}\int_{\mbb{R}^3}d\ro{}\, h(\rd{} - \ro{}, \so{}; \mh{p}_i)f(\ro{}, \so{}),\qquad i=1, 2,\ldots,N. \label{eq:fwdpolillum}
\end{align}
Notice that we have used a single function $h(\rd{} - \ro{}, \so{}; \mh{p}_i)$
to describe the complete imaging system, and we have shifted the index $i$ to
the illumination polarization. A typical choice of frames is given by $N=4$ and
\begin{align}
  \mh{p}_1 = \mh{x},\hspace{2em}
  \mh{p}_2 = \frac{1}{\sqrt{2}}(\mh{x} + \mh{y}),\hspace{2em}
  \mh{p}_3 = \mh{y},\hspace{2em}
  \mh{p}_4 = \frac{1}{\sqrt{2}}(\mh{x} - \mh{y}). 
\end{align}
\indent The point response function of the microscope now contains seven terms
in the $l=0$, 2, and 4 bands. Each term is radially symmetric which means that
we expect a radially symmetric point spread function for any angular
distribution of fluorophores. Notice that the $h_l^0$ terms do not depend on the
polarizer orientation, while the $h_l^{-2}$ and $h_l^{2}$ do depend on the
polarizer orientation. This is because the fluorophore distributions
corresponding to the $h_l^0$ terms are rotationally symmetric about the optic
axis, while the other terms are not rotationally symmetric.

\subsection{Optical transfer function}
The optical transfer function for this microscope is given by
\begin{align}
  H_l^m(\nu; \mh{p}) = \frac{1}{\tilde{M}}\sum_{l'=0}^{\infty}\sum_{m'=-l'}^{l'} H_{l'}^{m'}(\nu; \mh{p})\delta(l - l', m - m'),  
\end{align}
where
$\tilde{M} = \left[\frac{\tilde{A}}{10} + \frac{1}{2}\right] + \left(\frac{\text{NA}}{n_o}\right)\left[-\frac{2\tilde{A}}{10} + \frac{1}{2}\right]$ is a
normalization constant, and
\begin{align}
  H_0^0(\nu) &= \left[\frac{\tilde{A}}{10} + \frac{1}{2}\right]A(\nu) + \left[-\frac{2\tilde{A}}{5} + 1\right]B(\nu),\\
  H_2^{-2}(\nu; \mh{p}) &= \left[\frac{9\sqrt{15}}{70}A(\nu) + \frac{3\sqrt{15}}{35}B(\nu)\right]\tilde{B}[(\mh{p}\cdot\mh{x})^2 - (\mh{p}\cdot\mh{y})^2],\\
  H_2^0(\nu) &= \left[-\frac{\sqrt{5}\tilde{A}}{14} + \frac{\sqrt{5}}{10}\right]A(\nu) + \left[-\frac{11\sqrt{5}\tilde{A}}{35} + \frac{2}{\sqrt{5}}\right]B(\nu),\\
  H_2^2(\nu; \mh{p}) &= -2\left[\frac{9\sqrt{15}}{70}A(\nu) + \frac{3\sqrt{15}}{35}B(\nu)\right]\tilde{B}(\mh{p}\cdot\mh{x})(\mh{p}\cdot\mh{y}),\\
  H_4^{-2}(\nu; \mh{p}) &= \left[-\frac{3\sqrt{5}}{70}A(\nu) + \frac{6\sqrt{5}}{35}B(\nu)\right]\tilde{B}[(\mh{p}\cdot\mh{x})^2 - (\mh{p}\cdot\mh{y})^2],\\
  H_4^0(\nu) &= \frac{3\tilde{A}}{35}[A(\nu) - 4B(\nu)],\\
  H_4^2(\nu; \mh{p}) &= -2\left[-\frac{3\sqrt{5}}{70}A(\nu) + \frac{6\sqrt{5}}{35}B(\nu)\right]\tilde{B}(\mh{p}\cdot\mh{x})(\mh{p}\cdot\mh{y}),
\end{align}
all other $H_l^m$ terms in the series are zero, and  
\begin{align}
  A(\nu) &= \frac{2}{\pi}\left[\arccos\left(\frac{\nu}{2\nu_o}\right) - \frac{\nu}{2\nu_o}\sqrt{1 - \left(\frac{\nu}{2\nu_o}\right)^2}\right]\Pi\left(\frac{\nu}{2\nu_o}\right),\\
  B(\nu) &= \frac{1}{\pi}\left(\frac{\text{NA}}{n_o}\right)^2\left[\arccos\left(\frac{\nu}{2\nu_o}\right) - \left[3 - 2\left(\frac{\nu}{2\nu_o}\right)^2\right]\frac{\nu}{2\nu_o} \sqrt{1 - \left(\frac{\nu}{2\nu_o}\right)^2}\right]\Pi\left(\frac{\nu}{2\nu_o}\right).                 
\end{align}

\subsection{Singular system}
% To find the singular system of the microscope we will carry out four steps.
% First, we will calculate the kernel of the microscope given by
% \begin{align}
%   k(\ro{} - \ro{}', \so{}, \so{}') = \sum_{i=0}^N\int_{\mathbb{R}^2}d\rd{}\, h(\rd{} - \ro{}, \so{})h(\rd{} - \ro{}', \so{}').
% \end{align}
% Second, we will use the kernel to set up the eigenvalue problem
% \begin{align}
%   [\mathcal{H}^{\dagger}\mathcal{H}]\mb{f}](\ro{}, \so{}) = \int_{\mbb{S}^2}d\so{}'\int_{\mbb{R}^2}d\ro{}'\, k(\ro{} - \ro{}', \so{}, \so{}')f(\ro{}', \so{}). \label{eq:eigeq}
% \end{align}
% Third, we will postulate the form of the eigenfunctions as a single spatial harmonic multiplied by a linear combination of spherical harmonics
% \begin{align}
%   u_{\bs{\rho},n}(\ro{}, \so{}) = \sum_{l=0}^\infty\sum_{m=-l}^l c_n^{lm}(\bs{\rho})y_l^m(\so{})\, \me{}^{-i 2\pi \ro{}\cdot\bs{\rho}}. \label{eq:eigfunc}
% \end{align}
% We will plug Eq. \ref{eq:eigfunc} into Eq. \ref{eq:eigeq} and solve for the
% singular values and object space singular vectors of the imaging system. Finally,
% we will find the data space singular vectors by calculating
% \begin{align}
%     v_{\bs{\rho},n,i}(\rd{}) = \int_{\mbb{S}^2}d\so{}\int_{\mbb{R}^3}d\ro{}\, h_i(\rd{}; \ro{}, \so{})u_{\bs{\rho},n}(\ro{}, \so{}),\qquad i=1, 2,\ldots,N, \label{eq:system}
% \end{align}

\bibliography{report}{}
\bibliographystyle{unsrt}

\end{document}
