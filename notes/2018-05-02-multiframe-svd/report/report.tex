
\documentclass[11pt]{article}

%%%%%%%%%%%%
% Packages %
%%%%%%%%%%%%
\usepackage[dvipsnames]{xcolor}
\hyphenpenalty=10000
\usepackage{tikz}
\usetikzlibrary{shapes,arrows}

\usepackage{tocloft}
\renewcommand\cftsecleader{\cftdotfill{\cftdotsep}}
\def\undertilde#1{\mathord{\vtop{\ialign{##\crcr
$\hfil\displaystyle{#1}\hfil$\crcr\noalign{\kern1.5pt\nointerlineskip}
$\hfil\tilde{}\hfil$\crcr\noalign{\kern1.5pt}}}}}
\usepackage{cleveref}
\usepackage{xcolor}
\usepackage[colorlinks = true,
            linkcolor = black,
            urlcolor  = blue,
            citecolor = black,
            anchorcolor = black]{hyperref}
\usepackage{epstopdf}
\usepackage{braket}
\usepackage{upgreek}
\usepackage{caption}
\usepackage{booktabs}
\usepackage{subcaption}
\usepackage{amssymb,latexsym,amsmath,gensymb}
\usepackage{latexsym}
\usepackage{graphicx}
\usepackage{float}
\usepackage{enumitem}
\usepackage{pdflscape}
\usepackage{url}
\usepackage{array}
\newcolumntype{C}{>{$\displaystyle} c <{$}}
\usepackage{tikz, calc}
\usetikzlibrary{shapes.geometric, arrows, calc}
\tikzstyle{norm} = [rectangle, rounded corners, minimum width=2cm, minimum height=1cm,text centered, draw=black]
\tikzstyle{arrow} = [thick, ->, >=stealth]

\newcommand{\argmin}{\arg\!\min}
\newcommand{\me}{\mathrm{e}}
\providecommand{\e}[1]{\ensuremath{\times 10^{#1}}} 
\providecommand{\mb}[1]{\mathbf{#1}}
\providecommand{\mf}[1]{\mathbf{#1}}
\providecommand{\ro}[1]{\mathbf{\mathbf{r}}_o}
\providecommand{\so}[1]{\mathbf{\hat{s}}_o}
\providecommand{\rb}[1]{\mathbf{r}_b}
\providecommand{\rbm}[1]{r_b^{\text{m}}}
\providecommand{\rd}[1]{\mathbf{r}_d}
\providecommand{\mh}[1]{\mathbf{\hat{#1}}}
\providecommand{\mbb}[1]{\mathbb{#1}}
\providecommand{\bs}[1]{\boldsymbol{#1}} 
\providecommand{\intinf}{\int_{-\infty}^{\infty}}
\providecommand{\fig}[4]{
  % filename, width, caption, label
\begin{figure}[h]
 \captionsetup{width=1.0\linewidth}
 \centering
 \includegraphics[width = #2\textwidth]{#1}
 \caption{#3}
 \label{fig:#4}
\end{figure}
}

\makeatletter
\renewcommand*\env@matrix[1][*\c@MaxMatrixCols c]{%
  \hskip -\arraycolsep
  \let\@ifnextchar\new@ifnextchar
  \array{#1}}
\makeatother

\newcommand{\tensor}[1]{\overset{\text{\tiny$\leftrightarrow$}}{\mb{#1}}}
\newcommand{\tunderbrace}[2]{\underbrace{#1}_{\textstyle#2}}
\providecommand{\figs}[7]{
  % filename1, filename2, caption1, caption2, label1, label2, shift
\begin{figure}[H]
\centering
\begin{minipage}[b]{.45\textwidth}
  \centering
  \includegraphics[width=1.0\linewidth]{#1}
  \captionsetup{justification=justified, singlelinecheck=true}
  \caption{#3}
  \label{fig:#5}
\end{minipage}
\hspace{2em}
\begin{minipage}[b]{.45\textwidth}
  \centering
  \includegraphics[width=1.0\linewidth]{#2}
  \vspace{#7em}
  \captionsetup{justification=justified}
  \caption{#4}
  \label{fig:#6}
\end{minipage}
\end{figure}
}
\makeatletter

\providecommand{\code}[1]{
\begin{center}
\lstinputlisting{#1}
\end{center}
}

\newcommand{\crefrangeconjunction}{--}
%%%%%%%%%%%
% Spacing %
%%%%%%%%%%%
% Margins
\usepackage[
top    = 1.5cm,
bottom = 1.5cm,
left   = 1.5cm,
right  = 1.5cm]{geometry}

% Indents, paragraph space
%\usepackage{parskip}
\setlength{\parskip}{1.5ex}

% Section spacing
\usepackage{titlesec}
\titlespacing*{\title}
{0pt}{0ex}{0ex}
\titlespacing*{\section}
{0pt}{0ex}{0ex}
\titlespacing*{\subsection}
{0pt}{0ex}{0ex}
\titlespacing*{\subsubsection}
{0pt}{0ex}{0ex}

% Line spacing
\linespread{1.1}

%%%%%%%%%%%%
% Document %
%%%%%%%%%%%%
\begin{document}
\title{\vspace{-2.5em} Singular value decomposition of multiframe\\ polarized
  fluorescence microscopes\vspace{-1em}} \author{Talon Chandler, Min Guo, Hari
  Shroff, Rudolf Oldenbourg, Patrick La Rivi\`ere}
\date{\vspace{-1em}\today\vspace{-1em}}
\maketitle
\section{Introduction}
In these notes we will develop the continuous models for several multiframe
polarized fluorescence microscopes. For each design we will calculate the
spatio-angular point spread function, optical transfer function, and singular
system consisting of the singular values, object-space singular functions, and
data-space singular functions.

All of the microscopes we will consider are imaging fields of oriented
fluorophores. To a good approximation any field of oriented fluorophores can be
represented by a member of the set
$\mbb{U} = \mbb{L}_2(\mbb{R}^3 \times \mbb{S}^2)$---square-integrable functions
that assign a scalar value to each position and orientation.

We will be considering multiframe microscopes that capture multiple images of
the same object (we assume that the object is static over the imaging time). The
data for the $n$th frame can be represented by a member of
$\mbb{V}_n = \mbb{L}_2(\mbb{R}^{2})$---square-integrable functions that assign a
scalar value to each point in a $2$-dimensional Euclidean space. If the
microscope collects $N$ frames, then all of the data can be represented by a
member of the larger set $\mbb{V} = \mbb{L}_2(\mbb{R}^{2N})$---square-integrable
functions that assign a scalar value to each point in a $2N$-dimensional
Euclidean space. We can say that the complete data space $\mbb{V}$ is built by
taking the \textit{orthogonal direct sum} of the data space for each frame
$\mbb{V}_n$
\begin{align}
  \mbb{V} = \bigoplus_{n=1}^N \mbb{V}_n.
\end{align}
Notice that we are assuming that data space is continuous---we are ignoring the
effects of finite pixels and a finite field-of-view.

We can model any linear relationship between object space and data space using
an integral transform
\begin{align}
  g_n(\rd{}) = [\mathcal{H}f]_n(\rd{}) = \int_{\mbb{S}^2}d\so{}\int_{\mbb{R}^3}d\ro{}\, h_n(\rd{}; \ro{}, \so{})f(\ro{}, \so{}),\qquad  n=1, 2,\ldots,N, \label{eq:full}
\end{align}
where $g_n(\rd{}) \in \mbb{V}_n$ is the data for the $n$th frame,
$f(\ro{}, \so{}) \in \mbb{U}$ is the object, and
$h_n(\rd{}; \ro{}, \so{}) \in \mbb{U} \times \mbb{V}_n$ is the point response
function of the imaging system for the $n$th frame.

In these notes we'll only be considering shift-invariant microscopes, so we can
simplify the model to
\begin{align}
  g_n(\rd{}) = [\mathcal{H}f]_n(\rd{}) = \int_{\mbb{S}^2}d\so{}\int_{\mbb{R}^3}d\ro{}\, h_n(\rd{} - \ro{}, \so{})f(\ro{}, \so{}),\qquad  n=1, 2,\ldots,N. \label{eq:full}
\end{align}
In these notes we will ignore magnification---in the previous notes we showed
that we can make a change of variables that puts a system with magnification in
the form of a system without magnification. In other words, we can drop the
primes that indicated magnified quantities in previous note sets. We will also
restrict ourselves to the paraxial approximation and drop the $(p)$
superscripts.

\section{Calculating the singular value decomposition}
In this section we will lay out the steps to calculate the singular value
decomposition of a multiframe polarized fluorescence microscope. We will follow
Section 2A of \cite{burvall06} closely and find the data-space singular
functions first. In Patrick's notes we found the object-space singular functions
first, but both approaches will give identical results and the eigenvalue
problem is smaller for the data-space singular functions.

The forward operator for the system is given by
\begin{align}
    [\mathcal{H}f]_n(\rd{}) = \int_{\mbb{S}^2}d\so{}\int_{\mbb{R}^3}d\ro{}\, h_n(\rd{} - \ro{}, \so{})f(\ro{}, \so{}),\qquad  n=1, 2,\ldots,N, \label{eq:fwd}
\end{align}
and the adjoint operator for the system is given by
\begin{align}
    [\mathcal{H}^{\dagger}\mb{g}](\ro{}, \so{}) = \sum_{j=1}^N\int_{\mbb{R}^2}d\mb{r}_{d}\, h_n(\rd{} - \ro{}, \so{})g(\rd{}).\label{eq:adj}
\end{align}
We will also need the data-space to data-space operator
\begin{align}
  [\mathcal{H}\mathcal{H}^{\dagger}\mb{g}]_n(\rd{}) = \sum_{n'=1}^N\int_{\mbb{R}^2}d\rd{}'\, k_{nn'}(\rd{} - \rd{}')g_{n'}(\rd{}'), \label{eq:backfwd}
\end{align}
where 
\begin{align}
  k_{nn'}(\rd{} - \rd{}') &= \int_{\mbb{S}^2}d\so{}\int_{\mbb{R}^2}d\ro{}\, h_n(\rd{} - \ro{}, \so{})h_{n'}(\rd{} - \ro{}, \so{}). \label{eq:kern}
\end{align}
To find the data-space singular functions and singular values, we need to
solve the following eigenequation
\begin{align}
  [\mathcal{H}\mathcal{H}^{\dagger}\mb{v}_{\bs{\rho}, j}]_n(\rd{}) = \mu_{\bs{\rho}, n} \mb{v}_{\bs{\rho}, j}(\rd{}). \label{eq:eigval}
\end{align}
where $\mb{v}_{\bs{\rho}, j}(\rd{})$ are the data-space eigenfunctions and
$\mu_{\bs{\rho}, j}$ are the eigenvalues. Each $\mb{v}_{\bs{\rho}, j}(\rd{})$ is
an $N\times 1$ vector where each element is a function of $\rd{}$. The eigenfunctions $\mb{v}_{\bs{\rho}, j}(\rd{})$ are indexed by a
continuous two-dimensional vector index $\bs{\rho}$ associated with the
transverse directions and a discrete index $j$ associated with the angular
directions. Since the imaging system is linear shift invariant, the lateral part
of the data-space eigenfunctions will be complex exponentials. We still need to
solve for the angular part, so the data-space eigenfunctions will be in the form
\begin{align}
  \mb{v}_{\bs{\rho},j}(\rd{}) = \mb{V}_j(\bs{\rho})\me{}^{i 2\pi \bs{\rho}\cdot\rd{}}. \label{eq:eigfunc}
\end{align}
Inserting Eqs. \ref{eq:eigfunc} and \ref{eq:kern} into \ref{eq:backfwd} yields
\begin{align}
  [\mathcal{H}\mathcal{H}^{\dagger}\mb{v}_{\bs{\rho},j}]_n(\rd{}) = \me{}^{i 2\pi \ro{}\cdot\bs{\rho}} \sum_{n'=1}^N K_{nn'}(\bs{\rho})[\mb{V}_j(\bs{\rho})]_{n'}, \label{eq:exp}
\end{align}
where
\begin{align}
  K_{nn'}(\bs{\rho}) = \int_{\mbb{R}^2}d\rd{}'\, k_{nn'}(\rd{}')\me{}^{-i 2\pi \bs{\rho}\cdot\rd{}'}. \label{eq:kft}
\end{align}
Comparing Eqs. \ref{eq:eigfunc} and \ref{eq:exp} yield the eigenvalue problem
\begin{align}
  \mathsf{K}(\bs{\rho})\mb{V}_j(\bs{\rho}) = \mu_{\bs{\rho}, j}\mb{V}_j(\bs{\rho}), \label{eq:evp}
\end{align}
where $\mathsf{K}(\bs{\rho})$ is an $N\times N$ matrix containing the elements
$K_{nn'}(\bs{\rho})$. To simplify the calculation of the elements
$K_{nn'}(\bs{\rho})$ we notice that we need to take the Fourier transform (Eq.
\ref{eq:kft}) of the autocorrelation of the spatio-angular point spread function
(Eq. \ref{eq:kern}). We can use the autocorrelation theorem to relate the
elements of $\mathsf{K}(\bs{\rho})$ to the spatio-angular optical transfer
function as
\begin{align}
  K_{nn'}(\bs{\rho}) = \sum_{l=0}^{\infty}\sum_{m=-l}^l H_{l,n}^m(\bs{\rho}) H_{l,n'}^m(\bs{\rho}). \label{eq:ksimp}
\end{align}
Once we have solved the $N\times N$ eigenvalue problem in Eq. \ref{eq:evp} for
each value of $\bs{\rho}$, we can build the complete data-space eigenfunctions using
Eq. \ref{eq:eigfunc}. Finally, we can calculate the object-space eigenfunctions from
the data-space eigenfunctions using
\begin{align}
  [\mathcal{H}^{\dagger}\mb{v}_{\bs{\rho},j}](\ro{}, \so{}) = \sqrt{\mu_{\bs{\rho}, j}}u_{\bs{\rho},j}(\ro{}, \so{}).
\end{align}
Plugging in Eq. \ref{eq:adj} and solving for $u_{\bs{\rho},j}(\ro{}, \so{})$ yields
\begin{align}
  u_{\bs{\rho},j}(\ro{}, \so{}) = \frac{1}{\sqrt{\mu_{\bs{\rho},j}}}\me{}^{i2\pi\bs{\rho}\cdot\ro{}}\sum_{n=1}^N[\mb{V}_j(\bs{\rho})]_n \sum_{l=0}^{\infty}\sum_{m=-l}^l H_{l,n}^m(\bs{\rho})y_l^m(\so{}).
\end{align}
As we'd expect from Patrick's notes, we can rewrite the object-space singular
functions as a complex exponential multiplied by a linear combination of
spherical harmonics
\begin{align}
u_{\bs{\rho},j}(\ro{}, \so{}) = U_j(\bs{\rho}, \so{}) \me{}^{i2\pi\bs{\rho}\cdot\ro{}},
\end{align}
where
\begin{align}
  U_{\bs{\rho},j}(\ro{}, \so{}) = \frac{1}{\sqrt{\mu_{\bs{\rho},j}}}\sum_{n=1}^N[\mb{V}_j(\bs{\rho})]_n \sum_{l=0}^{\infty}\sum_{m=-l}^l H_{l,n}^m(\bs{\rho})y_l^m(\so{}).
\end{align}
In summary, these are the steps to calculate the singular system for a
polarized multiframe microscope.
\begin{enumerate}
\item Calculate the spatio-angular point spread function
  $h_n(\rd{} - \ro{}, \so{})$ for $n = 1, 2, \ldots, N$.
\item Calculate the spatio-angular optical transfer function
  $H_{l,n}^{m}(\bs{\nu})$ for $n = 1, 2, \ldots, N$.
\item Calculate the elements of the matrix $\mathsf{K}(\bs{\rho})$ using
  $K_{nn'}(\bs{\rho}) = \sum_{l=0}^{\infty}\sum_{m=-l}^l H_{l,n}^m(\bs{\rho}) H_{l,n'}^m(\bs{\rho})$.
\item Solve the eigenvalue problem
  $\mathsf{K}(\bs{\rho})\mb{V}_j(\bs{\rho}) = \mu_{\bs{\rho}, j}\mb{V}_j(\bs{\rho})$ at each value of $\bs{\rho}$. Store the eigenvalues $\mu_{\bs{\rho}, j}$ and eigenvectors $\mb{V}_j(\bs{\rho})$. 
\item Calculate the data-space singular functions using $\mb{v}_{\bs{\rho},j}(\rd{}) = \mb{V}_j(\bs{\rho})\me{}^{i 2\pi \bs{\rho}\cdot\rd{}}$.
\item Calculate the object-space singular functions using
  \begin{align}
   u_{\bs{\rho},j}(\ro{}, \so{}) = \frac{1}{\sqrt{\mu_{\bs{\rho},j}}}\me{}^{i2\pi\bs{\rho}\cdot\ro{}}\sum_{n=1}^N[\mb{V}_j(\bs{\rho})]_n \sum_{l=0}^{\infty}\sum_{m=-l}^l H_{l,n}^m(\bs{\rho})y_l^m(\so{}). 
  \end{align}
\end{enumerate}

Under the paraxial approximation we will be able to perform steps 1 and 2
analytically. We could perform step 3 analytically, but I don't think this step
will yield much insight so I will do it numerically. Step 4 can be performed
analytically for $N \leq 4$ ($N > 4$ yields a quintic equation which has no
solution in terms of radicals), but I will solve it numerically in all of the
cases we'll consider. Steps 5 and 6 can be carried out numerically.

Outside of the paraxial approximation we will only be able to write the
spatio-angular point spread function in terms of an integral, so we will
perform all steps numerically. In these notes we'll only consider the paraxial
case, though.

\section{Polarized epi-illumination with unpolarized epi-detection}
We will start by restricting our analysis of epi-illumination microscopes to
in-focus objects. This means that our object space is
$\mbb{L}_2(\mbb{R}^2 \times \mbb{S}^2)$.

\subsection{Point response function}
In the previous notes we showed that the excitation point response function for
polarized epi-illumination is given by
\begin{align}
  h^{\mh{z}}_{\text{exc}}(\so{}; \mh{p}) &= y_0^0(\so{}) - \frac{1}{\sqrt{5}}\tilde{A}y_2^0(\so{}) + \sqrt{\frac{3}{5}}\tilde{B}\left\{[(\mh{p}\cdot\mh{x})^2 - (\mh{p}\cdot\mh{y})^2]y_2^2(\so{}) - 2(\mh{p}\cdot\mh{x})(\mh{p}\cdot\mh{y})y_2^{-2}(\so{})\right\}, \label{eq:genpsf}
\end{align}
where
\begin{subequations}
\begin{align}
  \tilde{A} &\equiv \cos^2(\alpha/2)\cos(\alpha),\\
  \tilde{B} &\equiv \frac{1}{12}(\cos^2\alpha + 4\cos\alpha + 7),
\end{align}\label{eq:coefficients}%
\end{subequations}
and $\alpha \equiv \arcsin(\text{NA}/n_o)$.

We also showed that the point response function for unpolarized epi-detection is
given by
\begin{align}
  h_{\text{det}}(\ro{}, \so{}) &= [{a}^2(r_o) + 2b^2(r_o)]y_0^0(\so{}) + \frac{1}{\sqrt{5}}\left[- a^2(r_o) + 4b^2(r_o)\right]y_2^0(\so{}),
\end{align}
where
\begin{align}
  a(r_o) = \frac{J_1(2\pi \nu_or_o)}{\pi \nu_or_o}, 
  &\hspace{2em}
    b(r_o) = \frac{\text{NA}}{n_o}\left[\frac{J_2(2\pi \nu_or_o)}{\pi \nu_or_o}\right],  \label{eq:abparadef}
  \intertext{and}
  \nu_o \equiv \frac{\text{NA}}{\lambda},&\hspace{2em}
  \text{NA} = n_o\sin\alpha.
\end{align}
The excitation and detection processes are incoherent, so to find the complete
point response function we can multiply the excitation and detection point response
functions which gives
\begin{align}
  h(\ro{}, \so{}; \mh{p}) = h^{\mh{z}}_{\text{exc}}(\so{}; \mh{p})h_{\text{det}}(\ro{}, \so{}) = \frac{1}{\tilde{N}}\sum_{l=0}^{\infty}\sum_{m=-l}^l h_l^m(\ro{}; \mh{p})y_l^m(\so{}), 
\end{align}
where $\tilde{N} = \frac{\tilde{A}}{10} + \frac{1}{2}$ is a normalization constant
and the terms in the series are given by
\begin{align}
  h_0^0(r_o) &= \left[\frac{\tilde{A}}{10} + \frac{1}{2}\right]a^2(r_o) + \left[-\frac{2\tilde{A}}{5} + 1\right]b^2(r_o),\\
  h_2^{-2}(r_o; \mh{p}) &= \left[\frac{9\sqrt{15}}{70}a^2(r_o) + \frac{3\sqrt{15}}{35}b^2(r_o)\right]\tilde{B}[(\mh{p}\cdot\mh{x})^2 - (\mh{p}\cdot\mh{y})^2],\\
  h_2^0(r_o) &= \left[-\frac{\sqrt{5}\tilde{A}}{14} + \frac{\sqrt{5}}{10}\right]a^2(r_o) + \left[-\frac{11\sqrt{5}\tilde{A}}{35} + \frac{2}{\sqrt{5}}\right]b^2(r_o),\\
  h_2^2(r_o; \mh{p}) &= -2\left[\frac{9\sqrt{15}}{70}a^2(r_o) + \frac{3\sqrt{15}}{35}b^2(r_o)\right]\tilde{B}(\mh{p}\cdot\mh{x})(\mh{p}\cdot\mh{y}),\\
  h_4^{-2}(r_o; \mh{p}) &= \left[-\frac{3\sqrt{5}}{70}a^2(r_o) + \frac{6\sqrt{5}}{35}b^2(r_o)\right]\tilde{B}[(\mh{p}\cdot\mh{x})^2 - (\mh{p}\cdot\mh{y})^2],\\
  h_4^0(r_o) &= \frac{3\tilde{A}}{35}[a^2(r_o) - 4b^2(r_o)],\\
  h_4^2(r_o; \mh{p}) &= -2\left[-\frac{3\sqrt{5}}{70}a^2(r_o) + \frac{6\sqrt{5}}{35}b^2(r_o)\right]\tilde{B}(\mh{p}\cdot\mh{x})(\mh{p}\cdot\mh{y}),
\end{align}
and all other $h_l^m$ terms in the series are zero. Finally, the complete
forward model for this class of microscope is given by
\begin{align}
    g_n(\rd{}) = \int_{\mbb{S}^2}d\so{}\int_{\mbb{R}^3}d\ro{}\, h(\rd{} - \ro{}, \so{}; \mh{p}_n)f(\ro{}, \so{}),\qquad  n=1, 2,\ldots,N. \label{eq:fwdpolillum}
\end{align}
Notice that we have used a single function $h(\rd{} - \ro{}, \so{}; \mh{p}_n)$
to describe the complete imaging system, and we have shifted the index $n$ to
the illumination polarization. A typical choice of frames is given by $N=4$ and
\begin{align}
  \mh{p}_1 = \mh{x},\hspace{2em}
  \mh{p}_2 = \frac{1}{\sqrt{2}}(\mh{x} + \mh{y}),\hspace{2em}
  \mh{p}_3 = \mh{y},\hspace{2em}
  \mh{p}_4 = \frac{1}{\sqrt{2}}(\mh{x} - \mh{y}). 
\end{align}
\indent The point response function of the microscope now contains seven terms
in the $l=0$, 2, and 4 bands. Each term is radially symmetric which means that
we expect a radially symmetric point spread function for any angular
distribution of fluorophores. Notice that the $h_l^0$ terms do not depend on the
polarizer orientation, while the $h_l^{-2}$ and $h_l^{2}$ do depend on the
polarizer orientation. This is because the fluorophore distributions
corresponding to the $h_l^0$ terms are rotationally symmetric about the optic
axis, while the other terms are not rotationally symmetric.

\subsection{Optical transfer function}
The optical transfer function for this microscope is given by
\begin{align}
  H_l^m(\nu; \mh{p}) = \frac{1}{\tilde{M}}\sum_{l'=0}^{\infty}\sum_{m'=-l'}^{l'} H_{l'}^{m'}(\nu; \mh{p})\delta(l - l', m - m'),  
\end{align}
where
$\tilde{M} = \left[\frac{\tilde{A}}{10} + \frac{1}{2}\right] + \left(\frac{\text{NA}}{n_o}\right)\left[-\frac{2\tilde{A}}{10} + \frac{1}{2}\right]$ is a
normalization constant, and
\begin{align}
  H_0^0(\nu) &= \left[\frac{\tilde{A}}{10} + \frac{1}{2}\right]A(\nu) + \left[-\frac{2\tilde{A}}{5} + 1\right]B(\nu),\\
  H_2^{-2}(\nu; \mh{p}) &= \left[\frac{9\sqrt{15}}{70}A(\nu) + \frac{3\sqrt{15}}{35}B(\nu)\right]\tilde{B}[(\mh{p}\cdot\mh{x})^2 - (\mh{p}\cdot\mh{y})^2],\\
  H_2^0(\nu) &= \left[-\frac{\sqrt{5}\tilde{A}}{14} + \frac{\sqrt{5}}{10}\right]A(\nu) + \left[-\frac{11\sqrt{5}\tilde{A}}{35} + \frac{2}{\sqrt{5}}\right]B(\nu),\\
  H_2^2(\nu; \mh{p}) &= -2\left[\frac{9\sqrt{15}}{70}A(\nu) + \frac{3\sqrt{15}}{35}B(\nu)\right]\tilde{B}(\mh{p}\cdot\mh{x})(\mh{p}\cdot\mh{y}),\\
  H_4^{-2}(\nu; \mh{p}) &= \left[-\frac{3\sqrt{5}}{70}A(\nu) + \frac{6\sqrt{5}}{35}B(\nu)\right]\tilde{B}[(\mh{p}\cdot\mh{x})^2 - (\mh{p}\cdot\mh{y})^2],\\
  H_4^0(\nu) &= \frac{3\tilde{A}}{35}[A(\nu) - 4B(\nu)],\\
  H_4^2(\nu; \mh{p}) &= -2\left[-\frac{3\sqrt{5}}{70}A(\nu) + \frac{6\sqrt{5}}{35}B(\nu)\right]\tilde{B}(\mh{p}\cdot\mh{x})(\mh{p}\cdot\mh{y}),
\end{align}
all other $H_l^m$ terms in the series are zero, and  
\begin{align}
  A(\nu) &= \frac{2}{\pi}\left[\arccos\left(\frac{\nu}{2\nu_o}\right) - \frac{\nu}{2\nu_o}\sqrt{1 - \left(\frac{\nu}{2\nu_o}\right)^2}\right]\Pi\left(\frac{\nu}{2\nu_o}\right),\\
  B(\nu) &= \frac{1}{\pi}\left(\frac{\text{NA}}{n_o}\right)^2\left[\arccos\left(\frac{\nu}{2\nu_o}\right) - \left[3 - 2\left(\frac{\nu}{2\nu_o}\right)^2\right]\frac{\nu}{2\nu_o} \sqrt{1 - \left(\frac{\nu}{2\nu_o}\right)^2}\right]\Pi\left(\frac{\nu}{2\nu_o}\right).                 
\end{align}


\bibliography{report}{}
\bibliographystyle{unsrt}

\end{document}
