\documentclass[11pt]{article}

%%%%%%%%%%%%
% Packages %
%%%%%%%%%%%%
\usepackage[dvipsnames]{xcolor}
\hyphenpenalty=10000
\usepackage{tikz}
\usetikzlibrary{shapes,arrows}

\usepackage{tocloft}
\renewcommand\cftsecleader{\cftdotfill{\cftdotsep}}
\def\undertilde#1{\mathord{\vtop{\ialign{##\crcr
$\hfil\displaystyle{#1}\hfil$\crcr\noalign{\kern1.5pt\nointerlineskip}
$\hfil\tilde{}\hfil$\crcr\noalign{\kern1.5pt}}}}}
\usepackage{cleveref}
\usepackage{xcolor}
\usepackage[colorlinks = true,
            linkcolor = black,
            urlcolor  = blue,
            citecolor = black,
            anchorcolor = black]{hyperref}
\usepackage{epstopdf}
\usepackage{braket}
\usepackage{upgreek}
\usepackage{caption}
\usepackage{booktabs}
\usepackage{subcaption}
\usepackage{amssymb,latexsym,amsmath,gensymb}
\usepackage{latexsym}
\usepackage{graphicx}
\usepackage{float}
\usepackage{enumitem}
\usepackage{pdflscape}
\usepackage{url}
\usepackage{array}
\newcolumntype{C}{>{$\displaystyle} c <{$}}
\usepackage{tikz, calc}
\usetikzlibrary{shapes.geometric, arrows, calc}
\tikzstyle{norm} = [rectangle, rounded corners, minimum width=2cm, minimum height=1cm,text centered, draw=black]
\tikzstyle{arrow} = [thick, ->, >=stealth]

\newcommand{\argmin}{\arg\!\min}
\newcommand{\me}{\mathrm{e}}
% \providecommand{\e}{\ensuremath{\times 10^{#1}}} 
\providecommand{\mb}[1]{\mathbf{#1}}
\providecommand{\mc}[1]{\mathcal{#1}}
\providecommand{\ro}{\mathbf{\mathfrak{r}}_o}
\providecommand{\so}{\mathbf{\hat{s}}_o}
\providecommand{\rb}{\mathbf{r}_b}
\providecommand{\rbm}{r_b^{\text{m}}}
\providecommand{\rd}{\mathbf{\mathfrak{r}}_d}
\providecommand{\mh}[1]{\mathbf{\hat{#1}}}
\providecommand{\mf}[1]{\mathfrak{#1}}
\providecommand{\ms}[1]{\mathsf{#1}}
\providecommand{\mbb}[1]{\mathbb{#1}}
\providecommand{\bs}[1]{\boldsymbol{#1}}
\providecommand{\tv}{\text{v}}
\providecommand{\tx}[1]{\text{#1}}
\providecommand{\bv}{\bs{\nu}}
\providecommand{\bp}{\bs{\rho}}
\providecommand{\p}{\mh{p}}
\providecommand{\lmsum}{\sum_{l=0}^\infty\sum_{m=-l}^{l}}
\providecommand{\intr}[1]{\int_{\mbb{R}^{#1}}}
\providecommand{\ints}[1]{\int_{\mbb{S}^{#1}}}
\providecommand{\intinf}{\int_{-\infty}^{\infty}}
\providecommand{\fig}[4]{
  % filename, width, caption, label
\begin{figure}[h]
 \captionsetup{width=1.0\linewidth}
 \centering
 \includegraphics[width = #2\textwidth]{#1}
 \caption{#3}
 \label{fig:#4}
\end{figure}
}

\makeatletter
\renewcommand*\env@matrix[1][*\c@MaxMatrixCols c]{%
  \hskip -\arraycolsep
  \let\@ifnextchar\new@ifnextchar
  \array{#1}}
\makeatother

\newcommand{\tensor}[1]{\overset{\text{\tiny$\leftrightarrow$}}{\mb{#1}}}
\newcommand{\tunderbrace}[2]{\underbrace{#1}_{\textstyle#2}}
\providecommand{\figs}[7]{
  % filename1, filename2, caption1, caption2, label1, label2, shift
\begin{figure}[H]
\centering
\begin{minipage}[b]{.45\textwidth}
  \centering
  \includegraphics[width=1.0\linewidth]{#1}
  \captionsetup{justification=justified, singlelinecheck=true}
  \caption{#3}
  \label{fig:#5}
\end{minipage}
\hspace{2em}
\begin{minipage}[b]{.45\textwidth}
  \centering
  \includegraphics[width=1.0\linewidth]{#2}
  \vspace{#7em}
  \captionsetup{justification=justified}
  \caption{#4}
  \label{fig:#6}
\end{minipage}
\end{figure}
}
\makeatletter

\providecommand{\code}[1]{
\begin{center}
\lstinputlisting{#1}
\end{center}
}

\newcommand{\crefrangeconjunction}{--}
%%%%%%%%%%%
% Spacing %
%%%%%%%%%%%
% Margins
\usepackage[
top    = 1.5cm,
bottom = 1.5cm,
left   = 1.5cm,
right  = 1.5cm]{geometry}

% Indents, paragraph space
%\usepackage{parskip}
\setlength{\parskip}{1.5ex}

% Section spacing
\usepackage{titlesec}
\titlespacing*{\title}
{0pt}{0ex}{0ex}
\titlespacing*{\section}
{0pt}{0ex}{0ex}
\titlespacing*{\subsection}
{0pt}{0ex}{0ex}
\titlespacing*{\subsubsection}
{0pt}{0ex}{0ex}

% Line spacing
\linespread{1.1}

%%%%%%%%%%%%
% Document %
%%%%%%%%%%%%
\begin{document}
\title{\vspace{-2.5em} Singular value decomposition of a dual orthogonal-view\\
  polarized fluorescence microscope\vspace{-1em}} \author{Talon Chandler, Min
  Guo, Hari Shroff, Rudolf Oldenbourg, Patrick La Rivi\`ere}
\date{\vspace{-1em}\today\vspace{-1em}}
\maketitle
\section{Introduction}
In these notes we will find the kernel, transfer function, and singular value
decomposition of a dual orthogonal view polarized fluorescence microscope
(diSPIM). We will model the relationship between the spatio-angular density of
fluorophores in a three-dimensional sample---a member of
$\mbb{L}_2(\mbb{R}^3 \times \mbb{S}^2)$---and the two three-dimensional volumes
for each illumination polarization setting---two members of
$\mbb{L}_2(\mbb{R}^3 \times \mbb{S}^1)$. We could consider the diSPIM as a
microscope that makes two samples of a larger space
$\mbb{L}_2(\mbb{R}^3 \times \mbb{S}^1 \times \mbb{S}^2)$, but I don't think this
will give us any extra insight at this point.

We can write the forward model for the polarized diSPIM as
\begin{align}
  g_{\tv}(\rd, \p) = \left[\mc{H}f\right]_\tv(\rd, \p) = \intr{3}d\ro \ints{2}d\so\, h_{\tv}(\rd - \ro, \p, \so)f(\ro, \so), \label{eq:fwd}
\end{align}
where $\tv \in \{\text{A}, \text{B}\}$ is the light path index, $\rd$ is the
three-dimensional detection coordinate, and all other coordinates have been
introduced in previous note sets. The adjoint operator is given by
\begin{align}
  f(\ro, \so) = [\mc{H}^\dagger \mb{g}](\ro, \so) = \sum_{\tv \in \{\text{A}, \text{B}\}} \intr{3}d\rd\ints{1}d\p\, h_{\tv}(\rd - \ro, \p, \so)g_{\tv}(\rd, \p). \label{eq:adj}
\end{align}
\section{Kernel}
The polarized diSPIM illuminates the sample with a uniform-width Gaussian beam
(we ignore light-sheet broadening) with variable polarization. In path A the
light sheet is in the $xy$ plane for viewing by an objective with an optical
axis along the $z$ axis. In path B the roles are reversed and the light sheet is
in the $yz$ plane for viewing by an objective with an optical axis along the $x$
axis.

In the
\href{https://github.com/talonchandler/polharmonic/blob/master/notes/2018-05-14-single-view-continuous-svd/report/report.pdf}{May
  14 notes} we derived the excitation kernel for a polarized collimated beam (NA
= 0) traveling along the $z$ axis. The excitation kernel for path B of the
diSPIM is identical to the previously derived excitation kernel weighted by
Gaussian profile of the light sheet
\begin{align}
  h_{\tx{B}, \tx{exc}}(\ro, \so; \mh{p}) = \left[z_{0}^{\tx{B}}(\mh{p})y_0^0(\so) + \sqrt{\frac{3}{5}}z_{-2}^{\tx{B}}(\mh{p})y_2^{-2}(\so) - \frac{1}{\sqrt{5}}z_{0}^{\tx{B}}(\mh{p})y_2^0(\so) + \sqrt{\frac{3}{5}}z_{2}^{\tx{B}}(\mh{p})y_2^{2}(\so)\right]\mf{g}(r_x, \sigma_{\tx{ls}}),\label{eq:pathbexc}
\end{align}
where $\mf{g}(x, \sigma) \equiv \tx{exp}(x^2/2\sigma^2)/\sqrt{2\pi\sigma^2}$,
$\ro = r_x\mh{x} + r_y\mh{y} + r_z\mh{z}$ is the three-dimensional position
vector in the object, and $\sigma_{\tx{ls}}$ is the spatial standard deviation
(width) of the light sheet.

We can find the excitation kernel for path A by swapping the $x$ and $z$
coordinates in Eq. \ref{eq:pathbexc}. This is a simple modification for the
spatial part of the excitation kernel (change $\mf{g}(r_x, \sigma_{\tx{ls}})$
to $\mf{g}(r_z, \sigma_{\tx{ls}})$), and the coefficients of the spherical
harmonics are transformed by a matrix multiplication
\begin{align}
  \begin{bmatrix}
    {c'}_0^{0}\\ {c'}_2^{-2}\\ {c'}_2^{-1}\\ {c'}_2^0\\ {c'}_2^1\\ {c'}_2^2\\
  \end{bmatrix} =
  \begin{bmatrix}
      1&0&0&0&0&0\\
      0&0&-1&0&0&0\\
      0&-1&0&0&0&0\\
      0&0&0&-1/2&0&\sqrt{3}/2\\
      0&0&0&0&1&0\\
      0&0&0&\sqrt{3}/{2}&0&1/2\\
  \end{bmatrix}
  \begin{bmatrix}
    {c}_0^{0}\\ {c}_2^{-2}\\ {c}_2^{-1}\\ {c}_2^0\\ {c}_2^1\\ {c}_2^2
  \end{bmatrix}, \label{eq:transform}
\end{align}
where $c_l^m$ is coefficient of the $y_l^m$ spherical harmonic in the original
frame and ${c'}_l^m$ is the coefficient of the same spherical harmonic in the
rotated frame. Applying this transformation gives
\begin{align}
  h_{\tx{A}, \tx{exc}}(\ro, \so; \mh{p}) = \Bigg[z_{0}^{\tx{A}}(\mh{p})y_0^0(\so) + \sqrt{\frac{3}{5}}z^{\tx{A}}_{-2}(\mh{p})y_2^{-1}(\so) &+ \frac{1}{2\sqrt{5}}\left\{3z_2^{\tx{A}}(\mh{p}) - z_{0}^{\tx{A}}(\mh{p}) \right\} y_2^0(\so)\nonumber \\ &+ \frac{1}{2}\sqrt{\frac{3}{5}}\left\{z^{\tx{A}}_{2}(\mh{p}) - z_{0}^{\tx{A}}(\mh{p})\right\}y_2^{2}(\so)\Bigg]\mf{g}(r_z, \sigma_{\tx{ls}}).\label{eq:pathaexc}
\end{align}
Notice that we have used a different set of circular harmonics for each view.
The $z_n^{\tx{B}}(\mh{p})$ circular harmonics are defined by the angle to the $x$-axis
from the $+y$-axis direction, and the $z_n^{\tx{A}}(\mh{p})$ circular harmonics are
defined by the angle to the $z$-axis from the $+y$-axis direction. This change
of definition follows directly from swapping the $x$ and $z$ coordinates to go
from path B to path A.

In the
\href{https://github.com/talonchandler/polharmonic/blob/master/notes/2018-05-14-single-view-continuous-svd/report/report.pdf}{May
  14 notes} we derived the two-dimensional detection kernel. The kernel is
two-dimensional because it only considers thin in-focus objects. For a complete
non-paraxial three-dimensional model I would eventually like to follow Backer
and Moerner \cite{backer2014}, but there is still a few weeks of work required
to find the associated transfer functions. Instead I'll follow the early diSPIM
deconvolution models \cite{wu2013} and use an axial Gaussian function to model
the detection kernel. 

The detection kernel for an objective along the $z$ axis (path A) as
\begin{align}
  h_{\tx{A}, \tx{det}}(\ro, \so) &= \left[[{a}^2(r_{xy}) + 2b^2(r_{xy})]y_0^0(\so) + \frac{1}{\sqrt{5}}\left[- a^2(r_{xy}) + 4b^2(r_{xy})\right]y_2^0(\so)\right]\mathfrak{g}(r_z, \sigma_{\tx{det}}), \label{eq:det1}
\end{align}
where
\begin{align}
  a(r) = \frac{J_1(2\pi \nu_or)}{\pi \nu_or}, 
  &\hspace{2em}
    b(r) = \frac{\tx{NA}}{n_o}\left[\frac{J_2(2\pi \nu_or)}{\pi \nu_or}\right],  \label{eq:abparadef}
    \intertext{and}
  r_{xy} \equiv \sqrt{r_x^2 + r_y^2},\hspace{2em}
  \nu_o \equiv &\frac{\tx{NA}}{\lambda},\hspace{2em}
  \tx{NA} = n_o\sin\alpha.
\end{align}

To find the two-dimensional detection kernel for path B we transform the
spherical harmonic coefficients using Eq. \ref{eq:transform}
\begin{align}
  h_{\tx{B}, \tx{det}}(\ro, \so) = \Bigg[[{a}^2(r_{yz}) + 2b^2(r_{yz})]y_0^0(\so) &+ \frac{1}{2\sqrt{5}}\left[- a^2(r_{yz}) + 4b^2(r_{yz})\right]y_2^0(\so) \nonumber \\ &+ \frac{1}{2}\sqrt{\frac{3}{5}}\left[- a^2(r_{yz}) + 4b^2(r_{yz})\right]y_2^2(\so)\Bigg]\mathfrak{g}(r_x, \sigma_{\tx{det}}), \label{eq:det2}
\end{align}
where $r_{yz} \equiv \sqrt{r_y^2 + r_z^2}$. 

The complete kernel is the product of the excitation and detection kernels
\begin{align}
  h_{\tv}(\ro, \so; \mh{p}) &= h_{\tv, \tx{exc}}(\ro, \so; \mh{p})h_{\tv, \tx{det}}(\ro, \so).
\end{align}

Notice that the both the excitation and detection kernels are Gaussian along the
axial detection axis, so the complete kernel is also Gaussian along that
dimension with standard deviation
$\sigma_{\tx{ax}} = \sqrt{\sigma_{\tx{ls}}^2 + \sigma_{\tx{det}}^2}$. Therefore,
we can rewrite the kernel in the form
\begin{align}
  h_{\tx{A}}(\ro, \so; \mh{p}) &= \left[\sum_{l=0}^{\infty}\sum_{m=-l}^l\sum_{n=-\infty}^{\infty} h_{l,n,\tx{A}}^{m}(r_{xy})y_l^m(\so)z_n(\mh{p})\right]\mathfrak{g}(r_z, \sigma_{\tx{ax}}),\\
  h_{\tx{B}}(\ro, \so; \mh{p}) &= \left[\sum_{l=0}^{\infty}\sum_{m=-l}^l\sum_{n=-\infty}^{\infty} h_{l,n, \tx{B}}^{m}(r_{yz})y_l^m(\so)z_n(\mh{p})\right]\mathfrak{g}(r_x, \sigma_{\tx{ax}}).
\end{align}
% Figures \ref{fig:kernA} and \ref{fig:kernB} show the kernels for both viewing
% directions. 

% \fig{../calculations/out/hhillx.pdf}{1.0}{Kernel $h_{l,n,\tx{A}}^m(\mb{r}_{xy})$
%   for view A of a polarized diSPIM microscope with a 0.8 NA detection objective.
%   \textbf{Rows:} $l$ indexes the object-space spherical harmonic band.
%   \textbf{Columns:} $m$ indexes the object-space spherical harmonic degree, and
%   $n$ indexes the data-space circular harmonic band. \textbf{Entries:} Each
%   column and row contains a continuous two-dimensional plot indexed by the
%   vector $\mb{r}_{xy}$ ranging from $-\lambda/2\text{NA}$ to
%   $+\lambda/2\text{NA}$ (although $\mb{r}_{xy}$ is rotationally symmetric so one
%   dimension would suffice). All values are normalized between $-1$
%   \textcolor{blue}{(blue)} and $+1$ \textcolor{red}{(red)} with $0$ colored
%   white.}{kernA}
%   % filename, width, caption, label

% \fig{../calculations/out/hhillz.pdf}{1.0}{Kernel $h_{l,n,\tx{B}}^m(\mb{r}_{yz})$
%   for view B of a polarized diSPIM microscope with a 0.8 NA detection objective.
%   See caption of Fig. \ref{fig:kernA} for details.}{kernB}
%   % filename, width, caption, label

\section{Transfer function}
We can rewrite Eqs. \ref{eq:fwd} and \ref{eq:adj} in the frequency domain as
\begin{align}
  G_{\tv, n}(\bv) &= \lmsum H_{l,n,\tv}^{m}(\bv)F_l^m(\bv), \label{eq:ffwd}\\
  F_l^m(\bv) &= \sum_{\tv \in \{\tx{A}, \tx{B}\}}\sum_{n=-\infty}^\infty H_{l,n,\tv}^{m}(\bv)G_{\tv, n}(\bv) \label{eq:fadj},
\end{align}
where
\begin{align}
  G_{\tv, n}(\bv) &= \intr{3}d\rd\, \me^{i2\pi\rd\cdot\bv}\ints{1}d\p\, z_n(\p) g_{\tv}(\rd, \p),\\
  H_{l,n,\tv}^m(\bv) &= \intr{3}d\ro\, \me^{i2\pi\ro\cdot\bv}\ints{1}d\p\, z_n(\p)\ints{2}d\so\, y_l^m(\so) h_{\tv}(\ro, \p, \so),\\
  F_l^m(\bv) &= \intr{3}d\ro\, \me^{i2\pi\ro\cdot\bv}\ints{2}d\so\, y_l^m(\so) f(\ro, \so).
\end{align}
In Figs. \ref{fig:tfA} and \ref{fig:tfB} we plot the transverse part of the
diSPIM transfer function. Extending these plots to three-dimensions is simple
given our assumption that the axial part is Gaussian. Notice that view B has a
non-zero element in the $m=-1$ column---this is a first for the microscopes
we've considered so far. Also notice that each view contributes new
information---we'll see this result in a clearer form when we find the complete
SVD.

\fig{../calculations/out/Hillx.pdf}{1.0}{Transverse transfer function
  $H_{l,n,\tx{A}}^m(\bv_{xy})$ for view A of a polarized diSPIM microscope with
  a 0.8 NA detection objective. \textbf{Rows:} $l$ indexes the object-space
  spherical harmonic band. \textbf{Columns:} $m$ indexes the object-space
  spherical harmonic degree, and $n$ indexes the data-space circular harmonic
  band. \textbf{Entries:} Each column and row contains a continuous
  two-dimensional plot indexed by the vector $\mb{r}_{xy}$ ranging from
  $-2\text{NA}/\lambda$ to $+2\text{NA}/\lambda$ (although $\mb{r}_{xy}$ is
  rotationally symmetric so one dimension would suffice). All values are
  normalized between $-1$ \textcolor{blue}{(blue)} and $+1$
  \textcolor{red}{(red)} with $0$ colored white.}{tfA}
  % filename, width, caption, label

\fig{../calculations/out/Hillz.pdf}{1.0}{Transverse transfer function
  $H_{l,n,\tx{B}}^m(\bv_{yz})$ for view B of a polarized diSPIM microscope with
  a 0.8 NA detection objective. See caption of Fig. \ref{fig:tfA} for
  details.}{tfB}
  % filename, width, caption, label

\section{Single-view singular value decomposition}
We'll start by looking at the singular value decomposition of each view by
itself. We have already developed all of the tools we need for this computation
in the
\href{https://github.com/talonchandler/polharmonic/blob/master/notes/2018-05-14-single-view-continuous-svd/report/report.pdf}{May
  14 notes}, so we will jump straight to the results.

Figures \ref{fig:sysA} and \ref{fig:sysB} show the singular systems for the two
individual views of the diSPIM. As we'd expect, the object space singular
functions for the two views are identical except for a swap of the $x$ and $z$
axes. From the object space singular functions in the second and third branches
of both views, we can see each view is most sensitive to orientations
perpendicular to the illumination optical axis---this is what we'd expect for a
polarized illumination microscope.

Also notice that uniform distributions (or anything resembling a uniform
distribution) are not in the measurement space of either view. We saw the same
thing for single-view epi-illumination microscopes, although in that case there
were distributions that were mostly uniform with slight depressions along the
optical axis.

\fig{../calculations/out/SVSx.pdf}{1.0}{Singular system for view A
  ($x$-illumination, $z$-detection) of a polarized diSPIM microscope with a 0.8
  NA detection objective. \textbf{Rows:} Discrete branches of the singular value
  system indexed by $j$. \textbf{Column 1:} Continuous singular value spectra
  for each branch $j$ indexed by spatial frequency $\bs{\rho}$. The singular
  value spectra are normalized with contour lines at 0 and 0.1. \textbf{Columns
    2-5:} Angular part of the object-space singular vector at spatial
  frequencies marked with `x's in the first column. The camera is facing the
  origin along the [1,1,1] axis with the $z$ axis pointing up along the page. A
  \textcolor{red}{red} surface indicates positive values, a
  \textcolor{blue}{blue} surface indicates negative values, and the distance
  from the origin indicates the magnitude.}{sysA}
  % filename, width, caption, label

\fig{../calculations/out/SVSz.pdf}{1.0}{Singular system for view B
  ($z$-illumination, $x$-detection) of a polarized diSPIM microscope with a 0.8
  NA detection objective. See Fig. \ref{fig:sysA} caption for details. }{sysB}
  % filename, width, caption, label

\section{Dual-view singular value decomposition}
We will follow Burvall et al. \cite{burvall06} and the single-view case and
calculate the singular value decomposition in data space. We want to find the
eigenvalues $\mu_{\bp, j}$ and eigenfunctions $v_{\bp, j, \tv}(\rd, \p)$ that
satisfy
\begin{align}
  [\mc{H}\mc{H}^\dagger \mb{v}_{\bp, j}]_{\tv}(\rd, \p) = \mu_{\bp, j}v_{\bp, j, \tv}(\rd, \p). \label{eq:eig}
\end{align}
In the frequency domain the data-space singular functions will be in the form
\begin{align}
  v_{\bp, j, \tv}(\rd, \p) = \me^{i2\pi\bp\cdot\rd} \sum_{n=-\infty}^{\infty} \left[V_{n, \tv}(\bp)\right]_j z_n(\p). \label{eq:fdata}
\end{align}
Plugging Eqs. \ref{eq:ffwd}, \ref{eq:fadj}, and \ref{eq:fdata} in \ref{eq:eig} gives
\begin{align}
  \sum_{n=-\infty}^{n=\infty} \sum_{\tv \in \{\tx{A}, \tx{B}\}} \lmsum H_{l,\tv',n'}^m(\bp)H_{l,\tv,n}^m(\bp)\left[V_{n, \tv}(\bp)\right]_j = \mu_{\bp, j}\left[V_{n, \tv}(\bp)\right]_j. \label{eq:eig2}
\end{align}
If we combine the indices $n$ and $\tv$ into a multi-index $\mb{i} = (n, \tv)$ then we
can rewrite \ref{eq:eig2} as
\begin{align}
  \sum_{\mb{i}} \left[\lmsum H_{l,\mb{i}'}^m(\bp)H_{l,\mb{i}}^m(\bp)\right]\left[V_{\mb{i}}(\bp)\right]_j = \mu_{\bp, j}\left[V_{\mb{i}}(\bp)\right]_j. \label{eq:eig3}
\end{align}
Finally, we can rewrite Eq. \ref{eq:eig3} in matrix form as
\begin{align}
  \ms{K}(\bp)\ms{V}_j(\bp) = \mu_{\bp, j}\ms{V}_j(\bp),
\end{align}
where the entries of $\ms{K}(\bp)$ are given by
\begin{align}
  K_{\mb{i}, \mb{i}'} (\bp) = \lmsum H_{l,\mb{i}'}^m(\bp)H_{l,\mb{i}}^m(\bp). 
\end{align}
nFor the diSPIM, the matrix $\mathsf{K}$ has $6\times 6$ non-zero entries---two
views with three circular harmonics for each view---so solving this eigenvalue
problem at each point in frequency space is computationally feasible.



\bibliography{report}{}
\bibliographystyle{unsrt}

\end{document}

