
\documentclass[11pt]{article}

%%%%%%%%%%%%
% Packages %
%%%%%%%%%%%%
\usepackage[dvipsnames]{xcolor}
\hyphenpenalty=10000
\usepackage{tikz}
\usetikzlibrary{shapes,arrows}

\usepackage{tocloft}
\renewcommand\cftsecleader{\cftdotfill{\cftdotsep}}
\def\undertilde#1{\mathord{\vtop{\ialign{##\crcr
$\hfil\displaystyle{#1}\hfil$\crcr\noalign{\kern1.5pt\nointerlineskip}
$\hfil\tilde{}\hfil$\crcr\noalign{\kern1.5pt}}}}}
\usepackage{cleveref}
\usepackage{xcolor}
\usepackage[colorlinks = true,
            linkcolor = black,
            urlcolor  = blue,
            citecolor = black,
            anchorcolor = black]{hyperref}
\usepackage{epstopdf}
\usepackage{braket}
\usepackage{upgreek}
\usepackage{caption}
\usepackage{booktabs}
\usepackage{subcaption}
\usepackage{amssymb,latexsym,amsmath,gensymb}
\usepackage{latexsym}
\usepackage{graphicx}
\usepackage{float}
\usepackage{enumitem}
\usepackage{pdflscape}
\usepackage{url}
\usepackage{array}
\newcolumntype{C}{>{$\displaystyle} c <{$}}
\usepackage{tikz, calc}
\usetikzlibrary{shapes.geometric, arrows, calc}
\tikzstyle{norm} = [rectangle, rounded corners, minimum width=2cm, minimum height=1cm,text centered, draw=black]
\tikzstyle{arrow} = [thick, ->, >=stealth]

\newcommand{\argmin}{\arg\!\min}
\newcommand{\me}{\mathrm{e}}
\providecommand{\e}[1]{\ensuremath{\times 10^{#1}}} 
\providecommand{\mb}[1]{\mathbf{#1}}
\providecommand{\mf}[1]{\mathbf{#1}}
\providecommand{\mc}[1]{\mathcal{#1}}
\providecommand{\ro}[1]{\mathbf{\mathbf{r}}_o}
\providecommand{\so}[1]{\mathbf{\hat{s}}_o}
\providecommand{\rb}[1]{\mathbf{r}_b}
\providecommand{\rbm}[1]{r_b^{\text{m}}}
\providecommand{\rd}[1]{\mathbf{r}_d}
\providecommand{\mh}[1]{\mathbf{\hat{#1}}}
\providecommand{\mbb}[1]{\mathbb{#1}}
\providecommand{\bs}[1]{\boldsymbol{#1}} 
\providecommand{\intinf}{\int_{-\infty}^{\infty}}
\providecommand{\fig}[4]{
  % filename, width, caption, label
\begin{figure}[h]
 \captionsetup{width=1.0\linewidth}
 \centering
 \includegraphics[width = #2\textwidth]{#1}
 \caption{#3}
 \label{fig:#4}
\end{figure}
}

\makeatletter
\renewcommand*\env@matrix[1][*\c@MaxMatrixCols c]{%
  \hskip -\arraycolsep
  \let\@ifnextchar\new@ifnextchar
  \array{#1}}
\makeatother

\newcommand{\tensor}[1]{\overset{\text{\tiny$\leftrightarrow$}}{\mb{#1}}}
\newcommand{\tunderbrace}[2]{\underbrace{#1}_{\textstyle#2}}
\providecommand{\figs}[7]{
  % filename1, filename2, caption1, caption2, label1, label2, shift
\begin{figure}[H]
\centering
\begin{minipage}[b]{.45\textwidth}
  \centering
  \includegraphics[width=1.0\linewidth]{#1}
  \captionsetup{justification=justified, singlelinecheck=true}
  \caption{#3}
  \label{fig:#5}
\end{minipage}
\hspace{2em}
\begin{minipage}[b]{.45\textwidth}
  \centering
  \includegraphics[width=1.0\linewidth]{#2}
  \vspace{#7em}
  \captionsetup{justification=justified}
  \caption{#4}
  \label{fig:#6}
\end{minipage}
\end{figure}
}
\makeatletter

\providecommand{\code}[1]{
\begin{center}
\lstinputlisting{#1}
\end{center}
}

\newcommand{\crefrangeconjunction}{--}
%%%%%%%%%%%
% Spacing %
%%%%%%%%%%%
% Margins
\usepackage[
top    = 1.5cm,
bottom = 1.5cm,
left   = 1.5cm,
right  = 1.5cm]{geometry}

% Indents, paragraph space
%\usepackage{parskip}
\setlength{\parskip}{1.5ex}

% Section spacing
\usepackage{titlesec}
\titlespacing*{\title}
{0pt}{0ex}{0ex}
\titlespacing*{\section}
{0pt}{0ex}{0ex}
\titlespacing*{\subsection}
{0pt}{0ex}{0ex}
\titlespacing*{\subsubsection}
{0pt}{0ex}{0ex}

% Line spacing
\linespread{1.1}

%%%%%%%%%%%%
% Document %
%%%%%%%%%%%%
\begin{document}
\title{\vspace{-2.5em} Singular value decomposition of single view\\ polarized
  fluorescence microscopes\vspace{-1em}} \author{Talon Chandler, Min Guo, Hari
  Shroff, Rudolf Oldenbourg, Patrick La Rivi\`ere}
\date{\vspace{-1em}\today\vspace{-1em}}
\maketitle
\section{Introduction}
In these notes we will develop the continuous models for single view polarized
fluorescence microscopes with either polarized illumination or polarized
detection. We will start by writing an integral transform that maps object space
to data space. Then we will calculate the kernels of the integral transform and
their corresponding transfer functions. Finally we will calculate the singular
value decomposition of the forward operator to find the limits of these designs.

\noindent We will make use of the real circular harmonic functions
\begin{align}
  z_n(\theta) =
  \begin{cases}
    \frac{1}{\sqrt{\pi}}\cos(n\theta), & n > 0,\\
    \frac{1}{\sqrt{2\pi}}, & n = 0,\\
    \frac{1}{\sqrt{\pi}}\sin(n\theta), & n < 0.\\
  \end{cases}
\end{align}
The real circular harmonic functions form an orthonormal basis for functions on
the circle because
\begin{align}
  \int_{\mbb{S}^1}d\mh{p}\, z_n(\mh{p})z_{n'}(\mh{p}) = \delta_{nn'}.
\end{align}
Notice that we allow $z_n(\cdot)$ to take a vector argument similar to the
spherical harmonics $y_l^m(\so{})$. This allows us to use these functions
without specifying a coordinate system. Also notice that we are using $n$ to
index the circular harmonic functions---in a previous note set we used $n$ to
index frames or views.

\section{General forward model}
We will consider the problem of imaging a two-dimensional field of oriented
fluorophores. To a good approximation these objects can be represented by a
member of the set
$\mbb{U} = \mbb{L}_2(\mbb{R}^2 \times \mbb{S}^2)$---square-integrable functions
that assign a scalar value to each position and orientation.

These microscopes can collect a two-dimensional frame of intensity measurements
for each position of a single polarizer on the illumination or detection arm.
Therefore, the collected data is a member of the set
$\mbb{V} = \mbb{L}_2(\mbb{R}^2 \times \mbb{S}^1)$---square-integrable functions
that assign a scalar value to each two-dimensional position on a circle. Notice
that we are considering the largest possible data space for these
microscopes---a continuous sampling of the spatial and polarizer dimensions.
This approach allows us to answer questions about the limits of these
microscopes without considering specific choices of spatial sampling (pixels) or
polarization sampling (polarizer settings). It will be helpful to think of the
space $\mbb{L}_2(\mbb{R}^2 \times \mbb{S}^1)$ geometrically as a continuous list
of images arranged face to face in a donut---if we slice the donut along the
smaller dimension (not like a pre-cut bagel!) we can see a single
image taken with a single polarizer setting.

If the imaging system is spatially shift-invariant, then the forward and adjoint operators are given by the following integral transforms
\begin{align}
  g(\rd{}, \mh{p}) = [\mathcal{H}f](\rd{}, \hat{\mb{p}}) &= \int_{\mbb{S}^2}d\so{}\int_{\mbb{R}^2}d\ro{}\, h(\rd{} -\ro{}, \so{}; \hat{\mb{p}})f(\ro{}, \so{}),\\
  f(\ro{}, \so{}) =  [\mathcal{H}^{\dagger}g](\ro{}, \so{}) &= \int_{\mbb{S}^1}d\hat{\mb{p}}\, \int_{\mbb{R}^2}d\mb{r}_{d}\, h(\rd{} - \ro{}, \so{}; \hat{\mb{p}})g(\rd{}, \hat{\mb{p}}),
\end{align}
where $f(\ro{}, \so{})$ is the object, $g(\rd{}, \hat{\mb{p}})$ is the data, and
$h(\rd{} -\ro{}, \so{}; \hat{\mb{p}})$ is the kernel of the integral transform.
The kernel can be interpreted as a ``point spread function'' if we visualize the
complete data space as a ``donut'' of images. A delta function at a point in object
space will give rise to the kernel evaluated at that point in data space.

We can write the forward and adjoint operators in the frequency domain as
\begin{align}
  G_n(\bs{\nu}) &= \sum_{l=0}^{\infty}\sum_{m=-l}^{l}H_{l,n}^m(\bs{\nu})F_l^m(\bs{\nu}),\label{eq:fwd}\\
  F_l^m(\bs{\nu}) &= \sum_{n=-\infty}^{\infty}H_{l,n}^m(\bs{\nu})G_n(\bs{\nu}),
\end{align}
where
\begin{align}
  G_{n}(\bs{\nu}) &= \int_{\mbb{S}^1}d\mh{p}\, z_n(\mh{p})\int_{\mbb{R}^2}d\rd{}\, \me{}^{i2\pi\rd{}\cdot\bs{\nu}}g(\rd{}, \mh{p}),\\
  H_{l,n}^m(\bs{\nu}) &= \int_{\mbb{S}^1}d\mh{p}\, z_n(\mh{p})\int_{\mbb{S}^2}d\so{}\, y_l^m(\so{})\int_{\mbb{R}^2}d\ro{}\, \me{}^{i2\pi\ro{}\cdot\bs{\nu}}h(\ro{}, \so{}; \mh{p}),\label{eq:genft}\\ 
  F_{l}^m(\bs{\nu}) &= \int_{\mbb{S}^2}d\so{}\, y_l^m(\so{})\int_{\mbb{R}^2}d\ro{}\, \me{}^{i2\pi\ro{}\cdot\bs{\nu}}f(\ro{}, \so{}).
\end{align}
To find the singular value decomposition of the forward operator, we need to solve
\begin{align}
  \mc{H}\mc{H}^{\dagger} v_{\bs{\rho},j}(\rd{}, \hat{\mb{p}}) = \mu_{\bs{\rho},j}v_{\bs{\rho},j}(\rd{}, \hat{\mb{p}}). 
\end{align}
It is much easier to solve the equivalent problem in the frequency domain
\begin{align}
  \mathsf{K}(\bs{\rho})\mathsf{V}_j(\bs{\rho}) = \mu_{\bs{\rho},j}\mathsf{V}_j(\bs{\rho}). \label{eq:freqsvd}
\end{align}
where the entries of $\mathsf{K}(\bs{\rho})$ are given by
\begin{align}
  K_{nn'}(\bs{\rho}) = \sum_{l=0}^\infty\sum_{m=-l}^{l}H_{l,n}^m(\bs{\rho})H_{l,n'}^m(\bs{\rho}),
\end{align}
and the entries of $\mathsf{V}_j(\bs{\rho})$ are related to the complete data space singular functions by
\begin{align}
  v_{\bs{\rho},j}(\rd{}, \mh{p}) = \me{}^{i 2\pi \bs{\rho}\cdot\rd{}}\sum_{n=-\infty}^{\infty}[V_n(\bs{\rho})]_j z_n(\mh{p}). \label{eq:eigfunc}
\end{align}
Once we've found the data space singular functions, we can find the object space singular functions using
\begin{align}
  [U_l^m(\bs{\rho})]_j &= \sum_{n=-\infty}^{\infty} H_{l,n}^m(\bs{\rho})[V_n(\bs{\rho})]_j, \label{eq:eigenfuncobj}\\
  u_{\bs{\rho}, j}(\ro{}, \so{}) &= \me{}^{i2\pi\bs{\rho}\cdot\ro{}}\sum_{l=0}^{\infty}\sum_{m=-l}^l [U_l^m(\bs{\rho})]_j y_l^m(\so{}).
\end{align}

\section{Polarized illumination}
\subsection{Kernel}
In previous notes we showed that the excitation point response function for
polarized epi-illumination is given by
\begin{align}
  h^{\mh{z}}_{\text{exc}}(\so{}; \mh{p}) &= y_0^0(\so{}) - \frac{1}{\sqrt{5}}\tilde{A}y_2^0(\so{}) + \sqrt{\frac{3}{5}}\tilde{B}\left\{[(\mh{p}\cdot\mh{x})^2 - (\mh{p}\cdot\mh{y})^2]y_2^2(\so{}) - 2(\mh{p}\cdot\mh{x})(\mh{p}\cdot\mh{y})y_2^{-2}(\so{})\right\}, \label{eq:genpsf}
\end{align}
where
\begin{subequations}
\begin{align}
  \tilde{A} &\equiv \cos^2(\alpha/2)\cos(\alpha),\\
  \tilde{B} &\equiv \frac{1}{12}(\cos^2\alpha + 4\cos\alpha + 7),
\end{align}\label{eq:coefficients}%
\end{subequations}
and $\alpha \equiv \arcsin(\text{NA}/n_o)$. It is more convenient to write the point response function in terms of the circular harmonics
\begin{align}
    h^{\mh{z}}_{\text{exc}}(\so{}; \mh{p}) &= y_0^0(\so{}) - \frac{1}{\sqrt{5}}\tilde{A}y_2^0(\so{}) + \sqrt{\frac{3}{5}}\tilde{B}\left\{y_2^2(\so{})z_2(\mh{p}) - y_2^{-2}(\so{})z_{-2}(\mh{p})\right\}.
\end{align}
We derived this excitation kernel using the model from Chandler et al.
\cite{chandler17}---a model modified from Fourkas \cite{fourkas2001}. Both
papers claim that their equations model aplanatic (shift-invariant) objectives,
but this is not strictly true. Fourkas and Chandler ignore the
$1/\sqrt{\cos\theta}$ apodization factor for aplanatic objectives and
unknowingly model a non-aplanatic objective that satisfies the Herschel
condition instead of Abbe's sine condition. The difference between these
models is small but non-zero under the paraxial approximation. 

In the most recent string of note sets I have been modeling a true aplanatic
objective under the paraxial approximation. The correct paraxial excitation model
for an aplanatic objective is 
\begin{align}
  h^{\text{(ap)},\mh{z}}_{\text{exc}}(\so{}; \mh{p}) &= y_0^0(\so{}) - \frac{1}{\sqrt{5}}\left[1 - 2\left(\frac{\text{NA}}{n_o}\right)^2\right]y_2^0(\so{}) + \sqrt{\frac{3}{5}}\left\{y_2^2(\so{})z_2(\mh{p}) - y_2^{-2}(\so{})z_{-2}(\mh{p})\right\}. \label{eq:aplan}
\end{align}
We will drop the (ap) superscript unless there is possibility for confusion. Eq.
\ref{eq:aplan} shows that for an aplanatic objective under the paraxial
approximation, the $y_2^0$ term is excited less relative to the $y_0^0$ term as
the NA increases (recall that the kernel is normalized). We also see that the
$y_2^2$ and $y_2^{-2}$ terms do not depend on the NA, but they do depend on the
polarizer orientation.

Relevant discussion of Abbe's sine condition, the Herschel condition,
apodization functions, and the paraxial approximation can be found in Barrett
9.6.7 \cite{barrett2004}, Mansuripur 1.0 \cite{mansuripur2002}, and especially
Gu 6.3 \cite{gu2000}. Although we (and Fourkas) incorrectly stated that we were
modeling an aplanatic objective in \cite{chandler17}, we brought attention to
our lack of apodization in the discussion section. All of our major conclusions
would still hold if we used the correct apodization (and our numerical results
hold for Herschel objectives although these are rarely used in microscopy).

We also showed that the kernel for unpolarized epi-detection is
given by
\begin{align}
  h_{\text{det}}(\ro{}, \so{}) &= [{a}^2(r_o) + 2b^2(r_o)]y_0^0(\so{}) + \frac{1}{\sqrt{5}}\left[- a^2(r_o) + 4b^2(r_o)\right]y_2^0(\so{}),
\end{align}
where
\begin{align}
  a(r_o) = \frac{J_1(2\pi \nu_or_o)}{\pi \nu_or_o}, 
  &\hspace{2em}
    b(r_o) = \frac{\text{NA}}{n_o}\left[\frac{J_2(2\pi \nu_or_o)}{\pi \nu_or_o}\right],  \label{eq:abparadef}
  \intertext{and}
  \nu_o \equiv \frac{\text{NA}}{\lambda},&\hspace{2em}
  \text{NA} = n_o\sin\alpha.
\end{align}
The excitation and detection processes are incoherent, so to find the complete
kernel we multiply the excitation and detection kernels
\begin{align}
  h(\ro{}, \so{}; \mh{p}) = h^{\mh{z}}_{\text{exc}}(\so{}; \mh{p})h_{\text{det}}(\ro{}, \so{}) = \sum_{l=0}^{\infty}\sum_{m=-l}^l\sum_{n=-\infty}^{\infty} h_{l,n}^m(\ro{})y_l^m(\so{})z_n(\mh{p}).
\end{align}
In previous notes we were able to calculate these products in terms of the
circular and spherical harmonics by hand, but this is time consuming and error
prone. To simplify the multiplication we precompute the triple integrals of the
circular and spherical harmonics and compute the product using a tensor
product---see Appendix A for details.

We can write a closed form expression for the kernel, but the expression is long
and not particularly useful. Instead, we plot each term of the kernel in Fig.
\ref{fig:illkern}.

\fig{../calculations/out/hhill.pdf}{1.0}{Five-dimensional kernel
  $h_{l,n}^m(\ro{})$ for a single-view polarized illumination microscope with
  0.8 NA epi-illumination and epi-detection. \textbf{Rows:} $l$ indexes the
  object-space spherical harmonic band. \textbf{Columns:} $m$ indexes the
  object-space spherical harmonic degree, and $n$ indexes the data-space
  circular harmonic band. \textbf{Entries:} Each column and row contains a
  continuous two-dimensional plot indexed by the vector $\ro{}$ ranging from
  $-2\lambda/\text{NA}$ to $+2\lambda/\text{NA}$ (although $\ro{}$ is
  rotationally symmetric so one dimension would suffice). All values are
  normalized between $-1$ \textcolor{blue}{(blue)} and $+1$
  \textcolor{red}{(red)} with $0$ colored white.}{illkern}
  % filename, width, caption, label

Each of the ($l,m,n$) terms of the kernel shown in Fig. \ref{fig:illkern} can be
interpreted as the shape of the $z_n(\mh{p})$ component of the intensity pattern
detected when the object consists of $y_l^m(\so{})$ distributed dipoles. For
example, the $l=0$, $m=0$, $n=0$ term shows the intensity pattern for the
$z_0(\mh{p})$ (constant) component detected when the object consists of
$y_0^0(\so{})$ (uniformly distributed) dipoles. Notice that the $l=0$, $m=0$,
$n=\pm 2$ terms are zero which means that the intensity pattern from a uniform
distribution of fluorophores is independent of the polarizer orientation.

Interestingly, every ($l$, $m$) term in the kernel has a single $n$ term that is
non-zero. This means that each spherical harmonic component in object space
gives rise to a single circular harmonic component in data space. This property
is specific to polarized illumination microscopes.

Finally, notice that the polarizer dimension is limited to $n= \{-2, 0, 2\}$.
The three terms mean that three polarizer orientations are sufficient to
completely sample that dimension. We have suspected that three measurements are
sufficient because the polarization dependence of an intensity signal can be
specified with three numbers---an offset, an amplitude, and a phase. The three
circular harmonic components of the signal capture the same information as the
offset, amplitude, and phase; and they are orthogonal functions so they allow us
to use frequency domain techniques.

\subsection{Transfer function}
We can calculate the transfer function using Eq. \ref{eq:genft} and the kernel,
but this will lead to a very long expression. Instead, we calculate the
illumination transfer function and the detection transfer function then use the
multiplication rules in Appendix A to find the complete transfer function.

The illumination transfer function is
\begin{align}
  H^m_{\text{exc},l,n} &= \delta_{l,0}\delta_{m,0}\delta_{n,0} - \frac{1}{\sqrt{5}}\left[1 - 2\left(\frac{\text{NA}}{n_o}\right)^2\right]\delta_{l,2}\delta_{m,0}\delta_{n,0} + \sqrt{\frac{3}{5}}\left\{\delta_{l,2}\delta_{m,2}\delta_{n,2} - \delta_{l,-2}\delta_{m,-2}\delta_{n,-2}\right\}, 
\end{align}
and the detection transfer function is 
\begin{align}
  H^m_{\text{det},l,n}(\nu) &= [A(\nu) + 2B(\nu)]\delta_{l,0}\delta_{m,0}\delta_{n,0} + \frac{1}{\sqrt{5}}\left[- A(\nu) + 4B(\nu)\right]\delta_{l,2}\delta_{m,0}\delta_{n,0},
\end{align}
where
\begin{align}
  A(\nu) &= \frac{2}{\pi}\left[\arccos\left(\frac{\nu}{2\nu_o}\right) - \frac{\nu}{2\nu_o}\sqrt{1 - \left(\frac{\nu}{2\nu_o}\right)^2}\right]\Pi\left(\frac{\nu}{2\nu_o}\right),\\
  B(\nu) &= \frac{1}{\pi}\left(\frac{\text{NA}}{n_o}\right)^2\left[\arccos\left(\frac{\nu}{2\nu_o}\right) - \left[3 - 2\left(\frac{\nu}{2\nu_o}\right)^2\right]\frac{\nu}{2\nu_o} \sqrt{1 - \left(\frac{\nu}{2\nu_o}\right)^2}\right]\Pi\left(\frac{\nu}{2\nu_o}\right).                 
\end{align}
The complete transfer function is given by
\begin{align}
  H_{l,n}^m(\nu) = H^m_{\text{exc},l,n} * H^m_{\text{det},l,n}(\nu),
\end{align}
where $*$ denotes a generalized convolution over all five indices. In Appendix A
we show how to evaluate the convolution over the discrete indices. The
excitation pattern is spatially uniform, so the spatial convolution is trivial.
We plot the complete transfer function in Fig. \ref{fig:illtran}.
\fig{../calculations/out/Hill.pdf}{1.0}{Five-dimensional transfer function
  $H_{l,n}^m(\bs{\nu})$ for a single-view polarized illumination microscope with
  0.8 NA epi-illumination and epi-detection. \textbf{Rows:} $l$ indexes the
  object-space spherical harmonic band. \textbf{Columns:} $m$ indexes the
  object-space spherical harmonic degree, and $n$ indexes the data-space
  circular harmonic band. \textbf{Entries:} Each column and row contains a
  continuous two-dimension plot indexed by the vector $\bs{\nu}$ ranging from
  $-2\text{NA}/\lambda$ to $+2\text{NA}/\lambda$. All values are normalized
  between $-1$ \textcolor{blue}{(blue)} and $+1$ \textcolor{red}{(red)} with $0$
  colored white. Black lines are contours at 0 and $\pm 0.1$.}{illtran}
% filename, width, caption, label

\subsection{Singular value decomposition}
Plugging the transfer function into the frequency-domain eigenvalue problem (Eq.
\ref{eq:freqsvd}) yields a denumerably infinite eigenvalue problem with only
$3\times 3$ non-zero entries in the matrix (we suppress the index $\bs{\rho}$
and understand that we need to solve this eigenvalue problem at each value of
$\bs{\rho}$)
\begin{align}
  \mathsf{K}\mathsf{V}_j &= \mu_{j}\mathsf{V}_j,\\
  \begin{bmatrix}
    \sum_{l,m}H_{l,-2}^{m}H_{l,-2}^{m}&\sum_{l,m}H_{l,-2}^{m}H_{l,0}^{m}&\sum_{l,m}H_{l,-2}^{m}H_{l,2}^{m}\\
    \sum_{l,m}H_{l,0}^{m}H_{l,-2}^{m}&\sum_{l,m}H_{l,0}^{m}H_{l,0}^{m}&\sum_{l,m}H_{l,0}^{m}H_{l,2}^{m}\\
    \sum_{l,m}H_{l,2}^{m}H_{l,-2}^{m}&\sum_{l,m}H_{l,2}^{m}H_{l,0}^{m}&\sum_{l,m}H_{l,2}^{m}H_{l,2}^{m}\\    
  \end{bmatrix}\mathsf{V}_j
&= \mu_{j}\mathsf{V}_j,\\
  \begin{bmatrix}
    \{H_{2,-2}^{-2}\}^2 + \{H_{4,-2}^{-2}\}^2&0&0\\
    0&\{H_{0,0}^{0}\}^2 + \{H_{2,0}^{0}\}^2 + \{H_{4,0}^{0}\}^2&0\\
    0&0&\{H_{2,2}^{2}\}^2 + \{H_{4,2}^{2}\}^2\\    
  \end{bmatrix}\mathsf{V}_j
&= \mu_{j}\mathsf{V}_j.
\end{align}
A diagonal matrix has its eigenvalues along the diagonal and its eigenvectors as
unit vectors so
\begin{align}
  \mu_{\bs{\rho},0} &= \{H_{0,0}^{0}(\rho)\}^2 + \{H_{2,0}^{0}(\rho)\}^2 + \{H_{4,0}^{0}(\rho)\}^2,\\
  \mu_{\bs{\rho},1} = \mu_{\bs{\rho},2} &= \{H_{2,2}^{2}(\rho)\}^2 + \{H_{4,2}^{2}(\rho)\}^2,
\end{align}
and
\begin{align}
  \mathsf{V}_0 =
  \begin{bmatrix}
    0\\1\\0\\
  \end{bmatrix},\qquad
  \mathsf{V}_1 =
  \begin{bmatrix}
    1\\0\\0\\
  \end{bmatrix},\qquad
  \mathsf{V}_2 =
  \begin{bmatrix}
    0\\0\\1\\
  \end{bmatrix}.
\end{align}
Notice that we have two degenerate eigenvalues, but we have found trivial
orthonormal eigenvectors---in general we will need to apply the Gram-Schmidt
procedure.

% We can substitute the $H_{l,n}^m$ functions to recover the complete eigenvalue
% spectrum as
% \begin{align}
%   \mu_{\bs{\rho},0} = &\frac{1}{70}[3\tilde{A}^2A^2(\rho) +2\tilde{A}A^2(\rho) + 21A^2(\rho) + 6\tilde{A}^2A(\rho)B(\rho) - 56\tilde{A}A(\rho)B(\rho) + \\ & \qquad 98A(\rho)B(\rho) + 54\tilde{A}^2B(\rho)^2 - 144\tilde{A}B^2(\rho) + 126B^2(\rho)],\\
%   \mu_{\bs{\rho},1} = &\mu_{\bs{\rho}, 2} = \frac{9\tilde{B}}{35}[A^2(\rho) + A(\rho)B(\rho) + B^2(\rho)].
% \end{align}
We can use Eq. \ref{eq:eigfunc} to recover the complete singular functions in
data space as
\begin{align}
  v_{\bs{\rho},0}(\rd{}, \mh{p}) &= \me{}^{i2\pi\bs{\rho}\cdot\rd{}}z_0(\mh{p}),\\
  v_{\bs{\rho},1}(\rd{}, \mh{p}) &= \me{}^{i2\pi\bs{\rho}\cdot\rd{}}z_{-2}(\mh{p}),\\
  v_{\bs{\rho},2}(\rd{}, \mh{p}) &= \me{}^{i2\pi\bs{\rho}\cdot\rd{}}z_2(\mh{p}).
\end{align}
For the final step we use Eq. \ref{eq:eigenfuncobj} to find the spectra of the
singular functions in object space
\begin{align}
  [U_l^m(\rho)]_0 &= H_{l,0}^m(\rho) = H_{0,0}^0(\rho)\delta_{l0}\delta_{m0} + H_{2,0}^0(\rho)\delta_{l2}\delta_{m0} + H_{4,0}^0(\rho)\delta_{l4}\delta_{m0},\\
  [U_l^m(\rho)]_1 &= H_{l,-2}^m(\rho) = H_{2,-2}^{-2}(\rho)\delta_{l2}\delta_{m-2} + H_{4,-2}^{-2}(\rho)\delta_{l4}\delta_{m-2},\\
  [U_l^m(\rho)]_2 &= H_{l,2}^m(\rho) = H_{l,2}^m(\rho) = H_{2,2}^{2}(\rho)\delta_{l2}\delta{m2} + H_{4,2}^{2}(\rho)\delta_{l4}\delta_{m2}.
\end{align}
Therefore, the complete singular functions in object space are
\begin{align}
  u_{\bs{\rho},0}(\ro{}, \so{}) &= \me{}^{i2\pi\bs{\rho}\cdot\ro{}}[H_{0,0}^0(\rho)y_0^0(\so{}) + H_{2,0}^0(\rho)y_2^0(\so{}) + H_{4,0}^0(\rho)y_4^0(\so{})],\\
  u_{\bs{\rho},1}(\ro{}, \so{}) &= \me{}^{i2\pi\bs{\rho}\cdot\ro{} }[H_{2,-2}^{-2}(\rho)y_2^{-2}(\so{}) + H_{4,-2}^{-2}(\rho)y_4^{-2}(\so{})],\\
  u_{\bs{\rho},2}(\ro{}, \so{}) &= \me{}^{i2\pi\bs{\rho}\cdot\ro{} }[H_{2,2}^2(\rho)y_2^2(\so{}) + H_{4,2}^2(\rho)y_4^2(\so{})].
\end{align}
Finally, the singular values are the square root of the eigenvalues
\begin{align}
  \sigma_{\bs{\rho},j} = \sqrt{\mu_{\bs{\rho},j}}. 
\end{align}

We have found the complete singular system for a single view microscope with
polarized illumination. In Fig. \ref{fig:illsvs} we show all three branches of
the singular spectrum along with the angular part of the object space singular
functions $u_{\bs{\rho},j}(0, \so{})$.

\fig{../calculations/out/SVSill.pdf}{1.0}{Singular system for a single-view
  polarized illumination microscope with 0.8 NA epi-illumination and
  epi-detection. \textbf{Rows:} Discrete branches of the singular value system
  indexed by $j$. \textbf{Column 1:} Continuous singular value spectra for each
  branch $j$ indexed by spatial frequency $\bs{\rho}$. The singular value
  spectra are normalized with contour lines at 0 and 0.1. \textbf{Columns 2-5:}
  Angular part of the object-space singular vector at spatial frequencies marked
  with `x's in the first column. The camera is facing the origin along the
  [1,1,1] axis with the optical axis of the microscope pointing up along the
  page. A \textcolor{red}{red} surface indicates positive values, a
  \textcolor{blue}{blue} surface indicates negative values, and the distance
  from the origin indicates the magnitude.}{illsvs}

All three branches of the singular system have object-space singular functions
that depend on the spatial frequency (the plots in each column and row of Fig.
\ref{fig:illsvs} are different). This dependence is largest in the $j=0$ branch
where at low spatial frequencies the singular functions are nearly spherical,
while at high spatial frequencies the singular functions are negative along the
optical axis. The fact that the singular functions of the $j=0$ branch are
non-spherical and have a spatial-frequency dependence is the most important
result of this section---it means that a single view microscope can distinguish
distributions of fluorophores that have different out-of-transverse-plane
components. The $j=1$ and $j=2$ object-space singular functions still depend on
the spatial frequency---the high-frequency singular functions are ``flatter'' in
the transverse plane than the low-frequency singular functions---but this
dependence is relatively small.

We can use the singular value spectrum to find the degeneracies and symmetries
of the microscope. The first major set of degeneracies is related to the fact
that $\sigma_{\bs{\rho},1} = \sigma_{\bs{\rho},2}$. The corresponding singular
functions in object space are related by changing the spherical harmonic index
$m$ to $-m$. This transformation is a rotation about the optical axis, so this
degeneracy corresponds to the rotational symmetry of the microscope. The second
set of degeneracies is related to the fact that the singular values depend only
on $\rho$, not the vector $\bs{\rho}$. The corresponding singular functions in
object space are spatial harmonics at the same spatial frequency in different
directions. Once again, this degeneracy corresponds to the rotational symmetry
of the microscope.

Recall the mental image of data space being arranged on a donut. If we rotate
the object that we are imaging then the data space will undergo two simultaneous
rotations---one rotation about an axis through the donut hole (imagine sticking
your finger through the donut) and another about the curved axis that runs
through the center of the dough (imagine twisting the donut in on itself so that
glaze on top would end up on the bottom). These two simultaneous rotations
correspond with the two sets of degeneracies we see in the singular value
spectrum.

We have only plotted the object space singular functions for spatial frequencies
along the $x$-axis, but these are sufficient to characterize the singular
functions for any spatial frequency direction. To see this, notice that the
$j=0$ singular functions are rotationally symmetric about the optical axis and
the $j=1$ and $j=2$ singular functions are degenerate and they span a
rotationally symmetric space of singular functions.

\section{Polarized detection}
We'll use the same notation as the previous section, so there will be notation
overlap.

\subsection{Kernel}
Earlier we showed that the excitation point response function for
unpolarized epi-illumination is given by
\begin{align}
  h^{\mh{z}}_{\text{exc}}(\so{}) &= y_0^0(\so{}) - \frac{1}{\sqrt{5}}\left[1 - 2\left(\frac{\text{NA}}{n_o}\right)^2\right]y_2^0(\so{}).
\end{align}
We also showed that the point response function for polarized detection is given
by (rewritten in terms of the circular harmonics)
\begin{align}
  h_{\text{det}}(\ro{}, \so{}; \mh{p}) = \sum_{l=0}^{\infty}\sum_{m=-l}^l\sum_{n=-\infty}^{\infty} h_{\text{det},l,n}^m(\ro{})y_l^m(\so{})z_n(\mh{p}_d), 
\end{align}
where
\begin{align}
  h_{\text{det},0,0}^0(\ro{}) &= a^2(r_o) + 2b^2(r_0),\\
  h_{\text{det},0,\pm 2}^0(\ro{}) &= 2\sqrt{\pi} b^2(r_0)z_{\pm 2}(\ro{}),\\
  h_{\text{det},2,0}^0(\ro{}) &= \frac{1}{\sqrt{5}}\left[-a^2(r_o) + 4b^2(r_0)\right],\\
  h_{\text{det},2,\pm 2}^0(\ro{}) &= 4\sqrt{\frac{\pi}{5}}b^2(r_0)z_{\pm 2}(\ro{}),\\
  h_{\text{det},2,2}^2(\ro{}) &= -h_{\text{det},2,-2}^{-2}(\ro{}) = \sqrt{\frac{3}{5}}a^2(r_o).
\end{align}
The complete kernel for unpolarized illumination and polarized detection is
given by the product of the excitation and detection kernels. We plot the
result in Fig. \ref{fig:detkern}.

\fig{../calculations/out/hhdet.pdf}{1.0}{Five-dimensional kernel
  $h_{l,n}^m(\ro{})$ for a single-view polarized detection microscope with
  0.8 NA epi-illumination and epi-detection. \textbf{Rows:} $l$ indexes the
  object-space spherical harmonic band. \textbf{Columns:} $m$ indexes the
  object-space spherical harmonic degree, and $n$ indexes the data-space
  circular harmonic band. \textbf{Entries:} Each column and row contains a
  continuous two-dimensional plot indexed by the vector $\ro{}$ ranging from
  $-2\lambda/\text{NA}$ to $+2\lambda/\text{NA}$ (although $\ro{}$ is
  rotationally symmetric so one dimension would suffice). All values are
  normalized between $-1$ \textcolor{blue}{(blue)} and $+1$
  \textcolor{red}{(red)} with $0$ colored white.}{detkern}
  % filename, width, caption, label

In the polarized illumination case every ($l$, $m$) term in the kernel had a
single $n$ term that is non-zero. For polarized detection this is no longer
true---for example there are non-zero components for $l=0$, $m=0$ in the
$n=0, \pm 2$ terms. This means that each spherical harmonic component in object
space gives rise to multiple circular harmonic components in data space.

\subsection{Transfer function}
The transfer function for unpolarized illumination is
\begin{align}
  H^m_{\text{exc}, l,n} &= \delta_{l,0}\delta_{m,0}\delta_{n,0} - \frac{1}{\sqrt{5}}\left[1 - 2\left(\frac{\text{NA}}{n_o}\right)^2\right]\delta_{l,2}\delta_{m,0}\delta_{n,0},
\end{align}
and the transfer function for polarized detection is
\begin{align}
  H_{\text{det},l,n}^m(\bs{\nu}) = \sum_{l=0}^{\infty}\sum_{m=-l}^l\sum_{n=-\infty}^{\infty} H_{\text{det},l,n}^m(\bs{\nu})\delta_{l,l'}\delta_{m,m'}\delta_{n,n'}, 
\end{align}
where
\begin{align}
  H_{\text{det},0,0}^0(\bs{\nu}) &= A(\nu) + 2B(\nu),\\
  H_{\text{det},0,\pm 2}^0(\bs{\nu}) &= 2\sqrt{\pi} C(\nu)z_{\pm 2}(\bs{\nu}),\\
  H_{\text{det},2,0}^0(\bs{\nu}) &= \frac{1}{\sqrt{5}}\left[-A(\nu) + 4B(\nu)\right],\\
  H_{\text{det},2,\pm 2}^0(\bs{\nu}) &= 4\sqrt{\frac{\pi}{5}}C(\nu)z_{\pm 2}(\bs{\nu}),\\
  H_{\text{det},2,2}^2(\bs{\nu}) &= -H_{\text{det},2,-2}^{-2}(\bs{\nu}) = \sqrt{\frac{3}{5}}A(\nu).
\end{align}
and
\begin{align}
  C(\nu) &= \frac{1}{\pi}\left(\frac{\text{NA}}{n_o}\right)^2\left[-\frac{4}{3}\frac{\nu}{2\nu_o}\sqrt[3]{1 - \left(\frac{\nu}{2\nu_o}\right)^2}\right]\Pi\left(\frac{\nu}{2\nu_o}\right).
\end{align}
Fig. \ref{fig:dettran} shows the complete transfer function for a polarized
detection microscope. 

\fig{../calculations/out/Hdet.pdf}{1.0}{Five-dimensional transfer function
  $H_{l,n}^m(\bs{\nu})$ for a single-view polarized detection microscope with
  0.8 NA epi-illumination and epi-detection. \textbf{Rows:} $l$ indexes the
  object-space spherical harmonic band. \textbf{Columns:} $m$ indexes the
  object-space spherical harmonic degree, and $n$ indexes the data-space
  circular harmonic band. \textbf{Entries:} Each column and row contains a
  continuous two-dimension plot indexed by the vector $\bs{\nu}$ ranging from
  $-2\text{NA}/\lambda$ to $+2\text{NA}/\lambda$. All values are normalized
  between $-1$ \textcolor{blue}{(blue)} and $+1$ \textcolor{red}{(red)} with $0$
  colored white. Black lines are contours at 0 and $\pm 0.1$.}{dettran}
% filename, width, caption, label


\subsection{Singular value decomposition}
We follow the same procedure and plug the transfer function into the
frequency-domain eigenvalue problem (Eq. \ref{eq:freqsvd}) which gives a
$3\times 3$ eigenvalue problem
\begin{align}
  \mathsf{K}\mathsf{V}_j &= \mu_{j}\mathsf{V}_j,
\end{align}
where the entries of $\mathsf{K}$ are given by
\begin{align}
  \mathsf{K}_{00} &= \sum_{l,m}H_{l,-2}^{m}H_{l,-2}^{m} = \{H_{0,-2}^{0}\}^2 + \{H_{2,-2}^{-2}\}^2 + \{H_{2,-2}^{0}\}^2 + \{H_{4,-2}^{-2}\}^2 + \{H_{4,-2}^{0}\}^2,\\
  \mathsf{K}_{11} &= \sum_{l,m}H_{l,0}^{m}H_{l,0}^{m} = \{H_{0,0}^{0}\}^2 + \{H_{2,0}^{0}\}^2 + \{H_{4,0}^{0}\}^2,\\
  \mathsf{K}_{22} &= \sum_{l,m}H_{l,2}^{m}H_{l,2}^{m} = \{H_{0,2}^{0}\}^2 + \{H_{2,2}^{2}\}^2 + \{H_{2,2}^{0}\}^2 + \{H_{4,2}^{2}\}^2 + \{H_{4,2}^{0}\}^2,\\
  \mathsf{K}_{01} &= \mathsf{K}_{10} = \sum_{l,m}H_{l,-2}^{m}H_{l,0}^{m} = H_{0,-2}^{0}H_{0,0}^{0} + H_{2,-2}^{0}H_{2,0}^{0} + H_{4,-2}^{0}H_{4,0}^{0},\\
  \mathsf{K}_{02} &= \mathsf{K}_{20} = \sum_{l,m}H_{l,-2}^{m}H_{l,2}^{m} = H_{0,-2}^{0}H_{0,2}^{0} + H_{2,-2}^{0}H_{2,2}^{0} + H_{4,-2}^{0}H_{4,2}^{0},\\
  \mathsf{K}_{12} &= \mathsf{K}_{21} = \sum_{l,m}H_{l,0}^{m}H_{l,2}^{m} = H_{0,0}^{0}H_{0,2}^{0} + H_{2,0}^{0}H_{2,2}^{0} + H_{4,0}^{0}H_{4,2}^{0}.
\end{align}
The eigenvalues and eigenvectors can be computed in a closed form (and I've done
this using a symbolic package), but writing the result would fill several pages
with notation and provide little insight. Instead we proceed straight to plots
of the singular system shown in Fig. \ref{fig:detsvs}.

\fig{../calculations/out/SVSdet.pdf}{1.0}{Singular system for a single-view
  polarized detection microscope with 0.8 NA epi-illumination and
  epi-detection. \textbf{Rows:} Discrete branches of the singular value system
  indexed by $j$. \textbf{Column 1:} Continuous singular value spectra for each
  branch $j$ indexed by spatial frequency $\bs{\rho}$. The singular value
  spectra are normalized with contour lines at 0 and 0.1. \textbf{Columns 2-5:}
  Angular part of the object-space singular vector at spatial frequencies marked
  with `x's in the first column. The camera is facing the origin along the
  [1,1,1] axis with the optical axis of the microscope pointing up along the
  page. A \textcolor{red}{red} surface indicates positive values, a
  \textcolor{blue}{blue} surface indicates negative values, and the distance
  from the origin indicates the magnitude.}{detsvs}

First, notice that the DC term ($\bs{\rho} = 0$, $j=0$) of the polarized
detection singular system in Fig. \ref{fig:detsvs} is identical to the DC term
of the polarized illumination singular system in FIg. \ref{fig:illsvs}. This is
a result we've seen before---if we ignore spatio-angular coupling then polarized
illumination and polarized detection microscopes measure identical components of
the object.

Second, notice that there are no longer any degeneracies between the three
branches of the singular spectrum (except at $\bs{\rho} = 0$). This means that
moving the polarizer to the detection arm breaks a symmetry. Rotating each
individual point in object space no longer has a corresponding transformation in
data space because the polarizer introduces spatio-angular coupling. Each of the
branches of the singular spectrum is still rotationally symmetric, though, so we
preserve one of the symmetries---rotating the object about the optical axis will
still rotate the data space about the curved axis that runs through the center
of donut (the ``glaze inversion''!).

Third, notice that the singular values of the polarized detection microscope are
larger than the singular values of the polarized illumination microscope at all
spatial frequencies except $\bs{\rho} = 0$ where they are equal. This means that
a polarized detection microscope passes orthogonal components of the object more
efficiently (the signal is less corruptible by noise) than a polarized
illumination microscope. Although the difference is not extremely large---3-4\%
at most---this is the most convincing reason that we should prefer polarized
detection over polarized illumination. 

Finally, consider the singular functions. The singular functions are no longer
rotationally symmetric at non-zero spatial frequencies which means that we would
see rotated singular functions if we showed the singular functions for spatial
frequencies off of the $x$-axis. We also see that the singular functions contain
out-of-plane components at high spatial frequencies (espsecially in the $j=1$
branch) just like the the polarized illumination case.



\bibliography{report}{}
\bibliographystyle{unsrt}


\appendix
\section{Calculating products of circular and spherical harmonics}
To calculate the kernels of arbitrary microscope designs we need an efficient
way to calculate the products of functions that are linear combinations of
circular and spherical harmonics. We start by considering the simplest case of
multiplying two linear combinations of circular harmonics given by
\begin{align}
  f(\mh{p}) = \sum_{n=0}^\infty c_nz_n(\mh{p}), \qquad f'(\mh{p}) = \sum_{n'=0}^\infty c_{n'}'z_{n'}(\mh{p}). 
\end{align}
The product of these two functions is given by 
\begin{align}
  f''(\mh{p}) &= f(\mh{p})f'(\mh{p})=\sum_{n=0}^\infty \sum_{n'=0}^\infty c_n c_{n'}' z_n(\mh{p}) z_{n'}(\mh{p}) = \sum_{n''=0}^{\infty} c_{n''}''z_{n''}(\mh{p}), \label{eq:two}
  \end{align}
  and we would like to find a relationship between the input coefficients ($c_n$
  $c_{n'}'$) and the output coefficients $c_{n''}''$. To find this relationship
  we can write the product the circular harmonics as a linear combination of
  circular harmonics
  \begin{align}
    z_n(\mh{p})z_{n'}(\mh{p}) = \sum_{j''=0}^{\infty}P_{n,n',n''}z_{j''}(\mh{p})\label{eq:product}
  \end{align}
  where $P^{n''}_{n,n'}$ are the triple integrals of the circular harmonics
  \begin{align}
    P_{n,n',n''} = \int_{\mbb{S}^1}d\mh{p}\, z_{n}(\mh{p})z_{n'}(\mh{p})z_{n''}(\mh{p}).\label{eq:triple}
  \end{align}
  Plugging Eq. \ref{eq:product} into Eq. \ref{eq:two} gives
  \begin{align}
    \sum_{n=0}^\infty \sum_{n'=0}^\infty c_n c_{n'}' \left[\sum_{j''=0}^{\infty}P^{n''}_{n,n'}z_{j''}(\mh{p})\right] = \sum_{n''=0}^{\infty} c_{n''}''z_{n''}(\mh{p}).    
  \end{align}
  Therefore, the coefficients are related by
  \begin{align}
    c_{n''}'' = \sum_{n=0}^\infty \sum_{n'=0}^\infty P_{n,n',n''} c_n c_{n'}'.
  \end{align}
  Rewriting this equation in Einstein notation (summation over matching indices
  is implied, upper indices are ``column'' indices, lower indices are ``row''
  indices) gives
  \begin{align}
    {c''}^{n''} = P^{n''}_{n,n'} c^n {c'}^{n'}.
  \end{align}
  We can precompute the triple integrals $P_{n,n',n''}$ using a symbolic package
  like Sympy.
  
  The discussion above applies to spherical harmonics as well---we only need to replace
  the triple integral of the circular harmonics $P_{n,n',n''}$ with the triple integral
  of the spherical harmonics
  \begin{align}
    G_{j,j',j''} = \int_{\mbb{S}^2}d\mh{s}\, y_{j}(\mh{s})y_{j'}(\mh{s})y_{j''}(\mh{s}).
  \end{align}
  where each of the $j$ indices are a single index over the spherical harmonics.
  We could compute the triple integrals symbolically, but these integrals can
  take several minutes. Instead, we can write the integral in terms of the Gaunt
  coefficients \cite{homeier1996} which are products of the Clebsch-Gordan
  coefficients or Wigner 3-j symbols. The Gaunt coefficients have a closed form
  expression in terms of $j,j'$, and $j''$ (usually expressed in terms of
  $l,l',l'',m,m',m''$) that is implemented in the Sympy library \cite{sympy}.

  To calculate the kernel for general polarized light microscopes we will need to multiply functions in the following form
  \begin{align}
  f(\mh{p}, \mh{s}) = \sum_{n=0}^\infty\sum_{j=0}^\infty c_{n,j} z_n(\mh{p})y_j(\mh{s}), \qquad f'(\mh{p}, \mh{s}) = \sum_{n'=0}^\infty\sum_{j'=0}^\infty c_{n',j'}' z_n(\mh{p})y_j(\mh{s}).
\end{align}
The product will be in the form
\begin{align}
  f(\mh{p}, \mh{s})f'(\mh{p}, \mh{s}) = \sum_{n=0}^\infty\sum_{j=0}^\infty c_{n'',j''}'' z_{n''}(\mh{p})y_{j''}(\mh{s}),
\end{align}
where
\begin{align}
  {c''}^{n'', j''} = P_{n,n'}^{n''} G_{j,j'}^{j''} c^{n,j} {c'}^{n', j'}.\label{eq:final}
\end{align}
Equation \ref{eq:final} is the main result of this section. It shows that we can
precalculate the triple integrals of the circular and spherical harmonics and
use the results to efficiently find the coefficients of the product of two
arbitrary kernels. We can think of Eq. \ref{eq:final} as a bilinear map that
acts within the vector space of harmonic function coefficients. The bilinear map
takes two elements of the vector space and maps them to another element of the
vector space by a rank-6 tensor product. A lower dimensional example of a
bilinear map is the cross product which takes two vectors in three-dimensional
Euclidean space and maps them to another vector by a rank-3 tensor product.
\end{document}

