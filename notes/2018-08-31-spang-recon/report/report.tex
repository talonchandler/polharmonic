\documentclass[11pt]{article}

%%%%%%%%%%%%
% Packages %
%%%%%%%%%%%%
\usepackage[dvipsnames]{xcolor}
\hyphenpenalty=10000
\usepackage{tikz}
\usetikzlibrary{shapes,arrows}

\usepackage{tocloft}
\usepackage[linesnumbered,ruled]{algorithm2e}
\renewcommand\cftsecleader{\cftdotfill{\cftdotsep}}
\def\undertilde#1{\mathord{\vtop{\ialign{##\crcr
$\hfil\displaystyle{#1}\hfil$\crcr\noalign{\kern1.5pt\nointerlineskip}
$\hfil\tilde{}\hfil$\crcr\noalign{\kern1.5pt}}}}}
\usepackage{bold-extra}
\usepackage{bm}
\usepackage{cleveref}
\usepackage{xcolor}
\usepackage[colorlinks = true,
            linkcolor = black,
            urlcolor  = blue,
            citecolor = black,
            anchorcolor = black]{hyperref}
\usepackage{epstopdf}
\usepackage{braket}
\usepackage{upgreek}
\usepackage{caption}
\usepackage{booktabs}
\usepackage{subcaption}
\usepackage{amssymb,latexsym,amsmath,gensymb}
\usepackage{latexsym}
\usepackage{graphicx}
\usepackage{float}
\usepackage{enumitem}
\usepackage{pdflscape}
\usepackage{url}
\usepackage{array}
\newcolumntype{C}{>{$\displaystyle} c <{$}}
\usepackage{tikz, calc}
\usetikzlibrary{shapes.geometric, arrows, calc}
\tikzstyle{norm} = [rectangle, rounded corners, minimum width=2cm, minimum height=1cm,text centered, draw=black]
\tikzstyle{arrow} = [thick, ->, >=stealth]

\newcommand{\argmin}{\arg\!\min}
\newcommand{\argmax}{\arg\!\max}
\newcommand{\me}{\mathrm{e}}
\providecommand{\e}[1]{\ensuremath{\times 10^{#1}}}
\providecommand{\mb}[1]{\mathbf{#1}}
\providecommand{\mc}[1]{\mathcal{#1}}
\providecommand{\ro}{\mathbf{\mathfrak{r}}_o}
\providecommand{\so}{\mathbf{\hat{s}}_o}
\providecommand{\rb}{\mathbf{r}_b}
\providecommand{\rbm}{r_b^{\text{m}}}
\providecommand{\rd}{\mathbf{\mathfrak{r}}_d}
\providecommand{\mh}[1]{\mathbf{\hat{#1}}}
\providecommand{\mf}[1]{\mathfrak{#1}}
\providecommand{\ms}[1]{\mathsf{#1}}
\providecommand{\mbb}[1]{\mathbb{#1}}
\providecommand{\bs}[1]{\boldsymbol{#1}}
\providecommand{\tv}{\texttt{v}}
\providecommand{\tx}[1]{\text{#1}}
\providecommand{\tb}[1]{\textbf{#1}}
\providecommand{\ttt}[1]{\texttt{#1}}
\providecommand{\bv}{\bs{\nu}}
\providecommand{\bp}{\bs{\rho}}
\providecommand{\p}{\mh{p}}
\providecommand{\lmsum}{\sum_{l=0}^\infty\sum_{m=-l}^{l}}
\providecommand{\intr}[1]{\int_{\mbb{R}^{#1}}}
\providecommand{\ints}[1]{\int_{\mbb{S}^{#1}}}
\providecommand{\intinf}{\int_{-\infty}^{\infty}}

\providecommand{\fig}[4]{
  % filename, width, caption, label
\begin{figure}[H]
 \captionsetup{width=1.0\linewidth}
 \centering
 \includegraphics[width = #2\textwidth]{#1}
 \caption{#3}
 \label{fig:#4}
\end{figure}
}

\makeatletter
\renewcommand*\env@matrix[1][*\c@MaxMatrixCols c]{%
  \hskip -\arraycolsep
  \let\@ifnextchar\new@ifnextchar
  \array{#1}}
\makeatother

\newcommand{\tensor}[1]{\overset{\text{\tiny$\leftrightarrow$}}{\mb{#1}}}
\newcommand{\tunderbrace}[2]{\underbrace{#1}_{\textstyle#2}}
\providecommand{\figs}[7]{
  % filename1, filename2, caption1, caption2, label1, label2, shift
\begin{figure}[H]
\centering
\begin{minipage}[b]{.45\textwidth}
  \centering
  \includegraphics[width=1.0\linewidth]{#1}
  \captionsetup{justification=justified, singlelinecheck=true}
  \caption{#3}
  \label{fig:#5}
\end{minipage}
\hspace{2em}
\begin{minipage}[b]{.45\textwidth}
  \centering
  \includegraphics[width=1.0\linewidth]{#2}
  \vspace{#7em}
  \captionsetup{justification=justified}
  \caption{#4}
  \label{fig:#6}
\end{minipage}
\end{figure}
}
\makeatletter

\providecommand{\code}[1]{
\begin{center}
\lstinputlisting{#1}
\end{center}
}

\newcommand{\crefrangeconjunction}{--}
%%%%%%%%%%%
% Spacing %
%%%%%%%%%%%
% Margins
\usepackage[
top    = 1.5cm,
bottom = 1.5cm,
left   = 1.5cm,
right  = 1.5cm]{geometry}

% Indents, paragraph space
%\usepackage{parskip}
\setlength{\parskip}{1.5ex}

% Section spacing
\usepackage{titlesec}
\titlespacing*{\title}
{0pt}{0ex}{0ex}
\titlespacing*{\section}
{0pt}{0ex}{0ex}
\titlespacing*{\subsection}
{0pt}{0ex}{0ex}
\titlespacing*{\subsubsection}
{0pt}{0ex}{0ex}

% Line spacing
\linespread{1.1}

%%%%%%%%%%%%
% Document %
%%%%%%%%%%%%
\begin{document}
\title{\vspace{-2.5em} Spatio-angular restoration of polarized diSPIM data:\\
  analytical, numerical, and experimental results
  \vspace{-1em}} % \author{Talon Chandler, Min Guo, Hari
  % Shroff, Rudolf Oldenbourg, Patrick La Rivi\`ere}
\date{\vspace{-3em}\today\vspace{-1em}}
\maketitle
\section{Introduction}
These notes detail the spatio-angular restoration pipeline for data collected
with the polarized diSPIM. In Section \ref{sec:fwd} we explain our forward
model---the relationship between an arbitrary spatio-angular density and the
data we collect with the polarized diSPIM. We will focus on our discrete
implementation and the computational issues that we faced. In Section
\ref{sec:inv} we will explain our inversion scheme---the set of computations
that allow us to recover a spatio-angular density from noise-corrupted data. In
Sections \ref{sec:num} and \ref{sec:exp} we will show numerical and experimental
results.

\section{Forward model} \label{sec:fwd}
\subsection{Continuous-to-continuous model} \label{sec:cc}
The objects we are attempting to simulate and reconstruct are spatio-angular
densities of fluorophores $f(\ro, \so)$---the number of fluorophores at each 3D
position $\ro$ oriented along a direction $\so$. Therefore, the complete object
space $\mbb{U}$ is the function space $\mbb{L}_2(\mbb{R}^3 \times \mbb{S}^2)$.
The data that we sample from each arm of the polarized diSPIM is
$g(\rd, \p)$---the irradiance measured at each 3D detector position $\rd$ with
the illumination polarizer oriented along $\p$. The data space for each view is
a member of the function space $\mbb{L}_2(\mbb{R}^3 \times \mbb{S}^1)$. We have
two views, so the complete data space $\mbb{V}$ is the function space
$[\mbb{L}_2(\mbb{R}^3 \times \mbb{S}^1)]^2$. Although this section is named
continuous-to-continuous (CC in Barrett's notation), strictly speaking this is
already a CC-CD model because we have a discrete number of views.

In previous note sets we have shown that we can write the CC model for the
polarized diSPIM as
\begin{align}
  g_{\tv}(\rd, \p) = \intr{3}d\ro \ints{2}d\so\, h_{\tv}(\rd - \ro, \p, \so)f(\ro, \so), \label{eq:fwd}
\end{align}
where $\tv \in \{\text{A}, \text{B}\}$ is the light path index, and
$h_{\tv}(\rd - \ro, \p, \so)$ is the kernel of the integral transform that maps
objects space to data space. The imaging process consists of incoherent
excitation and detection steps, so we can write the kernel as a product of an
excitation and detection kernel
\begin{align}
  h_{\tv}(\ro, \so; \mh{p}) &= h_{\tv, \tx{exc}}(\ro, \so; \mh{p})h_{\tv, \tx{det}}(\ro, \so).\label{eq:split}
\end{align}
Notice that we are considering polarized illumination, so the polarizer
coordinate $\p$ is a variable in the excitation kernel only.

The excitation kernels for each view are given by
\begin{align}
  h_{\tx{B}, \tx{exc}}(\ro, \so; \mh{p}) &= \left[z_{0}(\mh{p})y_0^0(\so) + \sqrt{\frac{3}{5}}z_{-2}(\mh{p})y_2^{-2}(\so) - \frac{1}{\sqrt{5}}z_{0}(\mh{p})y_2^0(\so) + \sqrt{\frac{3}{5}}z_{2}(\mh{p})y_2^{2}(\so)\right]\mf{g}(r_x, \sigma_{\tx{ls}}),\\
  h_{\tx{A}, \tx{exc}}(\ro, \so; \mh{p}) &= \mc{R}_{xz}\left\{h_{\tx{B}, \tx{exc}}(\ro, \so; \mh{p})\right\},
\end{align}
where $z_n(\p)$ are circular harmonic functions; $y_l^m(\so)$ are spherical
harmonic functions; $\mf{g}(\mu, \sigma)$ is a Gaussian function with mean $\mu$
and standard deviation $\sigma$; and $\mc{R}_{xz}$ is a rotation operator that
swaps the $x$ and $z$ coordinates within the function. Note that the operator
$\mc{R}_{xz}$ applies a rotation transformation to the circular and spherical
harmonics as well.

The detection kernels are given by
\begin{align}
  h_{\tx{A}, \tx{det}}(\ro, \so) &= \left\{\left[a_{1,\tx{A}}(r_{xy}) + \frac{\alpha_{\tx{A}}^2}{4}a_{2,\tx{A}}(r_{xy})\right]y_0^0(\so) + \frac{1}{\sqrt{5}}\left[-a_{1,\tx{A}}(r_{xy}) + \frac{\alpha_{\tx{A}}^2}{2}a_{2,\tx{A}}(r_{xy})\right]y_2^0(\so)\right\}\mathfrak{g}(r_z, \sigma_{\tx{det}}),\\
  h_{\tx{B}, \tx{det}}(\ro, \so) &= \mc{R}_{xz}\left\{h_{\tx{A}, \tx{det}}(\ro, \so)\right\}.
\end{align}
where
\begin{align}
  a_{n,\tv}(r) &\equiv \frac{n}{\pi}\left[\frac{J_n(2\pi\nu_{o,\tv} r)}{2\pi\nu_{o,\tv} r}\right],
           \intertext{and}
  r_{xy} \equiv \sqrt{r_x^2 + r_y^2},\hspace{2em}
  \nu_{o,\tv} &\equiv \frac{\tx{NA}_{\tv}}{\lambda},\hspace{2em}
  \alpha_{\tv} \equiv \frac{\tx{NA}_{\tv}}{n_0}. \label{eq:alpha}
\end{align}
Notice that we have added a subscript $\tv$ to NA, $\alpha$, and $\nu_o$ to allow for the possibility of an asymmetric diSPIM. 

Eqs. \ref{eq:fwd}--\ref{eq:alpha} are sufficient to simulate the forward model
of a polarized diSPIM, but the integrals in Eq. \ref{eq:fwd} are extremely
expensive to calculate. If we naively discretize $f(\ro, \so)$ into
$N = 1000 \times 1000 \times 1000 \times 1000 = 10^{12}$ spatio-angular points
and discretize the data into
$M = 1000 \times 1000 \times 1000 \times 4\ \text{polarizers} \times 2\
\text{views} = 8\e{9}$ data points (a realistic data set), then simulating the
forward model (assuming the kernel is inexpensive to evaluate) would require
$\mc{O}(NM) \approx 8\e{21}$ floating-point operations (FLOPs)---not realistic
on a typical $10^{11}$ floating-point operations per second (FLOPS) machine (a
very good laptop). Aside: there's a cute mnemonic for converting large numbers
of seconds---$\pi$ seconds is approximately one nanocentury or
$\pi\e{7}\ \text{s} \approx 1\ \text{y}$. Luckily we can evaluate the forward
model much more quickly by exploiting the symmetries and rank of the integral
operator.

The first symmetry we'll exploit is shift invariance. We notice that the
spatial integral is in the form of a convolution, so we apply the
convolution theorem and rewrite Eq. \ref{eq:fwd} as
\begin{align}
  g_{\tv}(\rd, \p) = \mc{F}^{-1}_{\mbb{R}^3}\left\{\ints{2}d\so\, \mc{F}_{\mbb{R}^3}\left\{h_{\tv}(\ro, \mh{p}, \so)\right\} \mc{F}_{\mbb{R}^3}\left\{f(\ro, \so)\right\}\right\}, \label{eq:filter1}
\end{align}
where $\mc{F}_{\mbb{R}^3}$ denotes a three-dimensional Fourier transform given by
\begin{align}
  \mc{F}_{\mbb{R}^3}\left\{f_{\tv}(\ro, \so)\right\} &\equiv \intr{3}d\ro\, \me^{i2\pi\ro\cdot\bv} f_{\tv}(\ro, \so).
\end{align}
Eq. \ref{eq:filter1} is already much easier to compute than Eq. \ref{eq:fwd},
but it still requires us to evaluate a spherical integral with two 3D Fourier
transforms in the integrand. We can simplify this integral by exploiting the
angular rank of the kernel. First, we apply the following identity
\begin{align}
  \ints{2}d\mh{s}{}\, f(\mh{s})g(\mh{s}) = \lmsum \mc{F}_{\mbb{S}^2}\left\{f(\mh{s})\right\}\mc{F}_{\mbb{S}^2}\left\{g(\mh{s})\right\}, \label{eq:plan}
\end{align}
where
\begin{align}
  \mc{F}_{\mbb{S}^2}\left\{f(\mh{s})\right\} \equiv \int_{\mbb{S}^2}f(\mh{s})y_l^m(\mh{s}).
\end{align}
Eq. \ref{eq:plan} is the generalized Plancharel theorem for spherical functions.
It is a special case of the fact that scalar products are invariant under
unitary transformations (see Barrett 3.78). Next, we plug Eq. \ref{eq:plan}
into Eq. \ref{eq:filter1} to get
\begin{align}
  g_{\tv}(\rd, \p) = \mc{F}^{-1}_{\mbb{R}^3}\left\{\lmsum H_{l,\tv}^m(\bv, \mh{p}) \mc{F}_{\mbb{R}^3\times\mbb{S}^2}\left\{f(\ro, \so)\right\}\right\} \label{eq:filter}
\end{align}
where
\begin{align}
  H_{l,\tv}^m(\bv, \mh{p}) \equiv \mc{F}_{\mbb{R}^3\times\mbb{S}^2}\left\{h_{\tv}(\ro, \p, \so)\right\} &\equiv \intr{3}d\ro\, \me^{i2\pi\ro\cdot\bv}\ints{2}d\so\, y_l^m(\so) h_{\tv}(\ro, \p, \so).\label{eq:transfer}
\end{align}
Eq. \ref{eq:filter} is the main result of this section. Right now it is not
obvious that Eq. \ref{eq:filter} is any more efficient than Eq.
\ref{eq:filter1}, but when we calculate the transfer function we will see that
the sum over spherical harmonics contains a small number of terms.

We compute the transfer function $H_{l,\tv}^m(\bv, \mh{p})$ by plugging Eq.
\ref{eq:split} into Eq. \ref{eq:transfer}
\begin{align}
  H_{l,\tv}^m(\bv, \mh{p}) = \mc{F}_{\mbb{R}^3\times\mbb{S}^2}\left\{h_{\tv, \tx{exc}}(\ro, \so; \mh{p})h_{\tv, \tx{det}}(\ro, \so)\right\}.
\end{align}
We can factor the kernel for both views into a transverse and axial part (a Gaussian) so
\begin{align}
  H_{l,\tv}^m(\bv, \mh{p}) &= \mc{F}_{\mbb{R}^3\times\mbb{S}^2}\left\{h^{\text{tr}}_{\tv, \tx{exc}}(\so; \mh{p})h^{\text{tr}}_{\tv, \tx{det}}(\ro, \so)\mf{g}(0, \sigma_{\text{ax}})\right\}.
\end{align}
Rearranging the Fourier transforms gives
\begin{align}
  H_{l,\tv}^m(\bv, \mh{p}) &= \mc{F}_{\mbb{S}^2}\left\{h^{\text{tr}}_{\tv, \tx{exc}}(\so; \mh{p})\mc{F}_{\mbb{R}^2}\left\{h^{\text{tr}}_{\tv, \tx{det}}(\ro, \so)\right\}\mc{F}_{\mbb{R}^1}\left\{\mf{g}(0, \sigma_{\text{ax}})\right\}\right\}.\label{eq:rearr}
\end{align}
The Fourier transform of a Gaussian is a Gaussian with inverse standard
deviation, and in previous notes we evaluated the Fourier transform of the
transverse detection kernel as
\begin{align}
  \mc{F}_{\mbb{R}^2}\left\{h^{\text{tr}}_{\tx{A}, \tx{det}}(\ro, \so)\right\} &= \left[A_{1,\tx{A}}(\nu_{xy}) + \frac{\alpha_{\tx{A}}^2}{4}A_{2,\tx{A}}(\nu_{xy})\right]y_0^0(\so) + \frac{1}{\sqrt{5}}\left[-A_{1,\tx{A}}(\nu_{xy}) + \frac{\alpha_{\tx{A}}^2}{2}A_{2,\tx{A}}(\nu_{xy})\right]y_2^0(\so),
\end{align}
where
\begin{align}
  A_{1,\tx{A}}(\nu_{xy}) &= \frac{2}{\pi}\left\{\cos^{-1}\left(\frac{\nu_{xy}}{2\nu_{o,\tx{A}}}\right) - \frac{\nu_{xy}}{2\nu_{o,\tx{A}}}\sqrt{1 - \left(\frac{\nu_{xy}}{2\nu_{o,\tx{A}}}\right)^2}\right\}\Pi\left(\frac{\nu_{xy}}{2\nu_{o,\tx{A}}}\right),\\
  A_{2,\tx{A}}(\nu_{xy}) &= \frac{2}{\pi}\Bigg\{\cos^{-1}\left(\frac{\nu_{xy}}{2\nu_{o,\tx{A}}}\right) - \left[3 - 2\left(\frac{\nu_{xy}}{2\nu_{o,\tx{A}}}\right)^2\right]\frac{\nu_{xy}}{2\nu_{o,\tx{A}}}\sqrt{1 - \left(\frac{\nu_{xy}}{2\nu_{o,\tx{A}}}\right)^2}\Bigg\}\Pi\left(\frac{\nu_{xy}}{2\nu_{o,\tx{A}}}\right).
\end{align}
We have closed form solutions for the inner Fourier transforms of Eq.
\ref{eq:rearr}, but we still need to evaluate the outer spherical Fourier
transform. The function is the product of spherical Fourier transforms, so we
need to take a spherical convolution of the coefficients of the spherical
Fourier transforms of the individual functions. We could do this by hand and
write out the result, but the equations would be long and wouldn't give much
insight. Instead, we've implemented a spherical convolution that uses the Gaunt
coefficients to efficiently compute the transfer function. The most important
result is that we can compute $H_{l,\tv}^m(\bv, \mh{p})$ directly without
evaluating expensive Bessel functions then taking the Fourier transform.

We notice that both the excitation and detection kernels contain just a few
spherical harmonic functions (four for the excitation kernel, two for the
detection kernel), and these terms are all in the $l=0$ and $l=2$ bands. When we
multiply out the spherical harmonics we get a total of 9 terms in the $l=0$,
$l=2$, and $l=4$ bands. This means that the sum over spherical harmonics in Eq.
\ref{eq:filter} contains only 9 terms which is an enormous computational savings
over the complete spherical integral.

\subsection{Continuous-to-discrete model}
To convert the CC model to a CD model we weight Eq. \ref{eq:fwd} by a sampling
aperture $w_{\tb{\ttt{r}}_{\ttt{d}}, \ttt{p}}(\rd, \p)$ and integrate over the
continuous detector coordinates $\rd$ and $\p$ to obtain
\begin{align}
  g_{\tb{\ttt{r}}_{\ttt{d}},\ttt{p},\ttt{v}} = \intr{3}d\rd\ints{1}d\p\, w_{\tb{\ttt{r}}_{\ttt{d}}, \ttt{p}}(\rd, \p)\left[\intr{3}d\ro \ints{2}d\so\, h_{\tv}(\rd - \ro, \p, \so)f(\ro, \so)\right]. \label{eq:cd}
\end{align}
Notice that we use monospace type for discrete (integer) indices---$\ttt{p}$ is the
polarizer setting index, $\ttt{v}$ is the view index, and
$\tb{\ttt{r}}_{\ttt{d}}$ is the voxel multi-index.

We will rearrange the integrals and group terms in Eq. \ref{eq:cd} to find 
\begin{align}
  g_{\tb{\ttt{r}}_{\ttt{d}},\ttt{p},\ttt{v}} = \intr{3}d\ro \ints{2}d\so\, h_{\tb{\ttt{r}}_{\ttt{d}}, \ttt{p}, \ttt{v}}(\ro, \so) f(\ro, \so), \label{eq:cd2}
\end{align}
where the new CD kernel is related to the CC kernel by
\begin{align}
   h_{\tb{\ttt{r}}_{\ttt{d}}, \ttt{p}, \ttt{v}}(\ro,\so) = \intr{3}d\rd\ints{1}d\p\, w_{\tb{\ttt{r}}_{\ttt{d}}, \ttt{p}}(\rd, \p) h_{\tv}(\rd - \ro, \p, \so).
\end{align}

We can interpret the sampling aperture
$w_{\tb{\ttt{r}}_{\ttt{d}}, \ttt{p}}(\rd, \p)$ as the sensitivity of the
$(\tb{\ttt{r}}_{\ttt{d}}, \ttt{p})\text{th}$ measurement to the irradiance at
position $(\rd, \p)$. The simplest choice for the sampling aperture is 
\begin{align}
  w_{\tb{\ttt{r}}_{\ttt{d}}, \ttt{p}}(\rd, \p) = \delta(\rd - \mb{\epsilon}\tb{\ttt{r}}_{\ttt{d}})\delta(\mh{p} - \mb{\epsilon}_{\phi}\ttt{p}), \label{eq:sampling}
\end{align}
where $\mb{\epsilon}$ is the distance between voxels and $\mb{\epsilon}_{\phi}$
is the angle between polarizer settings. With this choice of sampling aperture the
CD forward model in Eq. \ref{eq:cd2} reduces to 
\begin{align}
  g_{\tb{\ttt{r}}_{\ttt{d}},\ttt{p},\ttt{v}} = \intr{3}d\ro \ints{2}d\so\, h_{\tv}(\mb{\epsilon}\tb{\ttt{r}}_{\ttt{d}} - \ro, \mb{\epsilon}_{\phi}\ttt{p}, \so)f(\ro, \so). 
\end{align}
Notice that the CD kernel is a sampled version of the CC kernel, but this is
only true when the sampling aperture is given by Eq. \ref{eq:sampling}.

A more realistic model of the sampling aperture is given by
\begin{align}
  w_{\tb{\ttt{r}}_{\ttt{d}}, \ttt{p}}(\rd, \p) = \frac{1}{\epsilon^3}\text{rect}\left(\frac{\rd - \mb{\epsilon}\tb{\ttt{r}}_{\ttt{d}}}{\mb{\epsilon}}\right)\delta(\mh{p} - \mb{\epsilon}_{\phi}\ttt{p}),
\end{align}
where the rect function models the aperture of each pixel. Finally, we mention
the possibility of modeling a rotating polarizer by using the following sampling
aperture
\begin{align}
  w_{\tb{\ttt{r}}_{\ttt{d}}, \ttt{p}}(\rd, \p) = \frac{1}{\epsilon^3\epsilon_{\phi}}\text{rect}\left(\frac{\rd - \mb{\epsilon}\tb{\ttt{r}}_{\ttt{d}}}{\mb{\epsilon}}\right)\text{rect}\left(\frac{\mh{p} - \mb{\epsilon}_{\phi}\ttt{p}}{\mb{\epsilon}_\phi}\right).
\end{align}
We might consider these sampling aperture in the future, but for now we will use
the sampling aperture in Eq. \ref{eq:sampling} and move forward with the CD model in Eq. \ref{eq:cd}. 

\subsection{Discrete-to-discrete model}
To convert the CD model to a DD model we need to expand our object-space
functions $f(\ro, \so)$ onto a discrete set of expansion functions with
\begin{align}
  f(\ro, \so) = \sum_{\ttt{\tb{r}}_{\ttt{o}},\ttt{j}} \theta_{\ttt{\tb{r}}_{\ttt{o}},\ttt{j}} \phi_{\ttt{\tb{r}}_{\ttt{o}},\ttt{j}}(\ro, \so).  \label{eq:exp}
\end{align}
We will choose our expansion functions as
\begin{align}
  \phi_{\ttt{\tb{r}}_{\ttt{o}},\ttt{j}}(\ro, \so) = \delta\left(\ro - \epsilon\ttt{\tb{r}}_{\ttt{o}}\right)y_{\ttt{j}}(\so), \label{eq:expansion}
\end{align}
where $\ttt{j}$ is a single index over the spherical harmonics. This choice of
basis function may seem odd---we are representing the spatial part of our object
in a sampled voxel basis and the angular part in a spherical harmonic basis. The
main reason for this choice is that the diSPIM has a small angular rank, so we
can't reconstruct more than a few angular coefficients anyway. By expanding our
object in a spherical harmonic basis, we significantly reduce the storage and
computation requirements for our simulations and reconstructions without any
loss of accuracy.

The expansion functions in Eq. \ref{eq:expansion} are orthonormal, but they do
not form a complete basis for $\mbb{L}_2(\mbb{R}^3\times \mbb{S}^2)$. A good
choice for a complete orthonormal basis is
\begin{align}
  \phi_{\ttt{\tb{r}}_{\ttt{o}},\ttt{j}}(\ro, \so) = \frac{1}{\epsilon^3}\text{rect}\left(\frac{\ro - \epsilon\ttt{\tb{r}}_{\ttt{o}}}{\epsilon}\right)y_{\ttt{j}}(\so), 
\end{align}
but we will use the basis in Eq. \ref{eq:expansion} for now to simplify our
model.

For a given object $f(\ro, \so)$, we can find the expansion coefficients using
\begin{align}
  \theta_{\ttt{\tb{r}}_{\ttt{o}},\ttt{j}} = \intr{3}d\ro\ints{2}d\so\, \phi_{\ttt{\tb{r}}_{\ttt{o}},\ttt{j}}(\ro, \so)f(\ro, \so). \label{eq:theta}
\end{align}
Plugging Eq. \ref{eq:expansion} into Eq. \ref{eq:theta} gives 
\begin{align}
  \theta_{\ttt{\tb{r}}_{\ttt{o}},\ttt{j}} = \ints{2}d\so\, f(\ro - \epsilon\ttt{\tb{r}}_{\ttt{o}},\so)y_{\ttt{j}}(\so) = \mc{F}_{\mbb{S}^2}\left\{f(\ro - \epsilon\ttt{\tb{r}}_{\ttt{o}},\so)\right\} \label{eq:f2sh}
\end{align}
which shows us the procedure to find the expansion coefficients of an
object---sample the object on a spatial grid then take the spherical Fourier
transform.

If we plug Eq. \ref{eq:exp} into Eq. \ref{eq:cd} we find
\begin{align}
  g_{\tb{\ttt{r}}_{\ttt{d}},\ttt{p},\ttt{v}} = \intr{3}d\ro \ints{2}d\so\, h_{\tv}(\mb{\epsilon}\tb{\ttt{r}}_{\ttt{d}} - \ro, \mb{\epsilon}_{\phi}\ttt{p}, \so)\left[\sum_{\ttt{\tb{r}}_{\ttt{o}},\ttt{j}} \theta_{\ttt{\tb{r}}_{\ttt{o}},\ttt{j}} \phi_{\ttt{\tb{r}}_{\ttt{o}},\ttt{j}}(\ro, \so)\right].
\end{align}
After plugging in the expansion functions in Eq. \ref{eq:expansion}, evaluating
the spatial integral, and rearranging terms we find
\begin{align}
  g_{\tb{\ttt{r}}_{\ttt{d}},\ttt{p},\ttt{v}} = \sum_{\ttt{\tb{r}}_{\ttt{o}},\ttt{j}}\ints{2}d\so\, h_{\tv}(\mb{\epsilon}[\tb{\ttt{r}}_{\ttt{d}} - \tb{\ttt{r}}_{\ttt{o}}], \mb{\epsilon}_{\phi}\ttt{p}, \so)y_{\ttt{j}}(\so)\theta_{\ttt{\tb{r}}_{\ttt{o}},\ttt{j}}.
\end{align}
We can rewrite this equation as a DD model
\begin{align}
  g_{\tb{\ttt{r}}_{\ttt{d}},\ttt{p},\ttt{v}} = \sum_{\ttt{\tb{r}}_{\ttt{o}},\ttt{j}}h_{\ttt{\tb{r}}_{\ttt{d}} - \ttt{\tb{r}}_{\ttt{o}},\ttt{p},\ttt{v},\ttt{j}}\theta_{\ttt{\tb{r}}_{\ttt{o}},\ttt{j}},\label{eq:dd}
\end{align}
where
\begin{align}
h_{\ttt{\tb{r}},\ttt{p},\ttt{v},\ttt{j}} = \ints{2}d\so\, h_{\tv}(\mb{\epsilon}\tb{\ttt{r}}, \mb{\epsilon}_{\phi}\ttt{p}, \so)y_{\ttt{j}}(\so).\label{eq:dd2}
\end{align}
Eqs. \ref{eq:dd} and \ref{eq:dd2} specify a discrete forward model, but they are
expensive to compute because of the sum over $\ttt{\tb{r}}$ and the integral
over $\mbb{S}^2$. Following the simplifications we made in Section \ref{sec:cc}, we
apply the discrete convolution multiplication theorem to rewrite the forward
model as
\begin{align}
  g_{\tb{\ttt{r}}_{\ttt{d}},\ttt{p},\ttt{v}} = \text{DFT}^{-1}\Big\{\sum_{\ttt{j}}H_{\bar{\bs{\nu}},\ttt{p},\ttt{v},\ttt{j}}\text{DFT}\left\{\theta_{\ttt{\tb{r}}_{\ttt{o}},\ttt{j}}\right\}\Big\}, \label{eq:fastfwd}
\end{align}
where
\begin{align}
  H_{\bar{\bs{\nu}},\ttt{p},\ttt{v},\ttt{j}} = \ints{2}d\so\, \text{DFT}\left\{h_{\tv}(\mb{\epsilon}\tb{\ttt{r}}, \mb{\epsilon}_{\phi}\ttt{p}, \so)\right\}y_{\ttt{j}}(\so) = H_{\ttt{v}, \ttt{j}}(\bar{\bs{\nu}}/\epsilon, \epsilon_{\phi}\ttt{p}), 
\end{align}
where the far LHS is the discrete spatio-angular transfer function and the far
RHS is the sampled continuous spatio-angular transfer function from Section 2.1. 

The forward model in Eq. \ref{eq:fastfwd} is extremely fast compared to the
naive discretization we considered previously. The 3D DFTs require approximately
$3N\log N$ FLOPs (about $10^4$ FLOPs for $N = 10^3$). We know that the sum over
$\ttt{j}$ has 15 terms, and we need to calculate these 15 DFTs for each of 4
polarizer settings and 2 views. Therefore, our approximate FLOP count is
$10^4 \times 15 \times 4 \times 2 = 1.2\e{6}$---about 15 orders of magnitude
faster than our naive approach.

\subsection{Implementation details}
For small input sizes we can precompute the entire array
$H_{\bar{\bs{\nu}},\ttt{p},\ttt{v},\ttt{j}}$ and apply Eq. \ref{eq:fastfwd}
directly. This array quickly becomes too large to store in memory
though---$10^9$ spatial points $\times$ 4 polarizers $\times$ 2 views $\times$
15 spherical harmonics $\times$ 32 bits = 480 GB. To avoid memory issues, we
separate $H_{\bar{\bs{\nu}},\ttt{p},\ttt{v},\ttt{j}}$ into transverse and axial
parts, precompute and store these small arrays, then assemble the complete array
in pieces during the application of the forward model. We also exploit the
8-fold spatial symmetry of the array
$H_{\bar{\bs{\nu}},\ttt{p},\ttt{v},\ttt{j}}$ and only store the positive octant.
Finally, the sum over $\ttt{j}$ needs to be computed at each point in
$\bar{\bs{\nu}}$, so we parallelize this loop over multiple CPUs.

\section{Inverse problem} \label{sec:inv} It will be convenient to rewrite the
forward model in Eq. \ref{eq:fastfwd} completely in the spatial frequency domain
as
\begin{align}
  G_{\bar{\bs{\nu}},\ttt{p},\ttt{v}} = \sum_{\ttt{j}}H_{\bar{\bs{\nu}},\ttt{p},\ttt{v},\ttt{j}}\Theta_{\bar{\bs{\nu}},\ttt{j}}, 
\end{align}
where
\begin{align}
  G_{\bar{\bs{\nu}},\ttt{p},\ttt{v}} = \text{DFT}\left\{g_{\tb{\ttt{r}}_{\ttt{d}},\ttt{p},\ttt{v}}\right\},\\
    \Theta_{\bar{\bs{\nu}},\ttt{j}} = \text{DFT}\left\{\theta_{\tb{\ttt{r}}_{\ttt{o}},\ttt{j}}\right\}.
\end{align}
If we combine the indices $\ttt{p}$ and $\ttt{v}$ into a single multi-index $\ttt{\tb{q}} = (\ttt{p}, \ttt{v})$ then the forward model becomes
\begin{align}
  G_{\bar{\bs{\nu}},\ttt{\tb{q}}} = \sum_{\ttt{j}}H_{\bar{\bs{\nu}},\ttt{\tb{q}},\ttt{j}}\Theta_{\bar{\bs{\nu}},\ttt{j}} \label{eq:matrix}
\end{align}
which has the form of a matrix multiplication at each discrete spatial frequency
$\bar{\bs{\nu}}$. We can rewrite Eq. \ref{eq:matrix} in matrix notation as
\begin{align}
  \mb{G}_{\bar{\bs{\nu}}} = \mb{H}_{\bar{\bs{\nu}}}\bs{\Theta}_{\bar{\bs{\nu}}}.  \label{eq:matrix2}
\end{align}
Taking the SVD of $\mb{H}_{\bar{\bs{\nu}}}$ gives
\begin{align}
  \mb{G}_{\bar{\bs{\nu}}} = \sum_{k=0}^R \sqrt{\mu_{\bar{\bs{\nu}}, k}}\textbf{v}_{\bar{\bs{\nu}}, k}\textbf{u}_{\bar{\bs{\nu}}, k}^{\dagger}\bs{\Theta}_{\bar{\bs{\nu}}}, \label{eq:svd}
\end{align}
where $\mu_{\bar{\bs{\nu}}, k}$, $\textbf{v}_{\bar{\bs{\nu}}, k}$,
$\textbf{u}_{\bar{\bs{\nu}}, k}$, and $R$ are the singular values, data space
singular vectors, object space singular vectors, and rank of
$\mb{H}_{\bar{\bs{\nu}}}$, respectively. Notice that Eq. \ref{eq:svd} gives us
an even more efficient way to compute the forward model---we only need to
compute the sum up to the rank instead of summing over the spherical harmonic
coefficients. Unfortunately, this trick is only feasible if we can store the
singular value decomposition of $\mb{H}_{\bar{\bs{\nu}}}$ in memory. In our case
we have to create $\mb{H}_{\bar{\bs{\nu}}}$ on the fly, and it would be too
expensive to calculate the SVD at each point on the fly.

We would like to estimate $\bs{\Theta}_{\bar{\bs{\nu}}}$ from noise-corrupted
measurements $\mb{G}_{\bar{\bs{\nu}}}$. We can rearrange Eq. \ref{eq:svd}
to find 
\begin{align}
  \bs{\hat{\Theta}}_{\bar{\bs{\nu}}} = \sum_{k=0}^R \frac{1}{\sqrt{\mu_{\bar{\bs{\nu}}, k}}}\textbf{u}_{\bar{\bs{\nu}}, k}\textbf{v}_{\bar{\bs{\nu}}, k}^{\dagger}\mb{G}_{\bar{\bs{\nu}}}. \label{eq:pinv}
\end{align}
where\, $\hat{\vspace{0.1em}}$\, denotes an estimate. This solution is called
the Moore-Penrose pseudoinverse and it gives a least-squares solution to any set
of linear equations. Notice that the sum in Eq. \ref{eq:pinv} is only up to the
rank $R$, so this solution avoids division by zero.

In practice, the pseudoinverse solution is susceptible to corruption by noise
because of division by small singular values. To avoid this problem, we use
a Tikhonov regularization parameter that suppresses the terms in Eq. \ref{eq:pinv}
with small singular values. The Tikhonov-regularized solution is given by
\begin{align}
  \bs{\hat{\Theta}}_{\bar{\bs{\nu}}} = \sum_{k=0}^R \frac{\sqrt{\mu_{\bar{\bs{\nu}}, k}}}{\mu_{\bar{\bs{\nu}}, k} + \eta}\textbf{u}_{\bar{\bs{\nu}}, k}\textbf{v}_{\bar{\bs{\nu}}, k}^{\dagger}\mb{G}_{\bar{\bs{\nu}}},  \label{eq:tik}
\end{align}
where $\eta$ is the regularization parameter.

Although we have derived the Tikhonov-regularized solution in a heuristic way in
this section, it is possible to show that Eq. \ref{eq:tik} is the solution to the
following optimization problem
\begin{align}
  \bs{\hat{\Theta}}_{\bar{\bs{\nu}}} = \argmin_{\bs{\Theta}_{\bar{\bs{\nu}}}}\left\{|| \mb{G}_{\bar{\bs{\nu}}} - \mb{H}_{\bar{\bs{\nu}}}\bs{\Theta}_{\bar{\bs{\nu}}}||^2 + \eta ||\bs{\Theta}_{\bar{\bs{\nu}}}||^2\right\}. 
\end{align}
In this form it is clear that increasing the regularization parameter $\eta$
favors solutions with a smaller norm.

After calculating $\bs{\hat{\Theta}}_{\bar{\bs{\nu}}}$ we can find the
object-space coefficients $\hat{\theta}_{\ttt{\tb{r}}_{\ttt{o}},\ttt{j}}$ with
an inverse DFT, and we have completed the inversion. In pseudocode the complete
reconstruction algorithm is given by\\
\begin{figure}[H]
\centering
\begin{minipage}{.65\linewidth}
\begin{algorithm}[H]
  \SetKwInOut{Input}{Input}
  \SetKwInOut{Output}{Output}
  \SetKwInOut{ForEach}{for each}
  \Input{Data array $g_{\tb{\ttt{r}}_{\ttt{d}},\ttt{p},\ttt{v}}$, regularization parameter $\eta$}
  \Output{Estimate of spatio-angular density coefficients $\hat{\theta}_{\ttt{\tb{r}}_{\ttt{o}},\ttt{j}}$}
  $\hat{\Theta}_{\bar{\bs{\nu}},\ttt{j}} \leftarrow \text{Zeros}()$\\
  $G_{\bar{\bs{\nu}},\ttt{p},\ttt{v}} \leftarrow \text{DFT}(g_{\tb{\ttt{r}}_{\ttt{d}},\ttt{p},\ttt{v}})$\\
  \textbf{for each} $\bar{\nu}$ in $\bar{\bs{\nu}}$:\\
  \qquad $H_{\ttt{p}, \ttt{v}, \ttt{j}} \leftarrow \text{CalculateH}(\bar{\nu}, \ttt{p}, \ttt{v}, \ttt{j})$\\
  \qquad $G_{\bar{\nu}, \ttt{q}} \leftarrow \text{Flatten}(G_{\bar{\nu}, \ttt{p},\ttt{v}}, [\ttt{p},\ttt{v}])$\\
  \qquad $H_{\ttt{q}, \ttt{j}} \leftarrow \text{Flatten}(H_{\ttt{p},\ttt{v}, \ttt{j}}, [\ttt{p},\ttt{v}])$\\
  \qquad $U_{\ttt{k}, \ttt{j}}, \mu_{\ttt{k}}, V_{\ttt{k},\ttt{q}} \leftarrow \text{SVD}(H_{\ttt{q}, \ttt{j}})$\\
  \qquad $\hat{\Theta}_{\bar{\nu},\ttt{j}} \leftarrow \sum_{\ttt{k}}^R\sum_{\ttt{q}}\frac{\sqrt{\mu_{\ttt{k}}}}{\mu_{\ttt{k}} + \eta}U_{\ttt{k},\ttt{j}}V_{\ttt{k},\ttt{q}}G_{\bar{\nu}, \ttt{q}} $\\
  $\hat{\theta}_{\ttt{\tb{r}}_{\ttt{o}},\ttt{j}} \leftarrow \text{DFT}^{-1}(\hat{\Theta}_{\bar{\bs{\nu}},\ttt{j}}).$
  \caption{diSPIM spatio-angular reconstruction}
\end{algorithm}
\end{minipage}
\end{figure}
Similar to the forward model calculation, we assemble
$H_{\ttt{p}, \ttt{v}, \ttt{j}}$ from precalculated matrices to avoid memory
limitations, we exploit the 8-fold symmetry of the kernel to reduce the number
of SVDs calculated in the loop, and we parallelize the loop over multiple CPUs.

\subsection{Summary parameters}
Algorithm 1 shows the complete spatio-angular reconstruction, but our choice to
reconstruct onto a spherical harmonic basis makes the result difficult to
interpret. Therefore, we need another set of computations that can convert the
spherical harmonic coefficients at each spatial point to interpretable values.
Recall that we cannot reconstruct directly to a spatio-angular density because
the result will not fit in memory---we need to reconstruct the spherical
harmonic coefficients then calculate summary parameters only as necessary.

The most informative (and expensive) set of summary parameters is to calculate
the complete orientation distribution function (ODF) from the set of spherical
harmonic coefficients. This computation is essentially an inverse spherical Fourier
transform given by
\begin{align}
  f_{\ttt{\tb{r}}_{\ttt{o}}}(\so) =  \sum_{\ttt{j}} \theta_{\ttt{\tb{r}}_{\ttt{o}},\ttt{j}} y_{\ttt{j}}(\mathbf{\hat{s}}_{o}). \label{eq:odf}
\end{align}
To calculate a visually informative ODF we need to evaluate
$f_{\ttt{\tb{r}}_{\ttt{o}}}(\so)$ at approximately 500 points, so calculating
the ODF amounts to a $15 \times 500$ matrix multiplication at each voxel. This
matrix multiplication is the most expensive computation in the entire pipeline,
and it is not feasible to compute it in every voxel for large data sets.
Typically we only calculate and visualize a subset of the ODFs---we either
restrict ourselves to an ROI or calculate and visualize ODFs in every $N$th
voxel.

Another useful summary parameter is the spatial density of fluorophores. For
angularly continuous functions the spatial density
$\rho_{\ttt{\tb{r}}_{\ttt{o}}}$ is related to the spatio-angular density
$f_{\ttt{\tb{r}}_{\ttt{o}}}(\so)$ by
\begin{align}
  \rho_{\ttt{\tb{r}}_{\ttt{o}}} = \int_{\mbb{S}^2}f_{\ttt{\tb{r}}_{\ttt{o}}}(\so). \label{eq:density}
\end{align}
The naive approach to computing $\rho_{\ttt{\tb{r}}_{\ttt{o}}}$ is to calculate
$f_{\ttt{\tb{r}}_{\ttt{o}}}(\so)$ using Eq. \ref{eq:odf} then apply Eq.
\ref{eq:density}. Instead we can calculate the density directly from the spherical
harmonic coefficients using
\begin{align}
  \rho_{\ttt{\tb{r}}_{\ttt{o}}} = \int_{\mbb{S}^2}\sum_{\ttt{j}} \theta_{\ttt{\tb{r}}_{\ttt{o}},\ttt{j}}y_{\ttt{j}}(\so), 
\end{align}
then rearranging the sum and integral
\begin{align}
  \rho_{\ttt{\tb{r}}_{\ttt{o}}} = \sum_{\ttt{j}} \theta_{\ttt{\tb{r}}_{\ttt{o}},\ttt{j}}\int_{\mbb{S}^2}y_{\ttt{j}}(\so), 
\end{align}
and finally noting that only the $l=0$ spherical harmonics have a non-zero
integral over the sphere
\begin{align}
  \rho_{\ttt{\tb{r}}_{\ttt{o}}} = \sum_{l}\theta_{\ttt{\tb{r}}_{\ttt{o}},\ttt{j}}\int_{\mbb{S}^2}y^0_{l}(\so). \label{eq:fd}
\end{align}
We can precompute the integrals then apply Eq. \ref{eq:fd} as a weighted sum
over just three spherical harmonic coefficients.

Another useful summary parameter is the generalized fractional anisotropy (GFA)
that was defined in (Tuch 2004) to extend the fractional anisotropy used in
diffusion tensor MRI to high-angular resolution MRI. Tuch defines the GFA as
\begin{align}
  \vartheta_{\ttt{\tb{r}}_{\ttt{o}}} = \frac{\text{StandardDeviation}\left\{f_{\ttt{\tb{r}}_{\ttt{o}}}(\so)\right\}}{\text{RootMeanSquare}\left\{f_{\ttt{\tb{r}}_{\ttt{o}}}(\so)\right\}}.\label{eq:gfa}
\end{align}
Similar to the density, the naive approach to computing
$\vartheta_{\ttt{\tb{r}}_{\ttt{o}}}$ would be to sample using Eq. \ref{eq:odf}
then apply Eq. \ref{eq:gfa}. We can improve on this by writing Eq. \ref{eq:gfa}
in terms of integrals then simplifying the integrals. We start with
\begin{align}
  \vartheta_{\ttt{\tb{r}}_{\ttt{o}}} = \sqrt{\frac{\ints{2}d\so\left[f(\so) - (\rho_{\ttt{\tb{r}}_{\ttt{o}}}/4\pi)\right]^2}{\ints{2}d\so\left[f(\so)\right]^2}},\\
\end{align}
then plug in the spherical harmonic coefficients
\begin{align}
  \vartheta_{\ttt{\tb{r}}_{\ttt{o}}} = \sqrt{\frac{\ints{2}d\so\left[\sum_{\ttt{j}}\theta_{\ttt{\tb{r}}_{\ttt{o}},\ttt{j}}y_{\ttt{j}}(\so) - (\rho_{\ttt{\tb{r}}_{\ttt{o}}}/4\pi)\right]^2}{\ints{2}d\so\left[\sum_{\ttt{j}} \theta_{\ttt{\tb{r}}_{\ttt{o}},\ttt{j}}y_{\ttt{j}}(\so)\right]^2}}.
\end{align}
After expanding the square brackets and exploiting the orthogonality of the
spherical harmonics we end up with
\begin{align}
  \vartheta_{\ttt{\tb{r}}_{\ttt{o}}} = \sqrt{\frac{\sum_{\ttt{j}} \theta^2_{\ttt{\tb{r}}_{\ttt{o}},\ttt{j}} + \rho^2_{\ttt{\tb{r}}_{\ttt{o}}}/2\pi + (\rho_{\ttt{\tb{r}}_{\ttt{o}}}/4\pi)^2}{\sum_{\ttt{j}} \theta^2_{\ttt{\tb{r}}_{\ttt{o}},\ttt{j}}}}.
\end{align}

The final summary parameter we would like to calculate is the direction along
which the most fluorophores are oriented
\begin{align}
  \mathbf{\hat{s}}_{\ttt{\tb{r}}_{\ttt{o}}, \text{max}} =  \argmax_{\so} f_{\ttt{\tb{r}}_{\ttt{o}}}(\so). 
\end{align}
Our problem is to find the maximum point of a function on a sphere given its
spherical harmonic coefficients. If we can tolerate an angular error of
$\Delta \mathbf{\hat{s}}$ steradians, then the brute-force approach is to sample
the sphere at $4\pi/\Delta \mathbf{\hat{s}}$ approximately equally spaced points
then choose the maximum. 

A faster approach is to use a simple optimization scheme that first samples
coarsely over the sphere then performs a gradient descent on the best of the
coarse samples. In general this type of scheme is not guaranteed to find the
global maximum, but I expect that a wise choice of initial sample points will
guarantee the global maximum for our bandlimited signals.

Currently I am calculating all of the summary statistics using the naive
approaches, and these calculations are the bottle neck in the reconstruction
pipeline. I will implement the improved routines soon.

\begin{table}[h]
\centering
\begin{tabular}{lll}\toprule
  Parameter name & Symbol & Interpretation\\
  \midrule
  Orientation distribution function (ODF) & $f_{\ttt{\tb{r}}_{\ttt{o}}}(\so)$ & Number of fluorophores at $\ttt{\tb{r}}_{\ttt{o}}$ oriented along $\so$.\\
  Density & $\rho_{\ttt{\tb{r}}_{\ttt{o}}}$ & Number of fluorophores at $\ttt{\tb{r}}_{\ttt{o}}$. \\
  Generalized fractional anisotropy (GFA) & $\vartheta_{\ttt{\tb{r}}_{\ttt{o}}}$ & A scale-invariant statistic of the angular spread of\\
                 & & fluorophores at $\ttt{\tb{r}}_{\ttt{o}}$. $\vartheta_{\ttt{\tb{r}}_{\ttt{o}}} = 1$ for very anisotropic distributions\\
                 & & and $\vartheta_{\ttt{\tb{r}}_{\ttt{o}}} = 0$ for isotropic fluorophores.  \\
  Peak direction (mode) & $\mathbf{\hat{s}}_{\ttt{\tb{r}}_{\ttt{o}}, \text{max}}$ & Direction along which most fluorophores are oriented at $\ttt{\tb{r}}_{\ttt{o}}$. \\
  
  \bottomrule
\end{tabular}
\caption{Summary of angular summary statistics.}
\end{table}

\section{Numerical results} \label{sec:num}
\subsection{Bead studies}
Figures \ref{fig:beadx}--\ref{fig:beadiso} show the results of numerical bead
studies that we used to debug and verify our forward model. We placed a single
bead in the center of a small field-of-view then applied our forward model to
generate data that would be collected under the four polarizer settings and two
viewing directions.

First, we generated four numerical phantoms---$\mh{x}$, $\mh{y}$, and $\mh{z}$
oriented beads and an isotropic bead. The continuous spatio-angular densities of
these phantoms are given by
\begin{align}
  f_1(\ro, \so) &= \delta(\ro)\, \delta(\so - \mh{x}), \\
  f_2(\ro, \so) &= \delta(\ro)\, \delta(\so - \mh{y}), \\
  f_3(\ro, \so) &= \delta(\ro)\, \delta(\so - \mh{z}), \\
  f_4(\ro, \so) &= \delta(\ro).
\end{align}
Next, we set the FOV by choosing the number of voxels (32 per side) and the
voxel dimensions (130 nm per side). Finally, we calculated the object-space
coefficients $\theta_{\ttt{\tb{r}}_{\ttt{o}},\ttt{j}}$ from the spatio-angular
density using
\begin{align}
  \theta_{\ttt{\tb{r}}_{\ttt{o}},\ttt{j}} = \ints{2}d\so\, f(\ro - \epsilon\ttt{\tb{r}}_{\ttt{o}},\so)y_{\ttt{j}}(\so). \label{eq:spang2co}
\end{align}

Now that we have a numerical phantom, the next step is to specify a microscope.
We used the following parameters for our numerical simulations
\begin{align}
  \text{NA}_{\text{A}} &= 1.1,\\
  \text{NA}_{\text{B}} &= 0.71,\\
  \lambda_{\text{em}} &= 525\, \text{nm} \leftarrow \text{Alexa 488 emission},\\
  n &= 1.33 \leftarrow \text{water}.
\end{align}
We use these parameters when we calculate the elements of
$H_{\bar{\bs{\nu}},\ttt{p},\ttt{v},\ttt{j}}$. Finally, we apply Eq.
\ref{eq:fastfwd} to calculate the data
$g_{\tb{\ttt{r}}_{\ttt{d}},\ttt{p},\ttt{v}}$. Figures
\ref{fig:beadx}--\ref{fig:beadiso} show summary views of the five-dimensional
datasets $g_{\tb{\ttt{r}}_{\ttt{d}},\ttt{p},\ttt{v}}$. For each polarizer
setting and view we plot the maximum intensity projection along each of the
three axes.

Although the entire kernel is specified exactly in section 1, it is useful to
have a few rules of thumb in mind when you look at the bead results:
\begin{enumerate}
\item The probability of exciting a dipole is proportional to
  $|\mh{p}\cdot\bs{\mu}|^2 = \cos^2\theta$ where $\mh{p}$ is the polarizer
  orientation and $\bs{\mu}$ is the excitation dipole moment---lots of
  excitation when the polarization is parallel to the dipole moment.
\item A dipole emitter radiates along a direction $\mh{s}$ with a power
  proportional to $1 - |\mh{s}\cdot\bs{\mu}|^2 = \sin^2\theta$---lots of
  radiation in the plane perpendicular to the dipole moment.
\item An objective with NA $>$ 0 will collect the radiation from a cone of
  $\mh{s}$ directions.
\item When a dipole moment is parallel to the detection optical axis, then the
  light in the back focal plane will be radially polarized and the resulting
  image will look like a doughnut.
\end{enumerate}

\fig{../figures/beads/data0.pdf}{1.0}{Simulated data from a bead with $\mh{x}$-oriented
  dipole moments.}{beadx}
\fig{../figures/beads/data1.pdf}{1.0}{Simulated data from a bead with
  $\mh{y}$-oriented dipole moments.}{}
\fig{../figures/beads/data2.pdf}{1.0}{Simulated
  data from a bead with $\mh{z}$-oriented dipole moments.}{bead4}
\fig{../figures/beads/data3.pdf}{1.0}{Simulated data from an isotropic bead.}{beadiso}

We have made several assumptions about the axial part of the kernel. First, we
assume that the axial part can be described by a Gaussian. The axial part would
be more accurately described by a widefield fluorescence kernel convolved with
the Gaussian width of the light sheet, but we expect that the width of the light
sheet dominates the kernel so a pure Gaussian is appropriate. Second, we have
assumed that the width of the light sheet is constant which allows us to
describe the system as shift-invariant. This assumption has been used in
previous diSPIM deconvolutions, and it allows us to keep the computations
tractable. Finally, we have chosen the width of the axial Gaussian to be
$\sim 3\times$ broader than the transverse width of the transverse kernel of an
isotropic bead. This width was chosen to match the axial FWHM of the kernels
supplied by Min in $\ttt{PSFA\_1p1NA.tif}$ and $\ttt{PSFB\_0p71NA.tif}$.
\textcolor{blue}{\textbf{Do you think these assumptions are appropriate? Do the
    bead PSFs look reasonable?}}

\subsection{Helix studies}
We created a more complex spatio-angular phantom to test our forward model and
inversion scheme. The helix phantom is shown in Figure \ref{fig:phant}---it
consists of three helices with different coil directions and dipoles that are
aligned along the tangent of the helix at each point. On one end of the helix
the dipoles are almost completely aligned along the tangent, and at the other
end of the helix the dipoles are isotropic.

We would like to express the helix phantom as a continuous spatio-angular
density. First, we specify a single helix with a parameterized line
\begin{align}
  c(t) = (r_{\text{h}}\cos t, r_{\text{h}}\sin t, p_{\text{h}} t)\quad \text{for}\, t \in [-2\pi, 2\pi].
\end{align}
Next, for each point in space $\ro$ we find the closest point on the helix then
find the corresponding parameter $t$ and name it $t_{\ro}^*$
\begin{align}
t_{\ro}^* = \text{min}_t||c(t) - \ro||^2. 
\end{align}
Next, we build the complete helix using
\begin{align}
  f_{\text{helix}}(\ro, \so) =
  \begin{cases}
    w(\ro, \so, c'(t_{\ro}^*), \kappa(t_{\ro}^*)), & t_{\ro}^* < r_{\text{cyl}} \\
    0 & t_{\ro}^* < r_{\text{cyl}}
  \end{cases}\label{eq:helix}
\end{align}
where
\begin{align}
  w(\ro, \so, \mh{n}, \kappa) &= \frac{1}{4\pi{}_1F_1(1/2,3/2,\kappa)}e^{\kappa(\mh{n}\cdot\so)},\label{eq:watson}\\
  \kappa(t) &= kt + k_o, \label{eq:kappa}
\end{align}
and ${}_1F_1(1/2,3/2,\kappa)$ is the Kummer confluent hypergeometric function.
Eq. \ref{eq:helix} restricts the phantom to within $r_{\text{cyl}}$ of the
helix, Eq. \ref{eq:watson} is a normalized Watson distribution with spread
parameter $\kappa$, and Eq. \ref{eq:kappa} linearly increases the spread parameter as we move along the helix.

Finally, we assemble three helices
\begin{align}
  f_{\text{phantom}}(\ro, \so) = f_{\text{helix}}(\ro - \mathbf{\mathfrak{r}}_0, \so) + \mc{R}_x\left\{f_{\text{helix}}(\ro - \mathbf{\mathfrak{r}}_1, \so)\right\} + \mc{R}_y\left\{f_{\text{helix}}(\ro - \mathbf{\mathfrak{r}}_2, \so)\right\}, 
\end{align}
where $\mathcal{R}_x$ is a rotation operator that rotates the helix direction to
along the $\mh{x}$ direction. For our simulations we chose the following set of parameters
\begin{align}
  r_{\text{h}} &= 700\ \text{nm},\\
  p_{\text{h}} &= 1000\ \text{nm},\\
  \mathbf{\mathfrak{r}}_0 &= (-2080,-2080,2080)\ \text{nm},\\
  \mathbf{\mathfrak{r}}_1 &= (0,0,0)\ \text{nm},\\
  \mathbf{\mathfrak{r}}_2 &= (2080,2080,2080)\ \text{nm},\\
  r_{\text{cyl}} &= 250\ \text{nm},\\
  k &= \frac{5}{4\pi},\\
  k_0 &= 2\pi.
\end{align}
We calculated the coefficients of the object using Eq. \ref{eq:spang2co}, then
simulated the noise-free data with the same imaging parameters as the previous
section. The data we generated is shown in Figure \ref{fig:data}.

We applied our spatio-angular restoration algorithm to the noise-free data and
recovered the object in Figure \ref{fig:recon}. Alternatively, we can interpret
Figure \ref{fig:recon} as the measurement-space component of the helix
phantom---we applied the forward model then reconstructed with the pseudoinverse
solution which is identical to projection onto the measurement space (Barrett
Table 1.2).

Finally, we simulated data using the helix phantom with Poisson noise at SNR =
10 (the maximum number of photons is 100), then attempted to reconstruct the
object with several different Tikhonov regularization parameters. Figure
\ref{fig:recontik} shows that the mean square error (MSE) between the phantom
and the reconstructed object has a minimum at a specific regularization
parameter as we would expect. The optimal regularization parameter will change
as we vary the SNR, but we can use Figure \ref{fig:recontik} as evidence that
the inversion algorithm is working well.

\fig{../figures/helix/phantom.pdf}{1.0}{Helix phantom summary.}{phant}
\fig{../figures/helix/data.pdf}{1.0}{Data generated by the helix phantom.}{data}
\fig{../figures/helix/phantom-recon.pdf}{1.0}{Helix phantom reconstruction.}{recon}
\fig{../figures/helix-noise/mse.pdf}{0.5}{Mean square error of helix phantom reconstruction with SNR = 10.}{recontik}
  % filename, width, caption, label

\section{Experimental results} \label{sec:exp}

We collected several data sets with the polarized light asymmetric diSPIM (1.1
and 0.71 NA). In this section we will review the preprocessing, calibration, and
reconstructed results.

\subsection{Min's preprocessing}
Min applies the following preprocessing steps to the raw data
\begin{enumerate}
\item ``Deshearing'' the data to correct for stage scanning.
\item ROI selection to reduce the size of the data.
\item ``Isometrization'' to convert all of the data to isometric voxels.
\item Registration of the two views so that they are in the same coordinate
  system.
\end{enumerate}

\subsection{Talon's preprocessing}
I applied the following preprocessing steps to prepare Min's preprocessed data for
spatio-angular restoration.
\begin{enumerate}
\item Background correction.
\item Correcting the views for power differences.
\item Correcting the views for voxel size differences.
\item Correcting the volumes (each view and polarization) with polarization
  calibration data.
\item TODO: I will apply a bleaching correction based on the first and last
  polarizer setting. The data in these notes had negligible bleaching, so I
  haven't needed this correction yet.
\end{enumerate}

For the background correction I subtracted the mean of a background ROI from all
of the data. I applied the same background correction to the volumes from each
polarizer setting, but I applied different background corrections to the data
from each view.

To correct the views for power differences I multiplied the data from the
high-NA view by a factor of $(1.1/0.71)^2 = 2.4$. This correction models the fact the we
balanced the input power of the two views so that the pixels from each view
would collect approximately the same number of photons.

Next, I applied a correction to account for the different voxel sizes of the raw
data. Min's preprocessing outputs isometric voxels, but it does not preserve the
total number of counts before and after the correction. For example, if we put a
single (100, 100, 200) nm voxel with a value of 50 into Min's preprocessing
algorithm, the output would be two (100, 100, 100) nm voxels each with a value
of 50. I am applying a correction factor so that the output would be two (100,
100, 100) nm voxels each with a value of 25.

In the GUV dataset the raw voxel size in view A is (130, 130, 549)
nm, and the raw voxel size in view B is (227, 227, 835.8) nm. To correct for
this difference I multiply the data in view A by a factor of
$(227/130)^2(835.8/549)$ = 4.64. \textcolor{blue}{\textbf{Min, Hari---does this make sense?}}

Finally, I applied a correction factor to each volume to account for
imperfections in the polarized illumination due to misalignment or polarization
errors introduced during reflection by the dichroics. To collect calibration
data Min imaged the center of a fluorescent lake of uniformly oriented
fluorophores under each polarizer setting. He also collected data from the epi
path and the imaging path for a total of 16 volumes (epi/imaging, 4 polarizer
settings, 2 views). I manually selected an ROI in each of these 16 volumes and
stored the mean value of these ROIs. Next, I applied correction factors to
correct for errors on the illumination path. We would expect that the intensity
measured from the epi path should be constant as the polarization changes, so I
applied correction factors to account for any deviation from a constant---the
largest correction factor was approximately 5\%. Finally, I applied correction
factors to correct for errors on the detection path. We have a model that
predicts the data we would expect to collect from a uniform lake of uniformly
oriented fluorophore---namely, the transfer function evaluated at zero
spatio-angular frequency $H_{0,\tv}^0(0,\mh{p})$. I compared the calibration
data to our model and applied correction factor to account for any
differences---the largest correction factor was approximately 10\%.
\textcolor{blue}{\textbf{Min, Hari---does this make sense?}}

I applied each of the correction factors to the data, but I could have applied
the correction factors to the model. I have chosen to correct the data to make
the comparison with the simulated data in the previous section easier. 

\subsection{GUV study}
Mai prepared a sample of giant unilamellar vesicles (GUVs) labelled with Alexa
Fluor 488. From two-dimensional fluorescence orientation studies we expect to
observe the dipole moments oriented normal to the surface of the vesicles.

Figure \ref{fig:guv-data} shows the corrected data and Figure
\ref{fig:guv-recon} shows the reconstructed spatio-angular density. The
reconstructed peak orientations are approximately normal to the sphere's
surface, although some orientations are not reconstructed well---the
``degeneracy'' between $\mh{x} + \mh{z}$ and $\mh{x} - \mh{z}$ oriented dipoles
is evident in the $xz$ projection of the peaks. In particular, it's
clear that the density is not recovered well for dipoles oriented along
$\mh{x} - \mh{z}$ at the points farthest from the cover slip. I suspect that
these portions of the object are in the null space of the microscope, but I am
still investigating. A GUV phantom will help us understand this issue.

\fig{../figures/guv-recon/data-corrected.pdf}{1.0}{GUV data after corrections.}{guv-data}
\vspace{-1em}
\fig{{../figures/guv-recon/guv-recon-1.0e+00}.pdf}{1.0}{GUV reconstruction with $\eta = 1$.}{guv-recon}
\vspace{-2em}
\subsection{Xylem study}
Tobias prepared a sample of xylem labeled with mScarlet. From two-dimensional
fluorescence orientation studies we expect to observe the dipole moments
oriented parallel to the helices.

Figure \ref{fig:xylem-data} shows the corrected data and Figure
\ref{fig:xylem-recon} shows the reconstructed spatio-angular density. The
reconstructed peak orientations are approximately parallel to the helix surface,
although once again the ``degeneracy'' between $\mh{x} + \mh{z}$ and
$\mh{x} - \mh{z}$ oriented dipoles is evident in the $xz$ projection of the
peaks. The $yz$ projection of the peak and ODFs is particularly promising---the
reconstructed orientations are clearly parallel to the helix around the whole
circle of orientations.

One part of the helix near the center of the $xy$ projection is nearly absent in
the data and the reconstruction. It is not clear to me why this region did not
give a larger signal in the $-\mh{x} + \mh{y}$ polarizer data since the helix is
parallel to the $-\mh{x} + \mh{y}$ direction in this region. In the numerical
phantom we successfully recovered all of the orientations in the phantom, so
it's not clear to me what is causing this problem. Reconstructing larger volumes
may point us in the right direction.

\fig{../figures/xylem-recon/data-corrected.pdf}{1.0}{Xylem data after corrections.}{xylem-data}
\fig{{../figures/xylem-recon/guv-recon-1.0e+00}.pdf}{1.0}{Xylem reconstruction with $\eta = 1$.}{xylem-recon}
\vspace{-2em}
\section{Next priorities}
Here are some of my next tasks in approximate order of priority. Feedback is
always welcome!
\begin{itemize}
\item Optimize the summary statistics calculations. These optimizations will make
  the full 1000 $\times$ 1000 $\times$ 1000 voxel reconstructions feasible.
\item Directly compare the spatial resolution of the (1) raw data (2) density
  after spatio-angular reconstruction (3) spatial deconvolution of data
  averaged over polarizations.
\item Continue reconstructing our existing data---the two color data in Gantt's
  cells, Bob's cells from last year, and the remainder of the helix data.
\item Draft a theory article for JOSA A or Optics Express. Working title:
  ``Spatio-angular transfer functions for fluorescence microscopes I. Basic
  theory''.
\end{itemize}


\end{document}

